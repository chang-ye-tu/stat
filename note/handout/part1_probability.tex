\documentclass[12pt,a4paper]{article}
\usepackage[left=1cm,right=1cm,bottom=15mm,top=20mm]{geometry}
\usepackage[AutoFakeBold,AutoFakeSlant]{xeCJK}
\setCJKmainfont[AutoFakeSlant=.1,AutoFakeBold=2]{Noto Serif CJK TC}
\usepackage{amsmath,amsthm,amssymb,amsfonts}
\usepackage{graphicx,xcolor,float}
\usepackage{booktabs,tabularx,multirow}
\usepackage{enumitem}
\usepackage{parskip}
\setlist{itemsep=0pt,parsep=0pt}
\usepackage{hyperref}
\hypersetup{
    colorlinks=true,
    linkcolor=blue!70!black,
    urlcolor=blue!80!black
}

% 定理環境
\theoremstyle{definition}
\newtheorem{definition}{定義}[section]
\newtheorem{example}{例題}[section]
\newtheorem{exercise}{習題}[section]
\newtheorem{theorem}{定理}[section]
\newtheorem{lemma}{引理}[section]
\newtheorem{corollary}{推論}[section]
\newtheorem{proposition}{命題}[section]
\newtheorem{property}{性質}[section]
\newtheorem*{remark}{註}
\newtheorem*{solution}{解答}
\newtheorem*{note}{說明}
\newtheorem*{prf}{證明}

% 常用指令
\newcommand{\ds}{\displaystyle}
\newcommand{\ie}{\;\Longrightarrow\;}
\newcommand{\ifff}{\;\Longleftrightarrow\;}
\newcommand\expc{\mathsf{E}}
\DeclareMathOperator\var{var}
\DeclareMathOperator\cov{cov}
\DeclareMathOperator\corr{corr}
\newcommand\prb{\mathsf{P}}
\newcommand{\Real}{\mathbb{R}}
\newcommand{\Nat}{\mathbb{N}}

% 頁面設定
\renewcommand{\figurename}{圖}
\renewcommand{\tablename}{表}
\renewcommand{\proofname}{\textbf{證明}}

\usepackage{fancyhdr}
\pagestyle{fancy}
\fancyhf{}
\fancyhead[L]{統計學講義}
\fancyhead[R]{第一部分:機率基礎}
\fancyfoot[C]{\thepage}
\renewcommand{\headrulewidth}{0.4pt}
\renewcommand{\footrulewidth}{0.4pt}

\title{\vspace{-2cm}\textbf{統計學講義}\\[3mm] \Large 第一部分:機率基礎}
\author{}
\date{\vspace{-2cm}}

\begin{document}
\maketitle
\thispagestyle{fancy}

\begin{center}
\fbox{\parbox{0.9\textwidth}{\centering
\textbf{參考書籍}\\[2mm]
Jeffrey S. Rosenthal, \textit{Probability and Statistics: The Science of Uncertainty}, 2nd Edition\\
Chapter 2: Probability\\[2mm]
\url{https://utstat.utoronto.ca/mikevans/jeffrosenthal/}
}}
\end{center}

%\tableofcontents

%=============================================================================
\section{機率公理與基本定義}
%=============================================================================

機率論是統計學的數學基礎。本節介紹機率的嚴格數學定義與基本性質。

\subsection{樣本空間與事件}

\begin{definition}[樣本空間]
\textbf{樣本空間}(Sample Space),記為 $\Omega$,是一次隨機實驗所有可能結果的集合。樣本空間中的每個元素稱為\textbf{樣本點}(sample point)。
\end{definition}

\begin{example}
不同隨機實驗的樣本空間:
\begin{enumerate}[label=(\alph*)]
    \item 擲一枚公正硬幣:$\Omega = \{\text{正面}, \text{反面}\}$ 或簡記為 $\Omega = \{H, T\}$
    \item 擲一顆公正骰子:$\Omega = \{1, 2, 3, 4, 5, 6\}$
    \item 測量燈泡壽命:$\Omega = [0, \infty)$(連續樣本空間)
    \item 計算某日顧客人數:$\Omega = \{0, 1, 2, 3, \ldots\} = \Nat \cup \{0\}$
\end{enumerate}
\end{example}

\begin{definition}[事件]
\textbf{事件}(Event)是樣本空間 $\Omega$ 的子集合。若隨機實驗的結果落在事件 $A$ 中,我們說「事件 $A$ 發生了」。
\end{definition}

\begin{example}
擲骰子的事件:
\begin{itemize}
    \item $A = \{2, 4, 6\}$:「擲出偶數」
    \item $B = \{1, 2, 3\}$:「擲出小於 4 的數」
    \item $A \cap B = \{2\}$:「擲出偶數且小於 4」
    \item $A \cup B = \{1, 2, 3, 4, 6\}$:「擲出偶數或小於 4」
    \item $A^c = \{1, 3, 5\}$:「擲出奇數」($A$ 的補集)
\end{itemize}
\end{example}

\begin{definition}[事件的集合運算]
設 $A$ 和 $B$ 為樣本空間 $\Omega$ 的兩個事件:
\begin{itemize}
    \item \textbf{聯集}:$A \cup B = \{\omega \in \Omega : \omega \in A \text{ 或 } \omega \in B\}$
    \item \textbf{交集}:$A \cap B = \{\omega \in \Omega : \omega \in A \text{ 且 } \omega \in B\}$
    \item \textbf{補集}:$A^c = \{\omega \in \Omega : \omega \notin A\}$
    \item \textbf{差集}:$A \setminus B = A \cap B^c = \{\omega \in \Omega : \omega \in A \text{ 且 } \omega \notin B\}$
\end{itemize}
\end{definition}

\begin{definition}[互斥事件]
若 $A \cap B = \varnothing$(空集合),則稱事件 $A$ 與 $B$ \textbf{互斥}(mutually exclusive)或\textbf{不相交}(disjoint)。
\end{definition}

\subsection{機率公理}

\begin{definition}[Kolmogorov 機率公理]
\textbf{機率}是一個從事件到實數的函數 $P$,滿足以下三條公理:
\begin{enumerate}[label=(\roman*)]
    \item \textbf{非負性}:對任意事件 $A$,$\prb(A) \geqslant 0$
    \item \textbf{規範性}:$\prb(\Omega) = 1$
    \item \textbf{可加性}:若 $A_1, A_2, A_3, \ldots$ 為兩兩互斥的事件(即 $A_i \cap A_j = \varnothing$ 當 $i \neq j$),則
    \[
    P\left(\bigcup_{i=1}^{\infty} A_i\right) = \sum_{i=1}^{\infty} \prb(A_i)
    \]
\end{enumerate}
\end{definition}

\begin{remark}
這三條公理是機率論的基石。所有機率的性質都可由這三條公理推導出來。
\end{remark}

\subsection{機率的基本性質}

由機率公理可推導出以下基本性質:

\begin{theorem}[機率的基本性質]\label{thm:prob-properties}
設 $A$ 和 $B$ 為任意事件,則:
\begin{enumerate}[label=(\alph*)]
    \item $\prb(\varnothing) = 0$
    \item $\prb(A^c) = 1 - \prb(A)$
    \item 若 $A \subseteq B$,則 $\prb(A) \leqslant \prb(B)$
    \item $0 \leqslant \prb(A) \leqslant 1$
    \item $\prb(A \cup B) = \prb(A) + \prb(B) - \prb(A \cap B)$(加法法則)
\end{enumerate}
\end{theorem}

\begin{prf}
我們證明其中幾個重要性質:

\textbf{(a) 證明 $\prb(\varnothing) = 0$:}

令 $A_1 = \Omega$,$A_2 = A_3 = \cdots = \varnothing$。由於這些事件兩兩互斥,且 $\bigcup_{i=1}^{\infty} A_i = \Omega$,由公理 (iii):
\[
\prb(\Omega) = \prb(A_1) + \prb(A_2) + \prb(A_3) + \cdots = \prb(\Omega) + \prb(\varnothing) + \prb(\varnothing) + \cdots
\]
由公理 (ii),$\prb(\Omega) = 1$。若 $\prb(\varnothing) > 0$,則右式為無窮大,矛盾。故 $\prb(\varnothing) = 0$。

\textbf{(b) 證明 $\prb(A^c) = 1 - \prb(A)$:}

由於 $A$ 與 $A^c$ 互斥且 $A \cup A^c = \Omega$,由公理 (ii) 和 (iii):
\[
1 = \prb(\Omega) = \prb(A \cup A^c) = \prb(A) + \prb(A^c)
\]
移項得 $\prb(A^c) = 1 - \prb(A)$。

\textbf{(e) 證明加法法則:}

將 $A \cup B$ 分解為三個互斥部分:
\[
A \cup B = (A \cap B^c) \cup (A \cap B) \cup (A^c \cap B)
\]
因此:
\[
\prb(A \cup B) = \prb(A \cap B^c) + \prb(A \cap B) + \prb(A^c \cap B)
\]

同時,$A = (A \cap B^c) \cup (A \cap B)$,故 $\prb(A) = \prb(A \cap B^c) + \prb(A \cap B)$。

類似地,$\prb(B) = \prb(A^c \cap B) + \prb(A \cap B)$。

將以上兩式相加:
\[
\prb(A) + \prb(B) = \prb(A \cap B^c) + \prb(A^c \cap B) + 2\prb(A \cap B)
\]

因此:
\[
\prb(A \cup B) = \prb(A) + \prb(B) - \prb(A \cap B)
\]
\end{prf}

\begin{corollary}[排容原理]
對於三個事件 $A$、$B$、$C$:
\begin{align*}
\prb(A \cup B \cup C) &= \prb(A) + \prb(B) + \prb(C) \\
&\quad - \prb(A \cap B) - \prb(A \cap C) - \prb(B \cap C) \\
&\quad + \prb(A \cap B \cap C)
\end{align*}
\end{corollary}

\begin{example}
一副 52 張撲克牌中隨機抽取一張。設事件 $A$ =「抽到紅心」,$B$ =「抽到人頭牌(J, Q, K)」。求 $\prb(A \cup B)$。
\end{example}

\begin{solution}
\begin{itemize}
  \item[]
    \item $\prb(A) = \dfrac{13}{52} = \dfrac{1}{4}$(紅心有 13 張)
    \item $\prb(B) = \dfrac{12}{52} = \dfrac{3}{13}$(人頭牌每種花色 3 張,共 12 張)
    \item $\prb(A \cap B) = \dfrac{3}{52}$(紅心的人頭牌有 3 張)
\end{itemize}
由加法法則:
\[
\prb(A \cup B) = \frac{13}{52} + \frac{12}{52} - \frac{3}{52} = \frac{22}{52} = \frac{11}{26}
\]
\end{solution}

%=============================================================================
\section{條件機率與獨立性}
%=============================================================================

\subsection{條件機率的定義}

在許多情況下,我們想知道「在某事件已發生的條件下,另一事件發生的機率」。

\begin{definition}[條件機率]
設 $A$ 和 $B$ 為兩事件,且 $\prb(B) > 0$。在事件 $B$ 發生的條件下,事件 $A$ 發生的\textbf{條件機率}(conditional probability)定義為:
\[
\prb(A|B) = \frac{\prb(A \cap B)}{\prb(B)}
\]
\end{definition}

\begin{remark}
直觀理解:條件機率 $\prb(A|B)$ 是「在 $B$ 已發生的前提下,$A$ 也發生」的機率。分母 $\prb(B)$ 將樣本空間「縮小」到 $B$,分子 $\prb(A \cap B)$ 則是 $A$ 與 $B$ 同時發生的部分。
\end{remark}

\begin{example}
擲兩顆公正骰子。設 $A$ =「兩骰子點數和為 8」,$B$ =「第一顆骰子為 3」。求 $\prb(A|B)$。
\end{example}

\begin{solution}
樣本空間共有 $6 \times 6 = 36$ 個等機率的樣本點。

\begin{itemize}
    \item $B = \{(3,1), (3,2), (3,3), (3,4), (3,5), (3,6)\}$,故 $\prb(B) = \dfrac{6}{36} = \dfrac{1}{6}$
    \item $A \cap B = \{(3,5)\}$(第一顆為 3,和為 8,故第二顆為 5),故 $\prb(A \cap B) = \dfrac{1}{36}$
\end{itemize}

因此:
\[
\prb(A|B) = \frac{\prb(A \cap B)}{\prb(B)} = \frac{1/36}{1/6} = \frac{1}{6}
\]

\textbf{驗證}:在 $B$ 發生的條件下(第一顆為 3),樣本空間縮小為 6 個點。其中只有 $(3,5)$ 使和為 8,故條件機率為 $1/6$。
\end{solution}

\subsection{乘法法則}

\begin{theorem}[乘法法則]
由條件機率的定義可立即得到:
\[
\prb(A \cap B) = \prb(A|B) \cdot \prb(B) = \prb(B|A) \cdot \prb(A)
\]
推廣到多個事件:
\[
\prb(A_1 \cap A_2 \cap \cdots \cap A_n) = \prb(A_1) \cdot \prb(A_2|A_1) \cdot \prb(A_3|A_1 \cap A_2) \cdots \prb(A_n|A_1 \cap \cdots \cap A_{n-1})
\]
\end{theorem}

\begin{example}
袋中有 5 顆紅球和 3 顆白球。不放回地連續抽取兩球。求兩球都是紅球的機率。
\end{example}

\begin{solution}
設 $A_1$ =「第一球為紅」,$A_2$ =「第二球為紅」。

\[
\prb(A_1 \cap A_2) = \prb(A_1) \cdot \prb(A_2|A_1) = \frac{5}{8} \times \frac{4}{7} = \frac{20}{56} = \frac{5}{14}
\]

說明:第一球為紅的機率是 $5/8$。若第一球為紅,剩下 7 球中有 4 顆紅球,故條件機率為 $4/7$。
\end{solution}

\subsection{事件的獨立性}

\begin{definition}[獨立事件]
若事件 $A$ 和 $B$ 滿足
\[
\prb(A \cap B) = \prb(A) \cdot \prb(B)
\]
則稱 $A$ 與 $B$ \textbf{(統計)獨立}(independent)。
\end{definition}

\begin{proposition}
若 $\prb(B) > 0$,則 $A$ 與 $B$ 獨立等價於
\[
\prb(A|B) = \prb(A)
\]
\end{proposition}

\begin{prf}
由條件機率定義:
\[
\prb(A|B) = \frac{\prb(A \cap B)}{\prb(B)}
\]
若 $A$ 與 $B$ 獨立,則 $\prb(A \cap B) = \prb(A) \cdot \prb(B)$,故:
\[
\prb(A|B) = \frac{\prb(A) \cdot \prb(B)}{\prb(B)} = \prb(A)
\]
反之亦然。
\end{prf}

\begin{remark}
獨立性的直觀意義:$B$ 的發生與否不影響 $A$ 發生的機率。這是機率論中極為重要的概念。
\end{remark}

\begin{example}
擲一顆公正骰子。設 $A$ =「點數為偶數」,$B$ =「點數小於 4」。判斷 $A$ 與 $B$ 是否獨立。
\end{example}

\begin{solution}
\begin{itemize}
  \item[]
    \item $A = \{2, 4, 6\}$,$\prb(A) = 3/6 = 1/2$
    \item $B = \{1, 2, 3\}$,$\prb(B) = 3/6 = 1/2$
    \item $A \cap B = \{2\}$,$\prb(A \cap B) = 1/6$
\end{itemize}

檢驗:$\prb(A) \cdot \prb(B) = \dfrac{1}{2} \times \dfrac{1}{2} = \dfrac{1}{4} \neq \dfrac{1}{6} = \prb(A \cap B)$

因此 $A$ 與 $B$ \textbf{不獨立}。
\end{solution}

\begin{theorem}[獨立性的等價條件]
若 $A$ 與 $B$ 獨立,則以下各對事件也獨立:
\begin{enumerate}[label=(\alph*)]
    \item $A$ 與 $B^c$
    \item $A^c$ 與 $B$
    \item $A^c$ 與 $B^c$
\end{enumerate}
\end{theorem}

\begin{prf}
證明 (a):
\begin{align*}
\prb(A \cap B^c) &= \prb(A) - \prb(A \cap B) \quad \text{(因 $A = (A \cap B) \cup (A \cap B^c)$ 且互斥)}\\
&= \prb(A) - \prb(A) \cdot \prb(B) \quad \text{(因 $A$ 與 $B$ 獨立)}\\
&= \prb(A)(1 - \prb(B))\\
&= \prb(A) \cdot \prb(B^c)
\end{align*}
故 $A$ 與 $B^c$ 獨立。(b) 和 (c) 的證明類似。
\end{prf}

\begin{definition}[多事件的獨立性]
事件 $A_1, A_2, \ldots, A_n$ 稱為\textbf{相互獨立}(mutually independent),若對任意 $k \geqslant 2$ 個事件的子集 $\{A_{i_1}, A_{i_2}, \ldots, A_{i_k}\}$,都有:
\[
\prb(A_{i_1} \cap A_{i_2} \cap \cdots \cap A_{i_k}) = \prb(A_{i_1}) \cdot \prb(A_{i_2}) \cdots \prb(A_{i_k})
\]
\end{definition}

\begin{remark}
注意:兩兩獨立(pairwise independent)不蘊含相互獨立(mutually independent)。相互獨立的條件更強。
\end{remark}

%=============================================================================
\section{貝氏定理與全機率定理}
%=============================================================================

\subsection{樣本空間的分割}

\begin{definition}[分割]
若事件 $B_1, B_2, \ldots, B_n$ 滿足:
\begin{enumerate}[label=(\roman*)]
    \item $B_i \cap B_j = \varnothing$ 當 $i \neq j$(兩兩互斥)
    \item $B_1 \cup B_2 \cup \cdots \cup B_n = \Omega$(覆蓋整個樣本空間)
    \item $\prb(B_i) > 0$ 對所有 $i$
\end{enumerate}
則稱 $\{B_1, B_2, \ldots, B_n\}$ 為樣本空間 $\Omega$ 的一個\textbf{分割}(partition)。
\end{definition}

\subsection{全機率定理}

\begin{theorem}[全機率定理 (Law of Total Probability)]
設 $\{B_1, B_2, \ldots, B_n\}$ 為樣本空間的一個分割,$A$ 為任意事件。則:
\[
\prb(A) = \sum_{i=1}^{n} \prb(A|B_i) \cdot \prb(B_i)
\]
\end{theorem}

\begin{prf}
由於 $\{B_1, \ldots, B_n\}$ 為分割,任何事件 $A$ 可分解為:
\[
A = A \cap \Omega = A \cap (B_1 \cup B_2 \cup \cdots \cup B_n) = (A \cap B_1) \cup (A \cap B_2) \cup \cdots \cup (A \cap B_n)
\]

因 $B_i$ 兩兩互斥,$(A \cap B_i)$ 也兩兩互斥。由機率公理 (iii):
\[
\prb(A) = \sum_{i=1}^{n} \prb(A \cap B_i) = \sum_{i=1}^{n} \prb(A|B_i) \cdot \prb(B_i)
\]
\end{prf}

\begin{example}
某工廠有三條生產線,產量分別佔總產量的 20\%、30\%、50\%。各生產線的不良率分別為 5\%、3\%、1\%。隨機抽取一件產品,求其為不良品的機率。
\end{example}

\begin{solution}
設 $B_1$、$B_2$、$B_3$ 分別表示產品來自生產線 1、2、3,$A$ 表示產品為不良品。

已知:
\begin{itemize}
    \item $\prb(B_1) = 0.20$,$\prb(B_2) = 0.30$,$\prb(B_3) = 0.50$
    \item $\prb(A|B_1) = 0.05$,$\prb(A|B_2) = 0.03$,$\prb(A|B_3) = 0.01$
\end{itemize}

由全機率定理:
\begin{align*}
\prb(A) &= \prb(A|B_1)\prb(B_1) + \prb(A|B_2)\prb(B_2) + \prb(A|B_3)\prb(B_3)\\
&= 0.05 \times 0.20 + 0.03 \times 0.30 + 0.01 \times 0.50\\
&= 0.010 + 0.009 + 0.005\\
&= 0.024
\end{align*}

故隨機抽取的產品為不良品的機率為 2.4\%。
\end{solution}

\subsection{貝氏定理}

\begin{theorem}[貝氏定理 (Bayes' Theorem)]
設 $\{B_1, B_2, \ldots, B_n\}$ 為樣本空間的一個分割,$A$ 為任意事件且 $\prb(A) > 0$。則對任意 $j = 1, 2, \ldots, n$:
\[
\prb(B_j|A) = \frac{\prb(A|B_j) \cdot \prb(B_j)}{\sum_{i=1}^{n} \prb(A|B_i) \cdot \prb(B_i)} = \frac{\prb(A|B_j) \cdot \prb(B_j)}{\prb(A)}
\]
\end{theorem}

\begin{prf}
由條件機率定義:
\[
\prb(B_j|A) = \frac{\prb(A \cap B_j)}{\prb(A)} = \frac{\prb(A|B_j) \cdot \prb(B_j)}{\prb(A)}
\]
再由全機率定理將 $\prb(A)$ 展開即得。
\end{prf}

\begin{note}[貝氏定理術語]
\begin{itemize}
  \item[]
    \item $\prb(B_j)$:\textbf{先驗機率}(prior probability)——在觀察到 $A$ 之前,對 $B_j$ 的機率判斷
    \item $\prb(B_j|A)$:\textbf{後驗機率}(posterior probability)——在觀察到 $A$ 之後,對 $B_j$ 的更新機率
    \item $\prb(A|B_j)$:\textbf{似然}(likelihood)——在 $B_j$ 條件下觀察到 $A$ 的機率
\end{itemize}
貝氏定理描述了「如何根據新證據更新我們的信念」。
\end{note}

\begin{example}[續上例]
若隨機抽取的產品為不良品,求其來自生產線 1 的機率。
\end{example}

\begin{solution}
由貝氏定理:
\[
\prb(B_1|A) = \frac{\prb(A|B_1) \cdot \prb(B_1)}{\prb(A)} = \frac{0.05 \times 0.20}{0.024} = \frac{0.010}{0.024} = \frac{10}{24} = \frac{5}{12} \approx 0.417
\]

雖然生產線 1 只佔總產量的 20\%,但因其不良率較高(5\%),在已知產品為不良品的條件下,它來自生產線 1 的機率提高到約 41.7\%。
\end{solution}

\begin{example}\label{ex:multiple-choice}
參加多選題考試,每題有 $c$ 個選項。你知道答案的機率為 $p$;若不知道答案則隨機猜測。已知你答對了某題,求你真正知道答案的機率。
\end{example}

\begin{solution}
設事件:
\begin{itemize}
    \item $K$ = 知道答案
    \item $C$ = 答對
\end{itemize}

已知條件:
\begin{itemize}
    \item $\prb(K) = p$,$\prb(K^c) = 1 - p$
    \item $\prb(C|K) = 1$(若知道答案,必答對)
    \item $\prb(C|K^c) = \dfrac{1}{c}$(不知道答案,隨機猜中的機率)
\end{itemize}

求 $\prb(K|C)$。

\textbf{步驟 1}:由全機率定理求 $\prb(C)$:
\begin{align*}
\prb(C) &= \prb(C|K) \cdot \prb(K) + \prb(C|K^c) \cdot \prb(K^c)\\
&= 1 \cdot p + \frac{1}{c} \cdot (1-p)\\
&= p + \frac{1-p}{c} = \frac{cp + 1 - p}{c} = \frac{1 + p(c-1)}{c}
\end{align*}

\textbf{步驟 2}:由貝氏定理求 $\prb(K|C)$:
\[
\prb(K|C) = \frac{\prb(C|K) \cdot \prb(K)}{\prb(C)} = \frac{1 \cdot p}{\dfrac{1 + p(c-1)}{c}} = \frac{cp}{1 + p(c-1)}
\]

\textbf{驗證(特殊情形)}:
\begin{itemize}
    \item 若 $p = 1$(總是知道答案):$\prb(K|C) = \dfrac{c}{c} = 1$ \checkmark
    \item 若 $p = 0$(完全不知道):$\prb(K|C) = \dfrac{0}{1} = 0$ \checkmark
    \item 若 $c \to \infty$(選項很多):$\prb(K|C) \to 1$(答對很可能是真知道)\checkmark
\end{itemize}
\end{solution}

%=============================================================================
\section{聯合機率與邊際機率}
%=============================================================================

\subsection{聯合機率}

\begin{definition}[聯合機率]
對於兩個事件 $A$ 和 $B$,它們同時發生的機率 $\prb(A \cap B)$ 稱為 $A$ 與 $B$ 的\textbf{聯合機率}(joint probability)。
\end{definition}

當處理兩個或多個隨機變數時,我們常用\textbf{聯合機率表}來呈現資料。

\begin{example}
超市根據 86,214 次購物記錄,統計顧客養貓數量與購買貓糧數量的聯合機率(部分數據):

\begin{center}
\begin{tabular}{l|ccccc}
\toprule
 & 無貓 & 1 隻貓 & 2 隻貓 & 3 隻貓 & 3 隻以上 \\
\midrule
無購買貓糧 & 0.0487 & 0.0217 & 0.0025 & 0.0002 & 0 \\
1--3 件 & 0.1698 & 0.0734 & 0.0104 & 0.0004 & 0.0002 \\
4--6 件 & 0.1182 & 0.0516 & 0.0093 & 0.0006 & 0.0002 \\
7--12 件 & 0.1160 & 0.0469 & 0.0113 & 0.0012 & 0.0005 \\
12 件以上 & 0.2103 & 0.0818 & 0.0216 & 0.0021 & 0.0011 \\
\bottomrule
\end{tabular}
\end{center}

表格中每個數值代表「養特定數量貓」\textbf{且}「購買特定數量貓糧」的機率。所有格子的機率加總應等於 1。
\end{example}

\subsection{邊際機率}

\begin{definition}[邊際機率]
對於聯合機率表,將某一行或某一列的機率加總,所得的機率稱為\textbf{邊際機率}(marginal probability)。
\end{definition}

\begin{example}[續上例]
求「無養貓」的邊際機率。
\end{example}

\begin{solution}
將「無貓」那一行的所有機率加總:
\begin{align*}
\prb(\text{無貓}) &= 0.0487 + 0.1698 + 0.1182 + 0.1160 + 0.2103\\
&= 0.6630
\end{align*}
\end{solution}

\subsection{聯合機率表與獨立性}

\begin{proposition}
兩個類別變數 $X$ 與 $Y$ 獨立,若且唯若對所有可能的值 $x$ 與 $y$:
\[
\prb(X = x, Y = y) = \prb(X = x) \cdot \prb(Y = y)
\]
即聯合機率等於兩個邊際機率的乘積。
\end{proposition}

\begin{example}
檢驗上例中「養貓數量」與「購買貓糧數量」是否獨立。
\end{example}

\begin{solution}
從上例計算:
\begin{itemize}
    \item $\prb(\text{無貓}) = 0.6630$
    \item $\prb(\text{3隻以上貓}) = 0 + 0.0002 + 0.0002 + 0.0005 + 0.0011 = 0.0020$
    \item $\prb(\text{無購買貓糧}) = 0.0487 + 0.0217 + 0.0025 + 0.0002 + 0 = 0.0731$
\end{itemize}

若獨立,應有:
\[
\prb(\text{無貓}, \text{無購買貓糧}) = \prb(\text{無貓}) \times \prb(\text{無購買貓糧}) = 0.6630 \times 0.0731 = 0.0485
\]

但實際觀察值為 $0.0487 \approx 0.0485$,非常接近。

再檢驗另一組:
\[
\prb(\text{3隻以上貓}, \text{12件以上}) = 0.0011
\]
若獨立:$\prb(\text{3隻以上貓}) \times \prb(\text{12件以上}) = ?$

需先算 $\prb(\text{12件以上}) = 0.2103 + 0.0818 + 0.0216 + 0.0021 + 0.0011 = 0.3169$

若獨立:$0.0020 \times 0.3169 = 0.00063$

但實際值為 $0.0011$,差距較大。

結論:養貓數量與購買貓糧數量\textbf{不獨立}(直覺上也合理:養越多貓,越可能購買更多貓糧)。
\end{solution}

\subsection{由聯合機率計算條件機率}

\begin{example}
利用上述購物資料,求「在養 3 隻以上貓的條件下,購買超過 3 件貓糧」的條件機率。
\end{example}

\begin{solution}
設 $A$ =「購買超過 3 件貓糧」(即 4--6 件、7--12 件、或 12 件以上)

設 $B$ =「養 3 隻以上貓」

先計算邊際機率:
\[
\prb(B) = \prb(\text{3隻以上貓}) = 0 + 0.0002 + 0.0002 + 0.0005 + 0.0011 = 0.0020
\]

計算聯合機率:
\[
\prb(A \cap B) = 0.0002 + 0.0005 + 0.0011 = 0.0018
\]
(養 3 隻以上貓且購買 4 件以上貓糧)

條件機率:
\[
\prb(A|B) = \frac{\prb(A \cap B)}{\prb(B)} = \frac{0.0018}{0.0020} = 0.90 = 90\%
\]

對比「無養貓」的顧客:
\[
\prb(A|\text{無貓}) = \frac{0.1182 + 0.1160 + 0.2103}{0.6630} = \frac{0.4445}{0.6630} \approx 0.67 = 67\%
\]

\textbf{結論}:養 3 隻以上貓的顧客購買超過 3 件貓糧的條件機率(90\%)遠高於無養貓的顧客(67\%)。這再次說明兩變數不獨立,且呈正相關。
\end{solution}

%=============================================================================
\section{本章習題}
%=============================================================================

\begin{exercise}
一個袋子裡有 4 顆紅球和 6 顆白球。隨機抽取一球後放回,再抽第二球。
\begin{enumerate}[label=(\alph*)]
    \item 求兩球都是紅球的機率。
    \item 求至少一球是紅球的機率。
    \item 設 $A$ =「第一球為紅」,$B$ =「第二球為紅」。判斷 $A$ 與 $B$ 是否獨立。
\end{enumerate}
\end{exercise}

\begin{solution}
\begin{enumerate}[label=(\alph*)]
  \item[]
    \item 由於放回抽樣,每次抽到紅球的機率都是 $4/10 = 2/5$。
    \[
    \prb(\text{兩球皆紅}) = \frac{2}{5} \times \frac{2}{5} = \frac{4}{25}
    \]
    
    \item 令 $C$ =「至少一球為紅」。則 $C^c$ =「兩球皆白」。
    \[
    \prb(C^c) = \frac{6}{10} \times \frac{6}{10} = \frac{36}{100} = \frac{9}{25}
    \]
    \[
    \prb(C) = 1 - \prb(C^c) = 1 - \frac{9}{25} = \frac{16}{25}
    \]
    
    \item 由於是放回抽樣:
    \begin{itemize}
        \item $\prb(A) = 4/10 = 2/5$
        \item $\prb(B) = 4/10 = 2/5$
        \item $\prb(A \cap B) = (2/5)(2/5) = 4/25$
        \item $\prb(A) \cdot \prb(B) = (2/5)(2/5) = 4/25 = \prb(A \cap B)$
    \end{itemize}
    故 $A$ 與 $B$ \textbf{獨立}。
\end{enumerate}
\end{solution}

\begin{exercise}
某疾病在人群中的盛行率為 1\%。有一種檢測方法,對於患病者有 99\% 的機率呈陽性(敏感度),對於健康者有 95\% 的機率呈陰性(特異度)。若某人檢測呈陽性,求其真正患病的機率。
\end{exercise}

\begin{solution}
設 $D$ =「患病」,$T^+$ =「檢測陽性」。

已知:
\begin{itemize}
    \item $\prb(D) = 0.01$,$\prb(D^c) = 0.99$
    \item $\prb(T^+|D) = 0.99$(敏感度)
    \item $\prb(T^-|D^c) = 0.95$,故 $\prb(T^+|D^c) = 0.05$(偽陽性率)
\end{itemize}

求 $\prb(D|T^+)$。

由全機率定理:
\begin{align*}
\prb(T^+) &= \prb(T^+|D) \cdot \prb(D) + \prb(T^+|D^c) \cdot \prb(D^c)\\
&= 0.99 \times 0.01 + 0.05 \times 0.99\\
&= 0.0099 + 0.0495 = 0.0594
\end{align*}

由貝氏定理:
\[
\prb(D|T^+) = \frac{\prb(T^+|D) \cdot \prb(D)}{\prb(T^+)} = \frac{0.99 \times 0.01}{0.0594} = \frac{0.0099}{0.0594} \approx 0.167
\]

\textbf{結論}:即使檢測呈陽性,真正患病的機率只有約 16.7\%。這個看似反直覺的結果是因為疾病盛行率很低(1\%),而偽陽性(5\%)相對較高。
\end{solution}

\begin{exercise}
三人 A、B、C 輪流擲硬幣,按 A$\to$B$\to$C$\to$A$\to \cdots$ 的順序進行。最先擲出正面者獲勝。假設硬幣為公正硬幣。求三人各自獲勝的機率。
\end{exercise}

\begin{solution}
設 $P_A$、$P_B$、$P_C$ 分別為 A、B、C 獲勝的機率。

A 獲勝的情形:A 第一輪擲出正面,或三人第一輪都擲反面後 A 再獲勝。
\[
P_A = \frac{1}{2} + \frac{1}{2} \cdot \frac{1}{2} \cdot \frac{1}{2} \cdot P_A = \frac{1}{2} + \frac{1}{8} P_A
\]

解得:
\[
P_A - \frac{1}{8} P_A = \frac{1}{2} \implies \frac{7}{8} P_A = \frac{1}{2} \implies P_A = \frac{4}{7}
\]

B 獲勝的情形:A 擲反面,B 擲正面,或 A、B、C 都擲反面後 B 再獲勝。
\[
P_B = \frac{1}{2} \cdot \frac{1}{2} + \frac{1}{8} P_B = \frac{1}{4} + \frac{1}{8} P_B
\]

解得:
\[
\frac{7}{8} P_B = \frac{1}{4} \implies P_B = \frac{2}{7}
\]

由 $P_A + P_B + P_C = 1$:
\[
P_C = 1 - \frac{4}{7} - \frac{2}{7} = \frac{1}{7}
\]

\textbf{驗證}:先擲的人有優勢,$P_A > P_B > P_C$,符合直覺。
\end{solution}

\begin{exercise}
設 $\prb(A) = 0.6$,$\prb(B) = 0.4$,$\prb(A|B) = 0.5$。求:
\begin{enumerate}[label=(\alph*)]
    \item $\prb(A \cap B)$
    \item $\prb(A \cup B)$
    \item $\prb(B|A)$
    \item 判斷 $A$ 與 $B$ 是否獨立
\end{enumerate}
\end{exercise}

\begin{solution}
\begin{enumerate}[label=(\alph*)]
    \item 由條件機率定義:
    \[
    \prb(A \cap B) = \prb(A|B) \cdot \prb(B) = 0.5 \times 0.4 = 0.2
    \]
    
    \item 由加法法則:
    \[
    \prb(A \cup B) = \prb(A) + \prb(B) - \prb(A \cap B) = 0.6 + 0.4 - 0.2 = 0.8
    \]
    
    \item 由條件機率定義:
    \[
    \prb(B|A) = \frac{\prb(A \cap B)}{\prb(A)} = \frac{0.2}{0.6} = \frac{1}{3}
    \]
    
    \item 檢驗獨立性:
    \[
    \prb(A) \cdot \prb(B) = 0.6 \times 0.4 = 0.24 \neq 0.2 = \prb(A \cap B)
    \]
    故 $A$ 與 $B$ \textbf{不獨立}。
    
    另一種檢驗:$\prb(A|B) = 0.5 \neq 0.6 = \prb(A)$,同樣說明不獨立。
\end{enumerate}
\end{solution}

%=============================================================================
%% \section*{本章重點整理}
%=============================================================================

% \begin{enumerate}
%     \item \textbf{機率公理}:非負性、規範性、可加性是機率的三條基本公理。
    
%     \item \textbf{加法法則}:$\prb(A \cup B) = \prb(A) + \prb(B) - \prb(A \cap B)$
    
%     \item \textbf{條件機率}:$\prb(A|B) = \dfrac{\prb(A \cap B)}{\prb(B)}$
    
%     \item \textbf{乘法法則}:$\prb(A \cap B) = \prb(A|B) \cdot \prb(B) = \prb(B|A) \cdot \prb(A)$
    
%     \item \textbf{獨立性}:$A$ 與 $B$ 獨立 $\Leftrightarrow$ $\prb(A \cap B) = \prb(A) \cdot \prb(B)$ $\Leftrightarrow$ $\prb(A|B) = \prb(A)$
    
%     \item \textbf{全機率定理}:$\prb(A) = \sum_i \prb(A|B_i) \cdot \prb(B_i)$
    
%     \item \textbf{貝氏定理}:$\prb(B_j|A) = \dfrac{\prb(A|B_j) \cdot \prb(B_j)}{\prb(A)}$
    
%     \item \textbf{聯合機率}:$\prb(X = x, Y = y)$ 表示兩事件同時發生的機率
    
%     \item \textbf{邊際機率}:由聯合機率表的行或列加總得到
% \end{enumerate}



\end{document}
