\documentclass[12pt,a4paper]{article}
\usepackage[left=1cm,right=1cm,bottom=15mm,top=20mm]{geometry}
\usepackage[AutoFakeBold,AutoFakeSlant]{xeCJK}
\setCJKmainfont[AutoFakeSlant=.1,AutoFakeBold=2]{Noto Serif CJK TC}
\usepackage{amsmath,amsthm,amssymb,amsfonts}
\usepackage{graphicx,xcolor,float}
\usepackage{booktabs,tabularx,multirow,array}
\usepackage{parskip}
\usepackage{enumitem}
\setlist{itemsep=0pt,parsep=0pt}
\usepackage{hyperref}
\hypersetup{
    colorlinks=true,
    linkcolor=blue!70!black,
    urlcolor=blue!80!black
}

% 機率與統計符號
\newcommand\expc{\mathsf{E}}
\newcommand\prb{\mathsf{P}}
\DeclareMathOperator\var{var}
\DeclareMathOperator\cov{cov}
\DeclareMathOperator\corr{corr}

% 中文化
\renewcommand{\figurename}{圖}
\renewcommand{\tablename}{表}
\renewcommand{\proofname}{\textbf{證明}}

% 定理環境
\theoremstyle{definition}
\newtheorem{definition}{定義}[section]
\newtheorem{example}{例題}[section]
\newtheorem{exercise}{習題}[section]
\newtheorem{theorem}{定理}[section]
\newtheorem{lemma}{引理}[section]
\newtheorem{corollary}{推論}[section]
\newtheorem{proposition}{命題}[section]
\newtheorem{property}{性質}[section]
\newtheorem*{remark}{註}
\newtheorem*{solution}{解答}
\newtheorem*{note}{說明}
\newtheorem*{prf}{證明}

% 常用指令
\newcommand{\ds}{\displaystyle}
\newcommand{\ie}{\;\Longrightarrow\;}
\newcommand{\SST}{\mathrm{SST}}
\newcommand{\SSR}{\mathrm{SSR}}
\newcommand{\SSE}{\mathrm{SSE}}
\newcommand{\MSR}{\mathrm{MSR}}
\newcommand{\MSE}{\mathrm{MSE}}
\newcommand{\SE}{\mathrm{SE}}

% 頁面設定
\usepackage{fancyhdr}
\pagestyle{fancy}
\fancyhf{}
\fancyhead[L]{統計學講義}
\fancyhead[R]{第六部分:迴歸分析}
\fancyfoot[C]{\thepage}
\renewcommand{\headrulewidth}{0.4pt}
\renewcommand{\footrulewidth}{0.4pt}

\title{\vspace{-2cm}\textbf{統計學講義}\\[3mm] \Large 第六部分:迴歸分析}
\author{}
\date{\vspace{-2cm}}

\begin{document}
\maketitle
\thispagestyle{fancy}

%=============================================================================
\section{簡單線性迴歸}
%=============================================================================

\subsection{模型與估計}

\begin{definition}[簡單線性迴歸模型]
\[
Y_i = \beta_0 + \beta_1 X_i + \varepsilon_i, \quad i = 1, \ldots, n
\]
其中 $\beta_0$ 為截距,$\beta_1$ 為斜率,$\varepsilon_i \sim N(0, \sigma^2)$ i.i.d.
\end{definition}

\begin{theorem}[OLS 估計量]\label{thm:ols}
最小平方法(OLS)估計量為:
\[
\boxed{\hat{\beta}_1 = \frac{\sum_{i=1}^{n}(X_i - \bar{X})(Y_i - \bar{Y})}{\sum_{i=1}^{n}(X_i - \bar{X})^2} = \frac{S_{XY}}{S_{XX}}}
\]
\[
\boxed{\hat{\beta}_0 = \bar{Y} - \hat{\beta}_1 \bar{X}}
\]
其中:$S_{XX} = \sum(X_i - \bar{X})^2$,$S_{XY} = \sum(X_i - \bar{X})(Y_i - \bar{Y})$。
\end{theorem}

\begin{prf}
OLS 最小化殘差平方和 $Q(\beta_0, \beta_1) = \sum_{i=1}^{n}(Y_i - \beta_0 - \beta_1 X_i)^2$。

\textbf{Step 1}:對 $\beta_0$ 偏微分並令其為零:
\[
\frac{\partial Q}{\partial \beta_0} = -2\sum_{i=1}^{n}(Y_i - \beta_0 - \beta_1 X_i) = 0
\]
展開得 $\sum Y_i - n\beta_0 - \beta_1\sum X_i = 0$,故
\[
\hat{\beta}_0 = \bar{Y} - \hat{\beta}_1\bar{X}
\]

\textbf{Step 2}:對 $\beta_1$ 偏微分並令其為零:
\[
\frac{\partial Q}{\partial \beta_1} = -2\sum_{i=1}^{n}X_i(Y_i - \beta_0 - \beta_1 X_i) = 0
\]
展開得 $\sum X_i Y_i - \beta_0\sum X_i - \beta_1\sum X_i^2 = 0$。

\textbf{Step 3}:將 $\hat{\beta}_0 = \bar{Y} - \hat{\beta}_1\bar{X}$ 代入:
\begin{align*}
\sum X_i Y_i - (\bar{Y} - \hat{\beta}_1\bar{X})\sum X_i - \hat{\beta}_1\sum X_i^2 &= 0\\
\sum X_i Y_i - n\bar{X}\bar{Y} + \hat{\beta}_1 n\bar{X}^2 - \hat{\beta}_1\sum X_i^2 &= 0\\
\sum X_i Y_i - n\bar{X}\bar{Y} &= \hat{\beta}_1(\sum X_i^2 - n\bar{X}^2)
\end{align*}

由於 $S_{XY} = \sum X_i Y_i - n\bar{X}\bar{Y}$ 且 $S_{XX} = \sum X_i^2 - n\bar{X}^2$,故
\[
\hat{\beta}_1 = \frac{S_{XY}}{S_{XX}} = \frac{\sum(X_i - \bar{X})(Y_i - \bar{Y})}{\sum(X_i - \bar{X})^2}
\]
\end{prf}

\begin{theorem}[OLS 估計量的不偏性]\label{thm:unbiased}
$\hat{\beta}_0$ 與 $\hat{\beta}_1$ 是 $\beta_0$ 與 $\beta_1$ 的不偏估計量。
\end{theorem}

\begin{prf}
\textbf{證明 $\expc[\hat{\beta}_1] = \beta_1$}:

將 $\hat{\beta}_1$ 改寫為 $Y_i$ 的線性組合。令 $w_i = \dfrac{X_i - \bar{X}}{S_{XX}}$,則 $\sum w_i = 0$ 且 $\sum w_i X_i = 1$。
\[
\hat{\beta}_1 = \frac{\sum(X_i - \bar{X})Y_i}{S_{XX}} = \sum w_i Y_i
\]

由模型 $Y_i = \beta_0 + \beta_1 X_i + \varepsilon_i$:
\begin{align*}
\expc[\hat{\beta}_1] &= \expc\left[\sum w_i Y_i\right] = \sum w_i \expc[Y_i] = \sum w_i(\beta_0 + \beta_1 X_i)\\
&= \beta_0 \sum w_i + \beta_1 \sum w_i X_i = \beta_0 \cdot 0 + \beta_1 \cdot 1 = \beta_1
\end{align*}

\textbf{證明 $\expc[\hat{\beta}_0] = \beta_0$}:
\[
\expc[\hat{\beta}_0] = \expc[\bar{Y} - \hat{\beta}_1\bar{X}] = \expc[\bar{Y}] - \bar{X}\expc[\hat{\beta}_1] = (\beta_0 + \beta_1\bar{X}) - \bar{X}\beta_1 = \beta_0
\]
\end{prf}

\begin{theorem}[OLS 估計量的變異數]\label{thm:var-beta}
\[
\boxed{\var(\hat{\beta}_1) = \frac{\sigma^2}{S_{XX}}}
\]
\[
\boxed{\var(\hat{\beta}_0) = \sigma^2\left(\frac{1}{n} + \frac{\bar{X}^2}{S_{XX}}\right)}
\]
\end{theorem}

\begin{prf}
\textbf{證明 $\var(\hat{\beta}_1) = \sigma^2/S_{XX}$}:

由 $\hat{\beta}_1 = \sum w_i Y_i$,其中 $w_i = (X_i - \bar{X})/S_{XX}$,且 $Y_i$ 獨立:
\begin{align*}
\var(\hat{\beta}_1) &= \var\left(\sum w_i Y_i\right) = \sum w_i^2 \var(Y_i) = \sigma^2 \sum w_i^2\\
&= \sigma^2 \sum \frac{(X_i - \bar{X})^2}{S_{XX}^2} = \sigma^2 \cdot \frac{S_{XX}}{S_{XX}^2} = \frac{\sigma^2}{S_{XX}}
\end{align*}

\textbf{證明 $\var(\hat{\beta}_0)$}:
\begin{align*}
\var(\hat{\beta}_0) &= \var(\bar{Y} - \hat{\beta}_1\bar{X}) = \var(\bar{Y}) + \bar{X}^2\var(\hat{\beta}_1) - 2\bar{X}\cov(\bar{Y}, \hat{\beta}_1)
\end{align*}

由於 $\cov(\bar{Y}, \hat{\beta}_1) = \cov\left(\frac{1}{n}\sum Y_i, \sum w_i Y_i\right) = \frac{\sigma^2}{n}\sum w_i = 0$,故
\[
\var(\hat{\beta}_0) = \frac{\sigma^2}{n} + \bar{X}^2 \cdot \frac{\sigma^2}{S_{XX}} = \sigma^2\left(\frac{1}{n} + \frac{\bar{X}^2}{S_{XX}}\right)
\]
\end{prf}

\subsection{變異數分解}

\begin{theorem}[變異數分解——ANOVA]\label{thm:anova}
\[
\boxed{\SST = \SSR + \SSE}
\]
其中:
\begin{itemize}
\item $\SST = \sum(Y_i - \bar{Y})^2$(總變異)
\item $\SSR = \sum(\hat{Y}_i - \bar{Y})^2$(迴歸變異)
\item $\SSE = \sum(Y_i - \hat{Y}_i)^2$(殘差變異)
\end{itemize}
\end{theorem}

\begin{prf}
將 $Y_i - \bar{Y}$ 分解為 $(Y_i - \hat{Y}_i) + (\hat{Y}_i - \bar{Y})$:
\begin{align*}
\SST &= \sum(Y_i - \bar{Y})^2 = \sum\big[(Y_i - \hat{Y}_i) + (\hat{Y}_i - \bar{Y})\big]^2\\
&= \sum(Y_i - \hat{Y}_i)^2 + \sum(\hat{Y}_i - \bar{Y})^2 + 2\sum(Y_i - \hat{Y}_i)(\hat{Y}_i - \bar{Y})
\end{align*}

\textbf{關鍵}:證明交叉項為零。令 $e_i = Y_i - \hat{Y}_i$(殘差),則:
\begin{align*}
\sum e_i(\hat{Y}_i - \bar{Y}) &= \sum e_i \hat{Y}_i - \bar{Y}\sum e_i
\end{align*}

由 OLS 正規方程式,$\sum e_i = 0$ 且 $\sum e_i X_i = 0$。

又 $\hat{Y}_i = \hat{\beta}_0 + \hat{\beta}_1 X_i$,故
\[
\sum e_i \hat{Y}_i = \hat{\beta}_0 \sum e_i + \hat{\beta}_1 \sum e_i X_i = 0
\]

因此交叉項為零,$\SST = \SSE + \SSR$。
\end{prf}

\begin{corollary}[SSR 的另一形式]
$\SSR = \hat{\beta}_1^2 S_{XX} = \hat{\beta}_1 S_{XY}$。
\end{corollary}

\begin{prf}
\begin{align*}
\SSR &= \sum(\hat{Y}_i - \bar{Y})^2 = \sum(\hat{\beta}_0 + \hat{\beta}_1 X_i - \bar{Y})^2\\
&= \sum\big[(\bar{Y} - \hat{\beta}_1\bar{X}) + \hat{\beta}_1 X_i - \bar{Y}\big]^2 = \sum\big[\hat{\beta}_1(X_i - \bar{X})\big]^2\\
&= \hat{\beta}_1^2 \sum(X_i - \bar{X})^2 = \hat{\beta}_1^2 S_{XX}
\end{align*}

又 $\hat{\beta}_1 = S_{XY}/S_{XX}$,故 $\SSR = \hat{\beta}_1^2 S_{XX} = \hat{\beta}_1 \cdot S_{XY}$。
\end{prf}

\begin{theorem}[$R^2$ 與相關係數的關係]\label{thm:r2-corr}
$R^2 = r_{XY}^2$,其中 $r_{XY}$ 是 $X$ 與 $Y$ 的樣本相關係數。
\end{theorem}

\begin{prf}
樣本相關係數 $\ds r_{XY} = \frac{S_{XY}}{\sqrt{S_{XX} \cdot S_{YY}}}$。

由 $R^2 = \SSR/\SST = \hat{\beta}_1^2 S_{XX}/S_{YY}$ 且 $\hat{\beta}_1 = S_{XY}/S_{XX}$:
\[
R^2 = \frac{(S_{XY}/S_{XX})^2 \cdot S_{XX}}{S_{YY}} = \frac{S_{XY}^2}{S_{XX} \cdot S_{YY}} = r_{XY}^2
\]
\end{prf}

\subsection{$\sigma^2$ 的估計}

\begin{theorem}[$\sigma^2$ 的不偏估計]\label{thm:mse}
\[
s^2 = \MSE = \frac{\SSE}{n-2}
\]
是 $\sigma^2$ 的不偏估計量。
\end{theorem}

\begin{prf}
需證明 $\expc[\SSE] = (n-2)\sigma^2$。

由 $\SSE = \SST - \SSR = \sum(Y_i - \bar{Y})^2 - \hat{\beta}_1^2 S_{XX}$:
\begin{align*}
\expc[\SSE] &= \expc[\SST] - S_{XX}\,\expc[\hat{\beta}_1^2]
\end{align*}

\textbf{計算 $\expc[\SST]$}:
\begin{align*}
\expc[\SST] &= \expc\left[\sum Y_i^2 - n\bar{Y}^2\right] = \sum\expc[Y_i^2] - n\,\expc[\bar{Y}^2]
\end{align*}

由 $\expc[Y_i^2] = \var(Y_i) + (\expc[Y_i])^2 = \sigma^2 + (\beta_0 + \beta_1 X_i)^2$,且
$\expc[\bar{Y}^2] = \var(\bar{Y}) + (\expc[\bar{Y}])^2 = \sigma^2/n + (\beta_0 + \beta_1\bar{X})^2$,經計算:
\[
\expc[\SST] = (n-1)\sigma^2 + \beta_1^2 S_{XX}
\]

\textbf{計算 $\expc[\hat{\beta}_1^2]$}:
\[
\expc[\hat{\beta}_1^2] = \var(\hat{\beta}_1) + (\expc[\hat{\beta}_1])^2 = \frac{\sigma^2}{S_{XX}} + \beta_1^2
\]

\textbf{合併}:
\begin{align*}
\expc[\SSE] &= (n-1)\sigma^2 + \beta_1^2 S_{XX} - S_{XX}\left(\frac{\sigma^2}{S_{XX}} + \beta_1^2\right)\\
&= (n-1)\sigma^2 + \beta_1^2 S_{XX} - \sigma^2 - \beta_1^2 S_{XX} = (n-2)\sigma^2
\end{align*}

故 $\expc[\MSE] = \expc[\SSE/(n-2)] = \sigma^2$。
\end{prf}

\subsection{假設檢定}

\begin{theorem}[斜率的 $t$ 檢定]\label{thm:t-test}
在 $H_0: \beta_1 = 0$ 下,檢定統計量
\[
\boxed{t = \frac{\hat{\beta}_1}{\SE(\hat{\beta}_1)} = \frac{\hat{\beta}_1}{s/\sqrt{S_{XX}}} \sim t_{n-2}}
\]
\end{theorem}

\begin{prf}
\textbf{Step 1}:標準化 $\hat{\beta}_1$。

由定理 \ref{thm:unbiased} 和 \ref{thm:var-beta},$\hat{\beta}_1 \sim N(\beta_1, \sigma^2/S_{XX})$,故在 $H_0: \beta_1 = 0$ 下:
\[
Z = \frac{\hat{\beta}_1 - 0}{\sigma/\sqrt{S_{XX}}} \sim N(0, 1)
\]

\textbf{Step 2}:$\SSE/\sigma^2$ 的分配。

可證明 $\SSE/\sigma^2 \sim \chi^2_{n-2}$(殘差有 $n$ 個觀測值減去 2 個估計參數)。

\textbf{Step 3}:$\hat{\beta}_1$ 與 $\SSE$ 獨立。

由常態分配的性質,$\hat{\beta}_1$(依賴 $\bar{Y}$ 的方向)與 $\SSE$(垂直於該方向)獨立。

\textbf{Step 4}:構造 $t$ 統計量。

由 $t$ 分配定義:
\[
t = \frac{Z}{\sqrt{\chi^2_{n-2}/(n-2)}} = \frac{\hat{\beta}_1/(\sigma/\sqrt{S_{XX}})}{\sqrt{\SSE/\sigma^2/(n-2)}} = \frac{\hat{\beta}_1}{s/\sqrt{S_{XX}}} \sim t_{n-2}
\]
\end{prf}

\begin{theorem}[整體 $F$ 檢定]\label{thm:f-test}
在 $H_0: \beta_1 = 0$ 下,
\[
\boxed{F = \frac{\MSR}{\MSE} = \frac{\SSR/1}{\SSE/(n-2)} \sim F_{1, n-2}}
\]
且 $F = t^2$。
\end{theorem}

\begin{prf}
\textbf{證明 $F \sim F_{1, n-2}$}:

在 $H_0: \beta_1 = 0$ 下,$\SSR/\sigma^2 \sim \chi^2_1$(因為 $\SSR$ 是一個參數的函數)。

由於 $\SSR$ 與 $\SSE$ 獨立(可由正交分解證明),且 $\SSE/\sigma^2 \sim \chi^2_{n-2}$:
\[
F = \frac{\SSR/\sigma^2/1}{\SSE/\sigma^2/(n-2)} = \frac{\SSR/1}{\SSE/(n-2)} \sim F_{1, n-2}
\]

\textbf{證明 $F = t^2$}:
\begin{align*}
F &= \frac{\SSR}{\MSE} = \frac{\hat{\beta}_1^2 S_{XX}}{s^2} = \frac{\hat{\beta}_1^2}{s^2/S_{XX}} = \left(\frac{\hat{\beta}_1}{s/\sqrt{S_{XX}}}\right)^2 = t^2
\end{align*}
\end{prf}

\subsection{調整後 $R^2$}

\begin{theorem}[調整後 $R^2$]\label{thm:adj-r2}
對於有 $k$ 個自變數的多元迴歸:
\[
\boxed{\bar{R}^2 = 1 - \frac{n-1}{n-k-1}(1 - R^2) = 1 - \frac{\SSE/(n-k-1)}{\SST/(n-1)}}
\]
\end{theorem}

\begin{prf}
由 $R^2 = 1 - \SSE/\SST$:
\[
1 - R^2 = \frac{\SSE}{\SST}
\]

調整後 $R^2$ 用均方取代平方和:
\[
\bar{R}^2 = 1 - \frac{\SSE/(n-k-1)}{\SST/(n-1)} = 1 - \frac{n-1}{n-k-1} \cdot \frac{\SSE}{\SST} = 1 - \frac{n-1}{n-k-1}(1 - R^2)
\]

這樣調整的原因:當增加無意義的自變數時,$\SSE$ 下降但自由度也下降,$\bar{R}^2$ 可能下降,避免過度擬合。
\end{prf}

%=============================================================================
\section{信賴區間與預測}
%=============================================================================

\begin{theorem}[迴歸係數的信賴區間]\label{thm:ci-beta}
$\beta_1$ 的 $(1-\alpha)$ 信賴區間為:
\[
\boxed{\hat{\beta}_1 \pm t_{\alpha/2, n-2} \cdot \frac{s}{\sqrt{S_{XX}}}}
\]
\end{theorem}

\begin{prf}
由定理 \ref{thm:t-test},$\dfrac{\hat{\beta}_1 - \beta_1}{s/\sqrt{S_{XX}}} \sim t_{n-2}$。

故 $\prb\left(-t_{\alpha/2, n-2} \leqslant \dfrac{\hat{\beta}_1 - \beta_1}{s/\sqrt{S_{XX}}} \leqslant t_{\alpha/2, n-2}\right) = 1 - \alpha$

移項得 $\prb\left(\hat{\beta}_1 - t_{\alpha/2, n-2} \cdot \dfrac{s}{\sqrt{S_{XX}}} \leqslant \beta_1 \leqslant \hat{\beta}_1 + t_{\alpha/2, n-2} \cdot \dfrac{s}{\sqrt{S_{XX}}}\right) = 1 - \alpha$
\end{prf}

\begin{theorem}[平均反應的信賴區間]\label{thm:ci-mean}
對於給定 $X = X_0$,$\expc[Y|X_0] = \beta_0 + \beta_1 X_0$ 的 $(1-\alpha)$ 信賴區間:
\[
\boxed{\hat{Y}_0 \pm t_{\alpha/2, n-2} \cdot s\sqrt{\frac{1}{n} + \frac{(X_0 - \bar{X})^2}{S_{XX}}}}
\]
\end{theorem}

\begin{prf}
令 $\hat{Y}_0 = \hat{\beta}_0 + \hat{\beta}_1 X_0$ 為 $\expc[Y|X_0]$ 的估計。

\textbf{計算 $\var(\hat{Y}_0)$}:
\begin{align*}
\hat{Y}_0 &= \bar{Y} - \hat{\beta}_1\bar{X} + \hat{\beta}_1 X_0 = \bar{Y} + \hat{\beta}_1(X_0 - \bar{X})
\end{align*}

由於 $\cov(\bar{Y}, \hat{\beta}_1) = 0$:
\begin{align*}
\var(\hat{Y}_0) &= \var(\bar{Y}) + (X_0 - \bar{X})^2 \var(\hat{\beta}_1)\\
&= \frac{\sigma^2}{n} + (X_0 - \bar{X})^2 \cdot \frac{\sigma^2}{S_{XX}} = \sigma^2\left(\frac{1}{n} + \frac{(X_0 - \bar{X})^2}{S_{XX}}\right)
\end{align*}

故 $\dfrac{\hat{Y}_0 - (\beta_0 + \beta_1 X_0)}{s\sqrt{\frac{1}{n} + \frac{(X_0 - \bar{X})^2}{S_{XX}}}} \sim t_{n-2}$,由此得信賴區間。
\end{prf}

\begin{theorem}[個別預測的預測區間]\label{thm:pi}
對於給定 $X = X_0$,個別觀測值 $Y_{\text{new}}$ 的 $(1-\alpha)$ 預測區間:
\[
\boxed{\hat{Y}_0 \pm t_{\alpha/2, n-2} \cdot s\sqrt{1 + \frac{1}{n} + \frac{(X_0 - \bar{X})^2}{S_{XX}}}}
\]
\end{theorem}

\begin{prf}
預測誤差為 $Y_{\text{new}} - \hat{Y}_0$。

由於 $Y_{\text{new}} = \beta_0 + \beta_1 X_0 + \varepsilon_{\text{new}}$ 與 $\hat{Y}_0$ 獨立:
\begin{align*}
\var(Y_{\text{new}} - \hat{Y}_0) &= \var(Y_{\text{new}}) + \var(\hat{Y}_0)\\
&= \sigma^2 + \sigma^2\left(\frac{1}{n} + \frac{(X_0 - \bar{X})^2}{S_{XX}}\right) = \sigma^2\left(1 + \frac{1}{n} + \frac{(X_0 - \bar{X})^2}{S_{XX}}\right)
\end{align*}

故預測區間比信賴區間多了 $\sigma^2$ 這一項(個別觀測的隨機變異)。
\end{prf}

%=============================================================================
\section{多元線性迴歸}
%=============================================================================

\begin{definition}[多元迴歸模型]
\[
Y_i = \beta_0 + \beta_1 X_{i1} + \cdots + \beta_k X_{ik} + \varepsilon_i, \quad \varepsilon_i \stackrel{\text{iid}}{\sim} N(0, \sigma^2)
\]
\end{definition}

\begin{theorem}[整體 $F$ 檢定(多元迴歸)]\label{thm:f-multiple}
檢定 $H_0: \beta_1 = \beta_2 = \cdots = \beta_k = 0$:
\[
\boxed{F = \frac{\MSR}{\MSE} = \frac{\SSR/k}{\SSE/(n-k-1)} = \frac{R^2/k}{(1-R^2)/(n-k-1)} \sim F_{k, n-k-1}}
\]
\end{theorem}

\begin{prf}
在 $H_0$ 下,$\SSR/\sigma^2 \sim \chi^2_k$($k$ 個限制式),$\SSE/\sigma^2 \sim \chi^2_{n-k-1}$($n$ 個觀測減去 $k+1$ 個參數),且兩者獨立。

由 $F$ 分配定義:
\[
F = \frac{\SSR/\sigma^2/k}{\SSE/\sigma^2/(n-k-1)} = \frac{\SSR/k}{\SSE/(n-k-1)} \sim F_{k, n-k-1}
\]

又 $R^2 = \SSR/\SST$,故 $\SSR = R^2 \cdot \SST$,$\SSE = (1-R^2)\SST$:
\[
F = \frac{R^2 \cdot \SST/k}{(1-R^2)\SST/(n-k-1)} = \frac{R^2/k}{(1-R^2)/(n-k-1)}
\]
\end{prf}

\begin{theorem}[個別係數 $t$ 檢定]\label{thm:t-individual}
對於 $H_0: \beta_j = 0$:
\[
\boxed{t = \frac{\hat{\beta}_j}{\SE(\hat{\beta}_j)} \sim t_{n-k-1}}
\]
\end{theorem}

\begin{prf}
由多元迴歸理論,$\hat{\beta}_j \sim N(\beta_j, \sigma^2 c_{jj})$,其中 $c_{jj}$ 是 $(\mathbf{X}'\mathbf{X})^{-1}$ 的第 $j$ 個對角元素。

以 $s^2$ 估計 $\sigma^2$ 後,$\dfrac{\hat{\beta}_j - \beta_j}{s\sqrt{c_{jj}}} \sim t_{n-k-1}$。

在 $H_0: \beta_j = 0$ 下,檢定統計量為 $t = \hat{\beta}_j/\SE(\hat{\beta}_j)$。
\end{prf}

%=============================================================================
\section{計算範例}
%=============================================================================

\begin{example}
給定資料:$(1, 2), (2, 4), (3, 5), (4, 4), (5, 5)$,進行完整線性迴歸分析。
\end{example}

\begin{solution}
\textbf{1. 基本統計量}:$n = 5$,$\bar{X} = 3$,$\bar{Y} = 4$,$S_{XX} = 10$,$S_{XY} = 6$,$S_{YY} = 6$。

\textbf{2. 估計值}:$\hat{\beta}_1 = 6/10 = 0.6$,$\hat{\beta}_0 = 4 - 0.6(3) = 2.2$。

\textbf{3. ANOVA}:$\SSR = (0.6)^2 \times 10 = 3.6$,$\SSE = 6 - 3.6 = 2.4$,$R^2 = 0.6$。

\textbf{4. 假設檢定}:$\MSE = 2.4/3 = 0.8$,$s = 0.894$。

$\SE(\hat{\beta}_1) = 0.894/\sqrt{10} = 0.283$,$t = 0.6/0.283 = 2.12$。

$df = 3$,$t_{0.025, 3} = 3.182$。因為 $|2.12| < 3.182$,不拒絕 $H_0$。

\textbf{5. 信賴區間}:$0.6 \pm 3.182 \times 0.283 = 0.6 \pm 0.90 = (-0.30, 1.50)$。
\end{solution}

\begin{example}
$n = 100$,$k = 4$,$R^2 = 0.36$,求 $F$ 值和調整後 $R^2$。
\end{example}

\begin{solution}
\textbf{$F$ 值}:
\[
F = \frac{0.36/4}{(1-0.36)/(100-4-1)} = \frac{0.09}{0.64/95} = \frac{0.09}{0.00674} = 13.35
\]

\textbf{調整後 $R^2$}:
\[
\bar{R}^2 = 1 - \frac{99}{95}(1 - 0.36) = 1 - 1.042 \times 0.64 = 1 - 0.667 = 0.333
\]
\end{solution}

%%=============================================================================
%\section{公式總整理}
%%=============================================================================
%
%\begin{center}
%\renewcommand{\arraystretch}{1.6}
%\begin{tabular}{ll}
%\toprule
%項目 & 公式 \\
%\midrule
%斜率估計 & $\hat{\beta}_1 = S_{XY}/S_{XX}$ \\
%截距估計 & $\hat{\beta}_0 = \bar{Y} - \hat{\beta}_1\bar{X}$ \\
%$\var(\hat{\beta}_1)$ & $\sigma^2/S_{XX}$ \\
%$\SE(\hat{\beta}_1)$ & $s/\sqrt{S_{XX}}$ \\
%判定係數 & $R^2 = \SSR/\SST = r_{XY}^2$ \\
%調整後 $R^2$ & $1 - \dfrac{n-1}{n-k-1}(1-R^2)$ \\
%$t$ 檢定 & $t = \hat{\beta}_j/\SE(\hat{\beta}_j) \sim t_{n-k-1}$ \\
%$F$ 檢定 & $F = \dfrac{R^2/k}{(1-R^2)/(n-k-1)} \sim F_{k, n-k-1}$ \\
%\bottomrule
%\end{tabular}
%\end{center}

\end{document}
