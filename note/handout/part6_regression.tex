\documentclass[12pt,a4paper]{article}
\usepackage[left=1cm,right=1cm,bottom=15mm,top=20mm]{geometry}
\usepackage[AutoFakeBold,AutoFakeSlant]{xeCJK}
\setCJKmainfont[AutoFakeSlant=.1,AutoFakeBold=2]{Noto Serif CJK TC}
\usepackage{amsmath,amsthm,amssymb,amsfonts}
\usepackage{graphicx,xcolor,float}
\usepackage{booktabs,tabularx,multirow,array}
\usepackage{parskip}
\usepackage{enumitem}
\setlist{itemsep=0pt,parsep=0pt}
\usepackage{hyperref}
\hypersetup{colorlinks=true, linkcolor=blue!70!black, urlcolor=blue!80!black}

% 機率與統計符號
\newcommand\expc{\mathsf{E}}
\newcommand\prb{\mathsf{P}}
\DeclareMathOperator\var{var}
\DeclareMathOperator\cov{cov}
\DeclareMathOperator\corr{corr}
\DeclareMathOperator\plim{plim}
\DeclareMathOperator\tr{tr}
\DeclareMathOperator\rank{rank}

% 中文化
\renewcommand{\figurename}{圖}
\renewcommand{\tablename}{表}

% proof 環境
\let\proof\relax
\let\endproof\relax
\newenvironment{proof}[1][證明]{%
  \par\noindent\textbf{#1}\quad
}{%
  \hfill$\square$\par\medskip
}

% 定理環境
\theoremstyle{definition}
\newtheorem{definition}{定義}[section]
\newtheorem{example}{例題}[section]
\newtheorem{exercise}{習題}[section]
\newtheorem{theorem}{定理}[section]
\newtheorem{lemma}{引理}[section]
\newtheorem{corollary}{推論}[section]
\newtheorem{proposition}{命題}[section]
\newtheorem{property}{性質}[section]
\newtheorem*{remark}{註}
\newtheorem*{solution}{解答}
\newtheorem*{note}{說明}

% 常用指令
\newcommand{\ds}{\displaystyle}
\newcommand{\ie}{\;\Longrightarrow\;}
\newcommand{\SST}{\mathrm{SST}}
\newcommand{\SSR}{\mathrm{SSR}}
\newcommand{\SSE}{\mathrm{SSE}}
\newcommand{\MSR}{\mathrm{MSR}}
\newcommand{\MSE}{\mathrm{MSE}}
\newcommand{\SE}{\mathrm{SE}}
\newcommand{\bX}{\mathbf{X}}
\newcommand{\bY}{\mathbf{Y}}
\newcommand{\by}{\mathbf{y}}
\newcommand{\bx}{\mathbf{x}}
\newcommand{\bb}{\boldsymbol{\beta}}
\newcommand{\be}{\boldsymbol{\varepsilon}}
\newcommand{\bI}{\mathbf{I}}
\newcommand{\bH}{\mathbf{H}}
\newcommand{\bM}{\mathbf{M}}
\newcommand{\bone}{\mathbf{1}}

% 頁面設定
\usepackage{fancyhdr}
\pagestyle{fancy}
\fancyhf{}
\fancyhead[L]{統計學講義}
\fancyhead[R]{第六部分:迴歸分析}
\fancyfoot[C]{\thepage}
\renewcommand{\headrulewidth}{0.4pt}
\renewcommand{\footrulewidth}{0.4pt}

\title{\vspace{-2cm}\textbf{統計學講義}\\[3mm] \Large 第六部分:迴歸分析}
\author{}
\date{\vspace{-2cm}}

\begin{document}
\maketitle
\thispagestyle{fancy}

%=============================================================================
\section{多元線性迴歸模型}
%=============================================================================

\subsection{模型設定}

\begin{definition}[多元線性迴歸模型]
設有 $n$ 個觀測值,每個觀測包含一個依變數 $Y_i$ 和 $k$ 個自變數 $X_{i1}, \ldots, X_{ik}$。\textbf{多元線性迴歸模型}假設:
\[
Y_i = \beta_0 + \beta_1 X_{i1} + \beta_2 X_{i2} + \cdots + \beta_k X_{ik} + \varepsilon_i, \quad i = 1, \ldots, n
\]
其中 $\beta_0$ 為截距,$\beta_1, \ldots, \beta_k$ 為斜率係數,$\varepsilon_i$ 為誤差項。
\end{definition}

\begin{definition}[矩陣表示法]
將模型寫成矩陣形式:
\[
\boxed{\bY = \bX\bb + \be}
\]
其中:
\[
\bY = \begin{pmatrix} Y_1 \\ Y_2 \\ \vdots \\ Y_n \end{pmatrix}_{n \times 1}, \quad
\bX = \begin{pmatrix} 1 & X_{11} & \cdots & X_{1k} \\ 1 & X_{21} & \cdots & X_{2k} \\ \vdots & \vdots & \ddots & \vdots \\ 1 & X_{n1} & \cdots & X_{nk} \end{pmatrix}_{n \times (k+1)}, \quad
\bb = \begin{pmatrix} \beta_0 \\ \beta_1 \\ \vdots \\ \beta_k \end{pmatrix}_{(k+1) \times 1}, \quad
\be = \begin{pmatrix} \varepsilon_1 \\ \varepsilon_2 \\ \vdots \\ \varepsilon_n \end{pmatrix}_{n \times 1}
\]
$\bX$ 稱為\textbf{設計矩陣}(design matrix),第一行全為 1 對應截距項。
\end{definition}

\begin{property}[古典假設]\label{prop:assumptions}
\begin{enumerate}[label=(A\arabic*)]
\item \textbf{線性關係}:$\expc[\bY | \bX] = \bX\bb$
\item \textbf{誤差期望為零}:$\expc[\be | \bX] = \mathbf{0}$
\item \textbf{球面誤差}(spherical errors):$\var(\be | \bX) = \sigma^2 \bI_n$
\begin{itemize}
    \item 同質變異數(homoscedasticity):$\var(\varepsilon_i) = \sigma^2$(常數)
    \item 誤差不相關:$\cov(\varepsilon_i, \varepsilon_j) = 0$ 對 $i \neq j$
\end{itemize}
\item \textbf{滿秩假設}:$\rank(\bX) = k + 1$($\bX$ 的行向量線性獨立)
\item \textbf{常態假設}(用於精確推論):$\be | \bX \sim N(\mathbf{0}, \sigma^2 \bI_n)$
\end{enumerate}
\end{property}

\begin{remark}
假設 (A1)--(A4) 稱為 \textbf{Gauss-Markov 假設},保證 OLS 為 BLUE:
\begin{itemize}
\item \textbf{OLS}:Ordinary Least Squares,普通最小平方法
\item \textbf{BLUE}:Best Linear Unbiased Estimator,最佳線性不偏估計量
\end{itemize}
假設 (A5) 用於有限樣本下的精確 $t$ 和 $F$ 檢定。
\end{remark}

%=============================================================================
\section{最小平方法}
%=============================================================================

\subsection{OLS 估計量}

\begin{definition}[最小平方法]
\textbf{普通最小平方法}(Ordinary Least Squares, OLS)選擇 $\hat{\bb}$ 以最小化殘差平方和:
\[
\SSE(\bb) = \sum_{i=1}^{n} (Y_i - \bx_i^\top\bb)^2 = (\bY - \bX\bb)^\top(\bY - \bX\bb) = \|\bY - \bX\bb\|^2
\]
\end{definition}

\begin{lemma}[矩陣微分公式]\label{lem:matrix-diff}
設 $\bb$ 為 $p \times 1$ 向量,$\mathbf{a}$ 為 $p \times 1$ 常數向量,$\mathbf{A}$ 為 $p \times p$ 常數矩陣。則:
\begin{enumerate}[label=(\roman*)]
  \item[]
\item $\dfrac{\partial}{\partial \bb}(\mathbf{a}^\top\bb) = \dfrac{\partial}{\partial \bb}(\bb^\top\mathbf{a}) = \mathbf{a}$
\item $\dfrac{\partial}{\partial \bb}(\bb^\top\mathbf{A}\bb) = (\mathbf{A} + \mathbf{A}^\top)\bb$

當 $\mathbf{A}$ 為對稱矩陣($\mathbf{A} = \mathbf{A}^\top$)時,簡化為 $2\mathbf{A}\bb$。
\end{enumerate}
\end{lemma}

\begin{proof}
(i) 設 $\mathbf{a} = (a_1, \ldots, a_p)^\top$,則 $\mathbf{a}^\top\bb = \sum_{j=1}^{p} a_j \beta_j$。對 $\beta_i$ 微分:$\frac{\partial}{\partial \beta_i}(\mathbf{a}^\top\bb) = a_i$。故 $\frac{\partial}{\partial \bb}(\mathbf{a}^\top\bb) = \mathbf{a}$。

(ii) 設 $\mathbf{A} = (a_{ij})$,則 $\bb^\top\mathbf{A}\bb = \sum_{i=1}^{p}\sum_{j=1}^{p} a_{ij}\beta_i\beta_j$。對 $\beta_\ell$ 微分:
\begin{align*}
\frac{\partial}{\partial \beta_\ell}(\bb^\top\mathbf{A}\bb) &= \sum_{j=1}^{p} a_{\ell j}\beta_j + \sum_{i=1}^{p} a_{i\ell}\beta_i = (\mathbf{A}\bb)_\ell + (\mathbf{A}^\top\bb)_\ell
\end{align*}
故 $\frac{\partial}{\partial \bb}(\bb^\top\mathbf{A}\bb) = \mathbf{A}\bb + \mathbf{A}^\top\bb = (\mathbf{A} + \mathbf{A}^\top)\bb$。

當 $\mathbf{A} = \mathbf{A}^\top$ 時,$(\mathbf{A} + \mathbf{A}^\top)\bb = 2\mathbf{A}\bb$。
\end{proof}

\begin{theorem}[OLS 估計量]\label{thm:ols}
在假設 (A4) 下,OLS 估計量為:
\[
\boxed{\hat{\bb} = (\bX^\top\bX)^{-1}\bX^\top\bY}
\]
\end{theorem}

\begin{proof}
\textbf{方法一:微分法}

展開 $\SSE$:
\begin{align*}
\SSE(\bb) &= (\bY - \bX\bb)^\top(\bY - \bX\bb)\\
&= \bY^\top\bY - \bY^\top\bX\bb - \bb^\top\bX^\top\bY + \bb^\top\bX^\top\bX\bb\\
&= \bY^\top\bY - 2\bb^\top\bX^\top\bY + \bb^\top\bX^\top\bX\bb
\end{align*}
(第二步利用 $\bY^\top\bX\bb$ 為純量,故等於其轉置 $\bb^\top\bX^\top\bY$。)

對 $\bb$ 微分。由引理~\ref{lem:matrix-diff},$\bX^\top\bX$ 為對稱矩陣,故:
\[
\frac{\partial \SSE}{\partial \bb} = -2\bX^\top\bY + 2\bX^\top\bX\bb = \mathbf{0}
\]

這給出\textbf{正規方程式}(normal equations):
\[
\bX^\top\bX\hat{\bb} = \bX^\top\bY
\]

由假設 (A4),$\bX^\top\bX$ 可逆,故:
\[
\hat{\bb} = (\bX^\top\bX)^{-1}\bX^\top\bY
\]

\textbf{二階條件}:Hessian 矩陣 $\frac{\partial^2 \SSE}{\partial \bb \partial \bb^\top} = 2\bX^\top\bX$ 為正定(當 $\rank(\bX) = k+1$ 時),確認為最小值。

\textbf{方法二:投影法}

$\hat{\bY} = \bX\hat{\bb}$ 是 $\bY$ 在 $\bX$ 的行空間(column space)上的正交投影。正交條件要求殘差 $\mathbf{e} = \bY - \hat{\bY}$ 與 $\bX$ 的每一行正交:
\[
\bX^\top(\bY - \bX\hat{\bb}) = \mathbf{0} \ie \bX^\top\bY = \bX^\top\bX\hat{\bb} \ie \hat{\bb} = (\bX^\top\bX)^{-1}\bX^\top\bY
\]
\end{proof}

\begin{definition}[帽子矩陣與殘差製造矩陣]
定義\textbf{帽子矩陣}(hat matrix):
\[
\bH = \bX(\bX^\top\bX)^{-1}\bX^\top
\]
則擬合值為 $\hat{\bY} = \bH\bY$,殘差為 $\mathbf{e} = \bY - \hat{\bY} = (\bI - \bH)\bY = \bM\bY$,其中 $\bM = \bI - \bH$ 為\textbf{殘差製造矩陣}(residual maker matrix)。
\end{definition}

\begin{definition}[冪等矩陣]
矩陣 $\mathbf{A}$ 稱為\textbf{冪等矩陣}(idempotent matrix),若 $\mathbf{A}^2 = \mathbf{A}$。
\end{definition}

\begin{lemma}[冪等矩陣的跡與秩]\label{lem:idempotent}
設 $\mathbf{A}$ 為 $n \times n$ 冪等矩陣,則 $\tr(\mathbf{A}) = \rank(\mathbf{A})$。
\end{lemma}

\begin{proof}
由於 $\mathbf{A}^2 = \mathbf{A}$,$\mathbf{A}$ 的特徵值 $\lambda$ 滿足 $\lambda^2 = \lambda$,即 $\lambda(\lambda - 1) = 0$,故 $\lambda \in \{0, 1\}$。

設 $\mathbf{A}$ 有 $r$ 個特徵值為 1,其餘 $n - r$ 個為 0。則:
\begin{itemize}
\item $\tr(\mathbf{A}) = r \cdot 1 + (n-r) \cdot 0 = r$
\item $\rank(\mathbf{A}) = r$(等於非零特徵值個數)
\end{itemize}
故 $\tr(\mathbf{A}) = \rank(\mathbf{A})$。
\end{proof}

\begin{theorem}[標準常態向量的二次型分配]\label{thm:quadratic-chisq}
設 $\mathbf{z} \sim N(\mathbf{0}, \bI_n)$,$\mathbf{A}$ 為 $n \times n$ 對稱冪等矩陣($\mathbf{A}^\top = \mathbf{A}$,$\mathbf{A}^2 = \mathbf{A}$)。則:
\[
\mathbf{z}^\top\mathbf{A}\mathbf{z} \sim \chi^2_r, \quad \text{其中 } r = \rank(\mathbf{A}) = \tr(\mathbf{A})
\]
\end{theorem}

\begin{proof}
\textbf{步驟一:對稱冪等矩陣的譜分解}

由於 $\mathbf{A}$ 對稱,存在正交矩陣 $\mathbf{P}$($\mathbf{P}^\top\mathbf{P} = \mathbf{P}\mathbf{P}^\top = \bI$)使得:
\[
\mathbf{A} = \mathbf{P}\boldsymbol{\Lambda}\mathbf{P}^\top
\]
其中 $\boldsymbol{\Lambda} = \text{diag}(\lambda_1, \ldots, \lambda_n)$ 為特徵值對角矩陣。

由引理~\ref{lem:idempotent},$\mathbf{A}$ 的特徵值只能是 0 或 1。設有 $r$ 個特徵值為 1,不失一般性令:
\[
\boldsymbol{\Lambda} = \text{diag}(\underbrace{1, \ldots, 1}_{r \text{ 個}}, \underbrace{0, \ldots, 0}_{n-r \text{ 個}})
\]

\textbf{步驟二:正交變換保持標準常態性}

令 $\mathbf{w} = \mathbf{P}^\top\mathbf{z}$。由於 $\mathbf{z} \sim N(\mathbf{0}, \bI_n)$:
\begin{itemize}
\item $\expc[\mathbf{w}] = \mathbf{P}^\top\expc[\mathbf{z}] = \mathbf{0}$
\item $\var(\mathbf{w}) = \mathbf{P}^\top\var(\mathbf{z})\mathbf{P} = \mathbf{P}^\top\bI\mathbf{P} = \bI_n$
\end{itemize}
故 $\mathbf{w} \sim N(\mathbf{0}, \bI_n)$,即 $w_1, \ldots, w_n$ 為獨立標準常態變數。

\textbf{步驟三:計算二次型}

\begin{align*}
\mathbf{z}^\top\mathbf{A}\mathbf{z} &= \mathbf{z}^\top\mathbf{P}\boldsymbol{\Lambda}\mathbf{P}^\top\mathbf{z}\\
&= (\mathbf{P}^\top\mathbf{z})^\top\boldsymbol{\Lambda}(\mathbf{P}^\top\mathbf{z})\\
&= \mathbf{w}^\top\boldsymbol{\Lambda}\mathbf{w}\\
&= \sum_{i=1}^{n} \lambda_i w_i^2\\
&= \sum_{i=1}^{r} 1 \cdot w_i^2 + \sum_{i=r+1}^{n} 0 \cdot w_i^2\\
&= \sum_{i=1}^{r} w_i^2
\end{align*}

\textbf{步驟四:結論}

由於 $w_1, \ldots, w_r$ 為獨立標準常態變數,$\sum_{i=1}^{r} w_i^2 \sim \chi^2_r$。
\end{proof}

\begin{property}[$\bH$ 和 $\bM$ 的性質]\label{prop:HM}
\begin{enumerate}[label=(\roman*)]
  \item[]
\item $\bH$ 和 $\bM$ 皆為\textbf{對稱冪等矩陣}(symmetric idempotent):$\bH^\top = \bH$,$\bH^2 = \bH$;$\bM^\top = \bM$,$\bM^2 = \bM$
\item $\bH\bX = \bX$,$\bM\bX = \mathbf{0}$
\item $\bH\bM = \bM\bH = \mathbf{0}$
\item $\tr(\bH) = k + 1$,$\tr(\bM) = n - k - 1$
\end{enumerate}
\end{property}

\begin{proof}
(i) \textbf{$\bH$ 的對稱性}:
\[
\bH^\top = [\bX(\bX^\top\bX)^{-1}\bX^\top]^\top = \bX[(\bX^\top\bX)^{-1}]^\top\bX^\top = \bX(\bX^\top\bX)^{-1}\bX^\top = \bH
\]
(利用 $(\bX^\top\bX)^{-1}$ 對稱,因為 $\bX^\top\bX$ 對稱。)

\textbf{$\bH$ 的冪等性}:
\[
\bH^2 = \bX(\bX^\top\bX)^{-1}\bX^\top \cdot \bX(\bX^\top\bX)^{-1}\bX^\top = \bX(\bX^\top\bX)^{-1}\bX^\top = \bH
\]

$\bM = \bI - \bH$ 的對稱性和冪等性由 $\bH$ 直接推得。

(ii) $\bH\bX = \bX(\bX^\top\bX)^{-1}\bX^\top\bX = \bX\bI = \bX$

$\bM\bX = (\bI - \bH)\bX = \bX - \bX = \mathbf{0}$

(iii) $\bH\bM = \bH(\bI - \bH) = \bH - \bH^2 = \bH - \bH = \mathbf{0}$

(iv) \textbf{$\tr(\bH)$ 的計算}:

由引理~\ref{lem:idempotent},$\tr(\bH) = \rank(\bH)$。

$\bH$ 將任意向量投影到 $\bX$ 的行空間(維度 $k+1$),故 $\rank(\bH) = k + 1$。

因此 $\tr(\bH) = k + 1$。

\textbf{$\tr(\bM)$ 的計算}:
\[
\tr(\bM) = \tr(\bI - \bH) = \tr(\bI) - \tr(\bH) = n - (k+1) = n - k - 1
\]
\end{proof}

\begin{property}[正規方程式的意涵]\label{prop:normal-eq}
OLS 殘差滿足:
\begin{enumerate}[label=(\roman*)]
\item $\bX^\top\mathbf{e} = \mathbf{0}$(殘差與所有自變數正交)
\item $\sum_{i=1}^{n} e_i = 0$(殘差和為零,因為 $\bX$ 包含截距項)
\item $\hat{\bY}^\top\mathbf{e} = 0$(殘差與擬合值正交)
\end{enumerate}
\end{property}

\subsection{OLS 估計量的統計性質}

\begin{theorem}[不偏性]\label{thm:unbiased}
在假設 (A1)--(A2) 下:
\[
\boxed{\expc[\hat{\bb} | \bX] = \bb}
\]
\end{theorem}

\begin{proof}
將 $\bY = \bX\bb + \be$ 代入 $\hat{\bb} = (\bX^\top\bX)^{-1}\bX^\top\bY$:
\begin{align*}
\hat{\bb} &= (\bX^\top\bX)^{-1}\bX^\top(\bX\bb + \be)\\
&= (\bX^\top\bX)^{-1}\bX^\top\bX\bb + (\bX^\top\bX)^{-1}\bX^\top\be\\
&= \bb + (\bX^\top\bX)^{-1}\bX^\top\be
\end{align*}

取條件期望:
\[
\expc[\hat{\bb} | \bX] = \bb + (\bX^\top\bX)^{-1}\bX^\top\expc[\be | \bX] = \bb + (\bX^\top\bX)^{-1}\bX^\top \cdot \mathbf{0} = \bb
\]
\end{proof}

\begin{theorem}[變異數-共變異數矩陣]\label{thm:var-beta}
在假設 (A1)--(A4) 下:
\[
\boxed{\var(\hat{\bb} | \bX) = \sigma^2 (\bX^\top\bX)^{-1}}
\]
\end{theorem}

\begin{proof}
由 $\hat{\bb} - \bb = (\bX^\top\bX)^{-1}\bX^\top\be$:
\begin{align*}
\var(\hat{\bb} | \bX) &= \expc[(\hat{\bb} - \bb)(\hat{\bb} - \bb)^\top | \bX]\\
&= \expc\left[(\bX^\top\bX)^{-1}\bX^\top\be\be^\top\bX(\bX^\top\bX)^{-1} | \bX\right]\\
&= (\bX^\top\bX)^{-1}\bX^\top \expc[\be\be^\top | \bX] \bX(\bX^\top\bX)^{-1}\\
&= (\bX^\top\bX)^{-1}\bX^\top (\sigma^2\bI) \bX(\bX^\top\bX)^{-1} \quad \text{(由假設 A3)}\\
&= \sigma^2 (\bX^\top\bX)^{-1}\bX^\top\bX(\bX^\top\bX)^{-1}\\
&= \sigma^2 (\bX^\top\bX)^{-1}
\end{align*}
\end{proof}

\begin{remark}
$(\bX^\top\bX)^{-1}$ 的對角元素乘以 $\sigma^2$ 給出各 $\hat{\beta}_j$ 的變異數,非對角元素乘以 $\sigma^2$ 給出共變異數。
\end{remark}

\begin{theorem}[Gauss-Markov 定理]\label{thm:gm}
在假設 (A1)--(A4) 下,OLS 估計量 $\hat{\bb}$ 是 $\bb$ 的\textbf{最佳線性不偏估計量}(Best Linear Unbiased Estimator, BLUE)。

即:對於任意線性不偏估計量 $\tilde{\bb} = \mathbf{C}\bY$(其中 $\mathbf{C}$ 為常數矩陣),有
\[
\var(\tilde{\beta}_j) \geqslant \var(\hat{\beta}_j), \quad \forall j
\]
或更一般地,$\var(\tilde{\bb}) - \var(\hat{\bb})$ 為半正定矩陣。
\end{theorem}

\begin{proof}
設 $\tilde{\bb} = \mathbf{C}\bY$ 為 $\bb$ 的線性不偏估計量,其中 $\mathbf{C}$ 為 $(k+1) \times n$ 常數矩陣。

\textbf{步驟一:不偏性的必要條件}

$\expc[\tilde{\bb}] = \mathbf{C}\expc[\bY] = \mathbf{C}\bX\bb = \bb$ 對所有 $\bb$ 成立,故必須 $\mathbf{C}\bX = \bI_{k+1}$。

\textbf{步驟二:分解 $\mathbf{C}$}

令 $\mathbf{C} = (\bX^\top\bX)^{-1}\bX^\top + \mathbf{D}$,其中 $\mathbf{D} = \mathbf{C} - (\bX^\top\bX)^{-1}\bX^\top$。

由 $\mathbf{C}\bX = \bI$:
\[
\mathbf{D}\bX = \mathbf{C}\bX - (\bX^\top\bX)^{-1}\bX^\top\bX = \bI - \bI = \mathbf{0}
\]

\textbf{步驟三:計算 $\var(\tilde{\bb})$}

\begin{align*}
\var(\tilde{\bb}) &= \var(\mathbf{C}\bY) = \mathbf{C} \var(\bY) \mathbf{C}^\top = \sigma^2 \mathbf{C}\mathbf{C}^\top\\
&= \sigma^2 \left[(\bX^\top\bX)^{-1}\bX^\top + \mathbf{D}\right]\left[\bX(\bX^\top\bX)^{-1} + \mathbf{D}^\top\right]\\
&= \sigma^2 \left[(\bX^\top\bX)^{-1}\bX^\top\bX(\bX^\top\bX)^{-1} + (\bX^\top\bX)^{-1}\bX^\top\mathbf{D}^\top + \mathbf{D}\bX(\bX^\top\bX)^{-1} + \mathbf{D}\mathbf{D}^\top\right]
\end{align*}

由 $\mathbf{D}\bX = \mathbf{0}$,中間兩項為零($(\bX^\top\bX)^{-1}\bX^\top\mathbf{D}^\top = [(\mathbf{D}\bX)(\bX^\top\bX)^{-1}]^\top = \mathbf{0}$):
\[
\var(\tilde{\bb}) = \sigma^2 (\bX^\top\bX)^{-1} + \sigma^2 \mathbf{D}\mathbf{D}^\top = \var(\hat{\bb}) + \sigma^2 \mathbf{D}\mathbf{D}^\top
\]

由於 $\mathbf{D}\mathbf{D}^\top$ 為半正定矩陣,$\var(\tilde{\bb}) - \var(\hat{\bb}) = \sigma^2 \mathbf{D}\mathbf{D}^\top \succeq \mathbf{0}$。

等號成立當且僅當 $\mathbf{D} = \mathbf{0}$,即 $\mathbf{C} = (\bX^\top\bX)^{-1}\bX^\top$,此時 $\tilde{\bb} = \hat{\bb}$。
\end{proof}

\subsection{$\sigma^2$ 的估計}

\begin{lemma}[二次型的期望值]\label{lem:quadratic-form}
設 $\mathbf{z}$ 為 $n \times 1$ 隨機向量,$\expc[\mathbf{z}] = \boldsymbol{\mu}$,$\var(\mathbf{z}) = \boldsymbol{\Sigma}$。設 $\mathbf{A}$ 為 $n \times n$ 常數矩陣。則:
\[
\expc[\mathbf{z}^\top\mathbf{A}\mathbf{z}] = \tr(\mathbf{A}\boldsymbol{\Sigma}) + \boldsymbol{\mu}^\top\mathbf{A}\boldsymbol{\mu}
\]
特別地,當 $\expc[\mathbf{z}] = \mathbf{0}$ 時,$\expc[\mathbf{z}^\top\mathbf{A}\mathbf{z}] = \tr(\mathbf{A}\boldsymbol{\Sigma})$。
\end{lemma}

\begin{proof}
$\mathbf{z}^\top\mathbf{A}\mathbf{z} = \sum_{i=1}^{n}\sum_{j=1}^{n} a_{ij} z_i z_j$ 為純量。取期望:
\[
\expc[\mathbf{z}^\top\mathbf{A}\mathbf{z}] = \sum_{i=1}^{n}\sum_{j=1}^{n} a_{ij} \expc[z_i z_j]
\]

由 $\cov(z_i, z_j) = \expc[z_i z_j] - \mu_i\mu_j$,得 $\expc[z_i z_j] = \Sigma_{ij} + \mu_i\mu_j$。代入:
\begin{align*}
\expc[\mathbf{z}^\top\mathbf{A}\mathbf{z}] &= \sum_{i,j} a_{ij}(\Sigma_{ij} + \mu_i\mu_j)\\
&= \sum_{i,j} a_{ij}\Sigma_{ij} + \sum_{i,j} a_{ij}\mu_i\mu_j\\
&= \tr(\mathbf{A}\boldsymbol{\Sigma}) + \boldsymbol{\mu}^\top\mathbf{A}\boldsymbol{\mu}
\end{align*}
(第一項利用 $\tr(\mathbf{A}\boldsymbol{\Sigma}) = \sum_i (\mathbf{A}\boldsymbol{\Sigma})_{ii} = \sum_i \sum_j a_{ij}\Sigma_{ji} = \sum_{i,j} a_{ij}\Sigma_{ij}$,因 $\boldsymbol{\Sigma}$ 對稱。)
\end{proof}

\begin{theorem}[$\MSE$ 的不偏性]\label{thm:mse-unbiased}
\[
s^2 = \MSE = \frac{\SSE}{n - k - 1} = \frac{\mathbf{e}^\top\mathbf{e}}{n - k - 1}
\]
是 $\sigma^2$ 的不偏估計量。
\end{theorem}

\begin{proof}
\textbf{步驟一:將 $\SSE$ 表示為誤差的二次型}

由 $\mathbf{e} = \bM\bY = \bM(\bX\bb + \be) = \bM\bX\bb + \bM\be = \bM\be$(因為 $\bM\bX = \mathbf{0}$)。

故:
\[
\SSE = \mathbf{e}^\top\mathbf{e} = (\bM\be)^\top(\bM\be) = \be^\top\bM^\top\bM\be = \be^\top\bM\be
\]
(最後一步利用 $\bM$ 對稱冪等:$\bM^\top\bM = \bM\bM = \bM$。)

\textbf{步驟二:計算 $\expc[\SSE]$}

應用引理~\ref{lem:quadratic-form},取 $\mathbf{z} = \be$,$\mathbf{A} = \bM$。

由假設 (A2),$\expc[\be] = \mathbf{0}$(即 $\boldsymbol{\mu} = \mathbf{0}$)。由假設 (A3),$\var(\be) = \sigma^2\bI$(即 $\boldsymbol{\Sigma} = \sigma^2\bI$)。

由引理~\ref{lem:quadratic-form}:
\begin{align*}
\expc[\SSE] &= \expc[\be^\top\bM\be]\\
&= \tr(\bM \cdot \sigma^2\bI) + \mathbf{0}^\top\bM\mathbf{0}\\
&= \sigma^2 \tr(\bM\bI) + 0\\
&= \sigma^2 \tr(\bM)
\end{align*}

\textbf{步驟三:計算 $\tr(\bM)$}

由性質~\ref{prop:HM}(iv),$\tr(\bM) = n - k - 1$。

故 $\expc[\SSE] = \sigma^2 (n - k - 1)$。

\textbf{步驟四:結論}
\[
\expc[\MSE] = \expc\left[\frac{\SSE}{n-k-1}\right] = \frac{\expc[\SSE]}{n-k-1} = \frac{\sigma^2(n-k-1)}{n-k-1} = \sigma^2
\]
故 $\MSE$ 是 $\sigma^2$ 的不偏估計量。
\end{proof}

%=============================================================================
\section{簡單線性迴歸:$k = 1$ 的特例}
%=============================================================================

當只有一個自變數時($k = 1$),多元迴歸退化為\textbf{簡單線性迴歸}:
\[
Y_i = \beta_0 + \beta_1 X_i + \varepsilon_i
\]

此時設計矩陣為:
\[
\bX = \begin{pmatrix} 1 & X_1 \\ 1 & X_2 \\ \vdots & \vdots \\ 1 & X_n \end{pmatrix}
\]

\begin{theorem}[簡單迴歸的顯式公式]
當 $k = 1$ 時,OLS 估計量為:
\[
\boxed{\hat{\beta}_1 = \frac{S_{XY}}{S_{XX}} = \frac{\sum_{i=1}^{n}(X_i - \bar{X})(Y_i - \bar{Y})}{\sum_{i=1}^{n}(X_i - \bar{X})^2}}
\]
\[
\boxed{\hat{\beta}_0 = \bar{Y} - \hat{\beta}_1 \bar{X}}
\]
其中 $S_{XX} = \sum(X_i - \bar{X})^2$,$S_{XY} = \sum(X_i - \bar{X})(Y_i - \bar{Y})$。
\end{theorem}

\begin{proof}
計算 $\bX^\top\bX$ 和 $\bX^\top\bY$:
\[
\bX^\top\bX = \begin{pmatrix} 1 & \cdots & 1 \\ X_1 & \cdots & X_n \end{pmatrix} \begin{pmatrix} 1 & X_1 \\ \vdots & \vdots \\ 1 & X_n \end{pmatrix} = \begin{pmatrix} n & \sum X_i \\ \sum X_i & \sum X_i^2 \end{pmatrix} = \begin{pmatrix} n & n\bar{X} \\ n\bar{X} & \sum X_i^2 \end{pmatrix}
\]
\[
\bX^\top\bY = \begin{pmatrix} \sum Y_i \\ \sum X_i Y_i \end{pmatrix} = \begin{pmatrix} n\bar{Y} \\ \sum X_i Y_i \end{pmatrix}
\]

$\bX^\top\bX$ 的行列式:$\det(\bX^\top\bX) = n\sum X_i^2 - n^2\bar{X}^2 = n(\sum X_i^2 - n\bar{X}^2) = n \cdot S_{XX}$

$(\bX^\top\bX)^{-1} = \frac{1}{n \cdot S_{XX}} \begin{pmatrix} \sum X_i^2 & -n\bar{X} \\ -n\bar{X} & n \end{pmatrix}$

由 $\hat{\bb} = (\bX^\top\bX)^{-1}\bX^\top\bY$:
\begin{align*}
\hat{\beta}_1 &= \frac{1}{n \cdot S_{XX}}\left[-n\bar{X} \cdot n\bar{Y} + n \sum X_i Y_i\right] = \frac{n(\sum X_i Y_i - n\bar{X}\bar{Y})}{n \cdot S_{XX}} = \frac{S_{XY}}{S_{XX}}
\end{align*}

由正規方程式第一列:$n\hat{\beta}_0 + n\bar{X}\hat{\beta}_1 = n\bar{Y} \ie \hat{\beta}_0 = \bar{Y} - \hat{\beta}_1\bar{X}$。
\end{proof}

\begin{corollary}[簡單迴歸估計量的變異數]
\[
\var(\hat{\beta}_1) = \frac{\sigma^2}{S_{XX}}, \qquad \var(\hat{\beta}_0) = \sigma^2\left(\frac{1}{n} + \frac{\bar{X}^2}{S_{XX}}\right)
\]
\end{corollary}

\begin{proof}
由 $\var(\hat{\bb}) = \sigma^2(\bX^\top\bX)^{-1}$:
\[
\var(\hat{\bb}) = \frac{\sigma^2}{n \cdot S_{XX}} \begin{pmatrix} \sum X_i^2 & -n\bar{X} \\ -n\bar{X} & n \end{pmatrix}
\]

$\var(\hat{\beta}_1) = \frac{\sigma^2 \cdot n}{n \cdot S_{XX}} = \frac{\sigma^2}{S_{XX}}$

$\var(\hat{\beta}_0) = \frac{\sigma^2 \sum X_i^2}{n \cdot S_{XX}} = \frac{\sigma^2(\sum X_i^2 - n\bar{X}^2 + n\bar{X}^2)}{n \cdot S_{XX}} = \frac{\sigma^2 S_{XX}}{n \cdot S_{XX}} + \frac{\sigma^2 n\bar{X}^2}{n \cdot S_{XX}} = \sigma^2\left(\frac{1}{n} + \frac{\bar{X}^2}{S_{XX}}\right)$
\end{proof}

%=============================================================================
\section{判定係數與 ANOVA}
%=============================================================================

\subsection{平方和分解}

\begin{theorem}[變異數分解]
\[
\boxed{\SST = \SSR + \SSE}
\]
其中:
\begin{itemize}
\item $\SST = \bY^\top\bM_0\bY = \sum(Y_i - \bar{Y})^2$(總平方和, Total Sum of Squares),$\bM_0 = \bI - \frac{1}{n}\bone\bone^\top$
\item $\SSR = \hat{\bY}^\top\bM_0\hat{\bY} = \sum(\hat{Y}_i - \bar{Y})^2$(迴歸平方和, Regression Sum of Squares)
\item $\SSE = \mathbf{e}^\top\mathbf{e} = \sum(Y_i - \hat{Y}_i)^2$(殘差平方和, Error Sum of Squares)
\end{itemize}
\end{theorem}

\begin{proof}
$\bY - \bar{Y}\bone = (\bY - \hat{\bY}) + (\hat{\bY} - \bar{Y}\bone) = \mathbf{e} + (\hat{\bY} - \bar{Y}\bone)$

平方展開:$\SST = \mathbf{e}^\top\mathbf{e} + (\hat{\bY} - \bar{Y}\bone)^\top(\hat{\bY} - \bar{Y}\bone) + 2\mathbf{e}^\top(\hat{\bY} - \bar{Y}\bone)$

交叉項:$\mathbf{e}^\top(\hat{\bY} - \bar{Y}\bone) = \mathbf{e}^\top\hat{\bY} - \bar{Y}\mathbf{e}^\top\bone = 0 - \bar{Y} \cdot 0 = 0$(由正規方程式)

故 $\SST = \SSE + \SSR$。
\end{proof}

\subsection{判定係數}

\begin{definition}[判定係數 $R^2$]
\[
\boxed{R^2 = \frac{\SSR}{\SST} = 1 - \frac{\SSE}{\SST}}
\]
$R^2$ 表示自變數能解釋依變數總變異的比例,$0 \leqslant R^2 \leqslant 1$。
\end{definition}

\begin{definition}[調整後判定係數]\label{def:adj-R2}
\[
\boxed{\bar{R}^2 = 1 - \frac{n-1}{n-k-1}(1 - R^2) = 1 - \frac{\SSE/(n-k-1)}{\SST/(n-1)} = 1 - \frac{\MSE}{s_Y^2}}
\]
其中 $s_Y^2 = \SST/(n-1)$ 為 $Y$ 的樣本變異數。
\end{definition}

\begin{remark}[自由度的來源]
調整後 $R^2$ 的分子分母分別除以相應的自由度:
\begin{itemize}
\item \textbf{分母 $n - 1$}:$\SST = \sum(Y_i - \bar{Y})^2$ 有 $n$ 項,但受到一個約束 $\sum(Y_i - \bar{Y}) = 0$,故自由度為 $n - 1$。
\item \textbf{分子 $n - k - 1$}:$\SSE = \sum e_i^2$ 有 $n$ 項,但受到 $k + 1$ 個約束(正規方程式 $\bX^\top\mathbf{e} = \mathbf{0}$ 有 $k + 1$ 個方程),故自由度為 $n - (k + 1) = n - k - 1$。
\end{itemize}
調整後 $R^2$ 對增加的自變數進行懲罰。若增加不顯著變數,$\bar{R}^2$ 可能下降。$\bar{R}^2$ 可為負。
\end{remark}

\begin{theorem}[簡單迴歸中 $R^2 = r^2$]
在簡單線性迴歸($k = 1$)中,$R^2 = r_{XY}^2$,其中 $r_{XY}$ 為 $X$ 與 $Y$ 的樣本相關係數。
\end{theorem}

\begin{proof}
$R^2 = \SSR/\SST = \hat{\beta}_1^2 S_{XX}/S_{YY} = (S_{XY}/S_{XX})^2 \cdot S_{XX}/S_{YY} = S_{XY}^2/(S_{XX} S_{YY}) = r_{XY}^2$
\end{proof}

%=============================================================================
\section{假設檢定}
%=============================================================================

\subsection{抽樣分配}

\begin{theorem}[OLS 估計量的抽樣分配]\label{thm:beta-dist}
在假設 (A1)--(A5) 下:
\[
\hat{\bb} | \bX \sim N\left(\bb, \sigma^2(\bX^\top\bX)^{-1}\right)
\]
\end{theorem}

\begin{proof}
由 $\hat{\bb} = \bb + (\bX^\top\bX)^{-1}\bX^\top\be$,$\hat{\bb}$ 是 $\be$ 的線性變換。

在假設 (A5) 下,$\be \sim N(\mathbf{0}, \sigma^2\bI)$。

常態分配的線性變換仍為常態:若 $\mathbf{z} \sim N(\boldsymbol{\mu}, \boldsymbol{\Sigma})$,則 $\mathbf{A}\mathbf{z} + \mathbf{b} \sim N(\mathbf{A}\boldsymbol{\mu} + \mathbf{b}, \mathbf{A}\boldsymbol{\Sigma}\mathbf{A}^\top)$。

取 $\mathbf{A} = (\bX^\top\bX)^{-1}\bX^\top$,$\mathbf{z} = \be$,$\mathbf{b} = \bb$:
\[
\hat{\bb} = \bb + (\bX^\top\bX)^{-1}\bX^\top\be \sim N\left(\bb + (\bX^\top\bX)^{-1}\bX^\top \cdot \mathbf{0}, (\bX^\top\bX)^{-1}\bX^\top(\sigma^2\bI)\bX(\bX^\top\bX)^{-1}\right)
\]
\[
= N\left(\bb, \sigma^2(\bX^\top\bX)^{-1}\right)
\]
\end{proof}

\begin{theorem}[$\SSE/\sigma^2$ 的分配]\label{thm:sse-dist}
在假設 (A1)--(A5) 下:
\[
\frac{\SSE}{\sigma^2} \sim \chi^2_{n-k-1}
\]
且 $\SSE$ 與 $\hat{\bb}$ 獨立。
\end{theorem}

\begin{proof}
\textbf{步驟一:標準化}

令 $\mathbf{z} = \be/\sigma$,則 $\mathbf{z} \sim N(\mathbf{0}, \bI_n)$(標準常態向量)。

由 $\mathbf{e} = \bM\be$:
\[
\frac{\SSE}{\sigma^2} = \frac{\be^\top\bM\be}{\sigma^2} = \mathbf{z}^\top\bM\mathbf{z}
\]

\textbf{步驟二:二次型的分配}

由定理~\ref{thm:quadratic-chisq},若 $\mathbf{z} \sim N(\mathbf{0}, \bI_n)$ 且 $\mathbf{A}$ 為對稱冪等矩陣,則 $\mathbf{z}^\top\mathbf{A}\mathbf{z} \sim \chi^2_r$,其中 $r = \tr(\mathbf{A})$。

$\bM$ 為對稱冪等矩陣,$\tr(\bM) = n - k - 1$,故:
\[
\frac{\SSE}{\sigma^2} = \mathbf{z}^\top\bM\mathbf{z} \sim \chi^2_{n-k-1}
\]

\textbf{步驟三:獨立性}

$\hat{\bb} = \bb + (\bX^\top\bX)^{-1}\bX^\top\be$ 與 $\mathbf{e} = \bM\be$ 的獨立性:

兩者皆為 $\be$ 的線性變換。由常態分配的性質,獨立性等價於共變異數為零:
\begin{align*}
\cov\left((\bX^\top\bX)^{-1}\bX^\top\be, \bM\be\right) &= (\bX^\top\bX)^{-1}\bX^\top \var(\be) \bM^\top\\
&= (\bX^\top\bX)^{-1}\bX^\top (\sigma^2\bI) \bM\\
&= \sigma^2 (\bX^\top\bX)^{-1}\bX^\top\bM = \mathbf{0}
\end{align*}
(最後一步利用 $\bX^\top\bM = (\bM\bX)^\top = \mathbf{0}^\top = \mathbf{0}$。)
\end{proof}

\subsection{整體 $F$ 檢定}

\begin{table}[H]
\centering
\caption{多元迴歸 ANOVA 表}
\renewcommand{\arraystretch}{1.4}
\begin{tabular}{l|c|c|c|c}
\toprule
\textbf{來源} & \textbf{SS} & \textbf{df} & \textbf{MS} & \textbf{$F$} \\
\midrule
迴歸 & $\SSR$ & $k$ & $\MSR = \SSR/k$ & $F = \MSR/\MSE$ \\
殘差 & $\SSE$ & $n-k-1$ & $\MSE = \SSE/(n-k-1)$ & \\
\midrule
總和 & $\SST$ & $n-1$ & & \\
\bottomrule
\end{tabular}
\end{table}

\begin{theorem}[整體 $F$ 檢定]\label{thm:f-test}
檢定 $H_0: \beta_1 = \beta_2 = \cdots = \beta_k = 0$:
\[
\boxed{F = \frac{\MSR}{\MSE} = \frac{R^2/k}{(1-R^2)/(n-k-1)} \sim F_{k, n-k-1}} \quad \text{(在 $H_0$ 下)}
\]
\end{theorem}

\begin{proof}
\textbf{$F$ 統計量的公式}:

$F = \frac{\SSR/k}{\SSE/(n-k-1)}$。由 $R^2 = \SSR/\SST$ 和 $1 - R^2 = \SSE/\SST$:
\[
F = \frac{R^2 \cdot \SST/k}{(1-R^2) \cdot \SST/(n-k-1)} = \frac{R^2/k}{(1-R^2)/(n-k-1)}
\]

\textbf{$F$ 分配的推導}:

在 $H_0$ 下,可以證明 $\SSR/\sigma^2 \sim \chi^2_k$ 且與 $\SSE$ 獨立。由定理~\ref{thm:sse-dist},$\SSE/\sigma^2 \sim \chi^2_{n-k-1}$。

$F$ 分配的定義:若 $U \sim \chi^2_m$,$V \sim \chi^2_n$ 獨立,則 $(U/m)/(V/n) \sim F_{m,n}$。

故:
\[
F = \frac{(\SSR/\sigma^2)/k}{(\SSE/\sigma^2)/(n-k-1)} = \frac{\SSR/k}{\SSE/(n-k-1)} \sim F_{k, n-k-1}
\]
\end{proof}

\subsection{個別係數 $t$ 檢定}

\begin{theorem}[個別係數檢定]\label{thm:t-test}
對於 $H_0: \beta_j = 0$:
\[
\boxed{t = \frac{\hat{\beta}_j}{\SE(\hat{\beta}_j)} \sim t_{n-k-1}}
\]
其中 $\SE(\hat{\beta}_j) = s \sqrt{[(\bX^\top\bX)^{-1}]_{jj}}$,$s = \sqrt{\MSE}$。
\end{theorem}

\begin{proof}
由定理~\ref{thm:beta-dist},$\hat{\beta}_j \sim N(\beta_j, \sigma^2[(\bX^\top\bX)^{-1}]_{jj})$。

標準化:$Z = \frac{\hat{\beta}_j - \beta_j}{\sigma\sqrt{[(\bX^\top\bX)^{-1}]_{jj}}} \sim N(0, 1)$

由定理~\ref{thm:sse-dist},$\SSE/\sigma^2 \sim \chi^2_{n-k-1}$,且與 $\hat{\beta}_j$ 獨立。

$t$ 分配的定義:若 $Z \sim N(0,1)$,$V \sim \chi^2_\nu$ 獨立,則 $Z/\sqrt{V/\nu} \sim t_\nu$。

令 $V = \SSE/\sigma^2$,$\nu = n - k - 1$:
\begin{align*}
t &= \frac{Z}{\sqrt{V/\nu}} = \frac{\frac{\hat{\beta}_j - \beta_j}{\sigma\sqrt{[(\bX^\top\bX)^{-1}]_{jj}}}}{\sqrt{\frac{\SSE/\sigma^2}{n-k-1}}}\\
&= \frac{\hat{\beta}_j - \beta_j}{\sqrt{\frac{\SSE}{n-k-1}} \cdot \sqrt{[(\bX^\top\bX)^{-1}]_{jj}}}\\
&= \frac{\hat{\beta}_j - \beta_j}{s \sqrt{[(\bX^\top\bX)^{-1}]_{jj}}} \sim t_{n-k-1}
\end{align*}

在 $H_0: \beta_j = 0$ 下,$t = \hat{\beta}_j / \SE(\hat{\beta}_j)$。
\end{proof}

\begin{theorem}[簡單迴歸中 $t^2 = F$]
在簡單線性迴歸($k = 1$)中,檢定 $H_0: \beta_1 = 0$ 時:
\[
t^2 = F
\]
\end{theorem}

\begin{proof}
$t = \hat{\beta}_1/(s/\sqrt{S_{XX}})$,故 $t^2 = \hat{\beta}_1^2 S_{XX}/s^2 = \hat{\beta}_1^2 S_{XX}/\MSE = \SSR/\MSE = F$
\end{proof}

%=============================================================================
\section{進階主題}
%=============================================================================

\subsection{含交互作用項的模型}

\begin{definition}[交互作用]
模型 $Y = \beta_0 + \beta_1 X_1 + \beta_2 X_2 + \beta_3 X_1 X_2 + \varepsilon$ 中,$X_1$ 對 $Y$ 的\textbf{邊際效應}為:
\[
\frac{\partial Y}{\partial X_1} = \beta_1 + \beta_3 X_2
\]
此效應取決於 $X_2$ 的值——這就是「交互作用」的含義。
\end{definition}

\begin{example}[廣告與定價的交互作用]
模型:$Y = 20 + 2.5 X_1 - 1.2 X_2 + 0.04 X_1 X_2$($Y$: 銷售,$X_1$: 廣告,$X_2$: 定價)

廣告的邊際效應 $= 2.5 + 0.04 X_2$。當定價 $X_2$ 越高,廣告效果越強。
\end{example}

\subsection{無截距模型}

\begin{definition}[無截距迴歸——多元形式]
考慮模型 $Y_i = \beta_1 X_{i1} + \cdots + \beta_k X_{ik} + \varepsilon_i$(無截距項),矩陣形式為 $\bY = \bX\bb + \be$,其中 $\bX$ 為 $n \times k$ 矩陣(不含全 1 的行)。

OLS 估計量仍為 $\hat{\bb} = (\bX^\top\bX)^{-1}\bX^\top\bY$。
\end{definition}

\begin{property}[無截距模型的特性]
\begin{enumerate}[label=(\roman*)]
  \item[]
\item $\sum_{i=1}^{n} e_i$ 不一定為零(因為 $\bX$ 不包含截距項,正規方程式不含 $\sum e_i = 0$)
\item $\bX^\top\mathbf{e} = \mathbf{0}$ 仍成立(這是正規方程式的直接結果)
\item Centered-$R^2 = 1 - \SSE/\SST$ 可能為負,因為 $\SST + \SSE$ 不一定等於 $\SSR$
\item 使用 Uncentered-$R^2 = 1 - \SSE/\sum Y_i^2$ 較為適當
\end{enumerate}
\end{property}

\begin{example}[CAPM 模型]
資本資產定價模型(Capital Asset Pricing Model, CAPM)的實證形式為:
\[
R_{it} - R_f = \beta_i (R_{mt} - R_f) + \varepsilon_{it}
\]
其中 $R_{it} - R_f$ 為資產 $i$ 的超額報酬,$R_{mt} - R_f$ 為市場超額報酬。這是典型的無截距迴歸。

若加入截距項(Jensen's alpha):$R_{it} - R_f = \alpha_i + \beta_i (R_{mt} - R_f) + \varepsilon_{it}$,則 $\alpha_i \neq 0$ 表示該資產的表現優於($\alpha > 0$)或劣於($\alpha < 0$)市場基準。
\end{example}

%=============================================================================
\section{計算範例}
%=============================================================================

\begin{example}
某分析師研究家庭垃圾產量的決定因素。資料:$n = 440$,$k = 3$(房屋大小、兒童數、成人數),$R^2 = 0.170$,$F = 29.80$。

\begin{enumerate}[label=(\alph*)]
\item 計算調整後 $R^2$
\item 計算殘差自由度
\item 進行整體 $F$ 檢定($\alpha = 0.05$)
\end{enumerate}
\end{example}

\begin{solution}
\textbf{(a) 調整後 $R^2$}:
\[
\bar{R}^2 = 1 - \frac{n-1}{n-k-1}(1-R^2) = 1 - \frac{439}{436}(0.830) = 1 - 1.0069 \times 0.830 = 1 - 0.836 = \boxed{0.164}
\]

\textbf{(b) 殘差自由度}:$df_{\text{殘差}} = n - k - 1 = 440 - 3 - 1 = \boxed{436}$

\textbf{(c) $F$ 檢定}:

驗算:$F = \frac{R^2/k}{(1-R^2)/(n-k-1)} = \frac{0.170/3}{0.830/436} = \frac{0.0567}{0.00190} = 29.8$ $\checkmark$

臨界值:$F_{0.05, 3, 436} \approx 2.63$

決策:$29.80 > 2.63$,\textbf{拒絕 $H_0$}。模型整體顯著。
\end{solution}

\begin{example}
$n = 100$,$k = 4$,$R^2 = 0.36$。
\end{example}

\begin{solution}
$F = \frac{0.36/4}{0.64/95} = \frac{0.09}{0.00674} = 13.35$

$\bar{R}^2 = 1 - \frac{99}{95}(0.64) = 1 - 0.667 = 0.333$
\end{solution}

\begin{example}
$Y$ 對 $X$ 迴歸得 $\hat{Y} = 2 + 0.5X$,$R^2 = 0.64$。求 $X$ 對 $Y$ 迴歸的斜率。
\end{example}

\begin{solution}
$r^2 = R^2 = 0.64$,$\hat{\beta}_1 = 0.5 > 0$,故 $r = 0.8$。

由 $\hat{\beta}_1 \cdot \hat{\gamma}_1 = r^2$:$\hat{\gamma}_1 = 0.64/0.5 = 1.28$
\end{solution}

%=============================================================================
\section{習題}
%=============================================================================

\begin{exercise}
已知 $\SST = 100$,$\SSE = 36$,$n = 12$,$k = 1$。求 $R^2$、$\bar{R}^2$、$F$。
\end{exercise}

\begin{solution}
$R^2 = 1 - 36/100 = 0.64$

$\bar{R}^2 = 1 - \frac{11}{10}(0.36) = 1 - 0.396 = 0.604$

$F = \frac{0.64/1}{0.36/10} = \frac{0.64}{0.036} = 17.78$
\end{solution}

\begin{exercise}
迴歸輸出:$n = 200$,$k = 3$,$R^2 = 0.72$,截距 $\hat{\beta}_0 = 7.19$,$\SE(\hat{\beta}_0) = 1.09$。

(a) 計算 $\bar{R}^2$

(b) 計算截距的 $t$ 值

(c) 在 $\alpha = 0.05$ 下檢定整體模型顯著性
\end{exercise}

\begin{solution}
\textbf{(a)} $\bar{R}^2 = 1 - \frac{199}{196}(0.28) = 1 - 0.284 = 0.716$

\textbf{(b)} $t = 7.19/1.09 = 6.59$

\textbf{(c)} $F = \frac{0.72/3}{0.28/196} = \frac{0.24}{0.00143} = 168$

$F_{0.05, 3, 196} \approx 2.65$。$168 > 2.65$,拒絕 $H_0$,模型整體顯著。
\end{solution}

\end{document}
