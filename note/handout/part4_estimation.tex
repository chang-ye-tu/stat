\documentclass[12pt,a4paper]{article}
\usepackage[left=1cm,right=1cm,bottom=15mm,top=20mm]{geometry}
\usepackage[AutoFakeBold,AutoFakeSlant]{xeCJK}
\setCJKmainfont[AutoFakeSlant=.1,AutoFakeBold=2]{Noto Serif CJK TC}
\usepackage{amsmath,amsthm,amssymb,amsfonts}
\usepackage{graphicx,xcolor,float}
\usepackage{booktabs,tabularx,multirow,array}
\usepackage{enumitem}
\usepackage{parskip}
\setlist{itemsep=0pt,parsep=0pt}
\usepackage{hyperref}
\hypersetup{
    colorlinks=true,
    linkcolor=blue!70!black,
    urlcolor=blue!80!black
}

% 定理環境
\theoremstyle{definition}
\newtheorem{definition}{定義}[section]
\newtheorem{example}{例題}[section]
\newtheorem{exercise}{習題}[section]
\newtheorem{theorem}{定理}[section]
\newtheorem{lemma}{引理}[section]
\newtheorem{corollary}{推論}[section]
\newtheorem{proposition}{命題}[section]
\newtheorem{property}{性質}[section]
\newtheorem*{remark}{註}
\newtheorem*{solution}{解答}
\newtheorem*{note}{說明}
\newtheorem*{prf}{證明}

% 常用指令
\newcommand{\ds}{\displaystyle}
\newcommand{\ie}{\;\Longrightarrow\;}
\newcommand{\ifff}{\;\Longleftrightarrow\;}
\newcommand\expc{\mathsf{E}}
\DeclareMathOperator\var{var}
\DeclareMathOperator\cov{cov}
\DeclareMathOperator\corr{corr}
\newcommand{\SE}{\mathrm{SE}}
\newcommand{\MSE}{\mathrm{MSE}}
\newcommand\prb{\mathsf{P}}
\newcommand{\Real}{\mathbb{R}}
\newcommand{\Nat}{\mathbb{N}}
\newcommand{\diff}[2]{\frac{\mathrm{d} #1}{\mathrm{d} #2}}
\newcommand{\pdiff}[2]{\frac{\partial #1}{\partial #2}}
\newcommand{\simark}{\stackrel{\text{i.i.d.}}{\sim}}
\newcommand{\pmark}{\stackrel{p}{\longrightarrow}}

% 頁面設定
\renewcommand{\figurename}{圖}
\renewcommand{\tablename}{表}
\renewcommand{\proofname}{\textbf{證明}}

\usepackage{fancyhdr}
\pagestyle{fancy}
\fancyhf{}
\fancyhead[L]{統計學講義}
\fancyhead[R]{第四部分:點估計與區間估計}
\fancyfoot[C]{\thepage}
\renewcommand{\headrulewidth}{0.4pt}
\renewcommand{\footrulewidth}{0.4pt}

\title{\vspace{-2cm}\textbf{統計學講義}\\[3mm] \Large 第四部分:點估計與區間估計}
\author{}
\date{\vspace{-2cm}}

\begin{document}
\maketitle
\thispagestyle{fancy}

\begin{center}
\fbox{\parbox{0.9\textwidth}{\centering
\textbf{參考書籍}\\[2mm]
Jeffrey S. Rosenthal, \textit{Probability and Statistics: The Science of Uncertainty}, 2nd Edition\\
Chapter 7: Estimation\\[2mm]
\url{https://utstat.utoronto.ca/mikevans/jeffrosenthal/}
}}
\end{center}

%\tableofcontents

%=============================================================================
\section{點估計的基本概念}
%=============================================================================

\subsection{估計量與估計值}

\begin{definition}[估計量與估計值]
\begin{itemize}
  \item[]
  \item \textbf{估計量}(estimator):用樣本 $X_1, \ldots, X_n$ 計算的統計量,用來估計未知母體參數 $\theta$,記為 $\hat{\theta}$ 或 $\hat{\theta}(X_1, \ldots, X_n)$。
    \item \textbf{估計值}(estimate):將特定樣本觀測值代入估計量後得到的數值。
\end{itemize}
\end{definition}

\begin{example}
設 $X_1, \ldots, X_n$ 為來自母體(平均數 $\mu$,變異數 $\sigma^2$)的隨機樣本。

\begin{itemize}
    \item $\hat{\mu} = \bar{X} = \dfrac{1}{n}\sum_{i=1}^{n} X_i$ 是 $\mu$ 的估計量。
    \item $\hat{\sigma}^2 = S^2 = \dfrac{1}{n-1}\sum_{i=1}^{n}(X_i - \bar{X})^2$ 是 $\sigma^2$ 的估計量。
    \item 若樣本觀測值為 $3, 5, 7, 9, 11$,則 $\hat{\mu} = 7$,$\hat{\sigma}^2 = 10$ 為估計值。
\end{itemize}
\end{example}

\subsection{良好估計量的性質}

\begin{definition}[不偏性]
若估計量 $\hat{\theta}$ 滿足
\[
\expc[\hat{\theta}] = \theta
\]
則稱 $\hat{\theta}$ 為 $\theta$ 的\textbf{不偏估計量}(unbiased estimator)。

若 $\expc[\hat{\theta}] \neq \theta$,則稱 $\hat{\theta}$ 為\textbf{有偏估計量}(biased estimator),\textbf{偏誤}(bias)定義為:
\[
\text{Bias}(\hat{\theta}) = \expc[\hat{\theta}] - \theta
\]
\end{definition}

\begin{theorem}[常見的不偏估計量]
設 $X_1, \ldots, X_n \simark F$,其中 $\expc[X_i] = \mu$,$\var(X_i) = \sigma^2$。則:
\begin{enumerate}[label=(\roman*)]
    \item $\bar{X}$ 是 $\mu$ 的不偏估計量:$\expc[\bar{X}] = \mu$
    \item $S^2$ 是 $\sigma^2$ 的不偏估計量:$\expc[S^2] = \sigma^2$
\end{enumerate}
\end{theorem}

\begin{remark}
注意:$\hat{\sigma}^2 = \dfrac{1}{n}\sum_{i=1}^{n}(X_i - \bar{X})^2$ 是 $\sigma^2$ 的\textbf{有偏}估計量:
\[
\expc\left[\frac{1}{n}\sum_{i=1}^{n}(X_i - \bar{X})^2\right] = \frac{n-1}{n}\sigma^2 \neq \sigma^2
\]
偏誤為 $-\sigma^2/n$,但當 $n \to \infty$ 時偏誤趨近 0(漸近不偏)。
\end{remark}

\begin{definition}[有效性]
在所有不偏估計量中,\textbf{變異數最小}的估計量稱為\textbf{最小變異數不偏估計量}(Minimum Variance Unbiased Estimator, MVUE)或\textbf{最佳不偏估計量}。

若 $\hat{\theta}_1$ 與 $\hat{\theta}_2$ 都是 $\theta$ 的不偏估計量,且 $\var(\hat{\theta}_1) < \var(\hat{\theta}_2)$,則稱 $\hat{\theta}_1$ 比 $\hat{\theta}_2$ \textbf{更有效}(more efficient)。

\textbf{相對效率}(relative efficiency):
\[
\text{eff}(\hat{\theta}_1, \hat{\theta}_2) = \frac{\var(\hat{\theta}_2)}{\var(\hat{\theta}_1)}
\]
\end{definition}

\begin{definition}[均方誤差]
估計量 $\hat{\theta}$ 的\textbf{均方誤差}(Mean Squared Error, MSE)定義為:
\[
\MSE(\hat{\theta}) = \expc\left[(\hat{\theta} - \theta)^2\right]
\]
\end{definition}

\begin{theorem}[MSE 分解公式]
\[
\MSE(\hat{\theta}) = \var(\hat{\theta}) + \left[\text{Bias}(\hat{\theta})\right]^2
\]
\end{theorem}

\begin{prf}
令 $b = \expc[\hat{\theta}] - \theta = \text{Bias}(\hat{\theta})$。
\begin{align*}
\MSE(\hat{\theta}) &= \expc[(\hat{\theta} - \theta)^2]\\
&= \expc[(\hat{\theta} - \expc[\hat{\theta}] + \expc[\hat{\theta}] - \theta)^2]\\
&= \expc[(\hat{\theta} - \expc[\hat{\theta}])^2] + 2\expc[(\hat{\theta} - \expc[\hat{\theta}])](\expc[\hat{\theta}] - \theta) + (\expc[\hat{\theta}] - \theta)^2\\
&= \var(\hat{\theta}) + 2 \cdot 0 \cdot b + b^2\\
&= \var(\hat{\theta}) + [\text{Bias}(\hat{\theta})]^2
\end{align*}
\end{prf}

\begin{remark}
對於不偏估計量,$\text{Bias} = 0$,故 $\MSE(\hat{\theta}) = \var(\hat{\theta})$。
\end{remark}

\begin{definition}[一致性]
若估計量序列 $\{\hat{\theta}_n\}$ 滿足:對所有 $\epsilon > 0$,
\[
\lim_{n \to \infty} \prb(|\hat{\theta}_n - \theta| > \epsilon) = 0
\]
則稱 $\hat{\theta}_n$ 為 $\theta$ 的\textbf{一致估計量}(consistent estimator),記為 $\hat{\theta}_n \pmark \theta$。
\end{definition}

\begin{theorem}[一致性的充分條件]
若 $\ds\lim_{n \to \infty} \expc[\hat{\theta}_n] = \theta$ 且 $\ds\lim_{n \to \infty} \var(\hat{\theta}_n) = 0$,則 $\hat{\theta}_n$ 是 $\theta$ 的一致估計量。
\end{theorem}

\begin{prf}
由 Chebyshev 不等式:
\[
\prb(|\hat{\theta}_n - \expc[\hat{\theta}_n]| > \epsilon) \leqslant \frac{\var(\hat{\theta}_n)}{\epsilon^2}
\]

當 $n \to \infty$,$\var(\hat{\theta}_n) \to 0$,故右邊趨近 0。

又 $\expc[\hat{\theta}_n] \to \theta$,故 $\hat{\theta}_n \pmark \theta$。
\end{prf}

\begin{example}
證明 $\bar{X}$ 是 $\mu$ 的一致估計量。
\end{example}

\begin{solution}
$\expc[\bar{X}] = \mu$(不偏)

$\var(\bar{X}) = \dfrac{\sigma^2}{n} \to 0$ 當 $n \to \infty$

由定理 1.4,$\bar{X} \pmark \mu$,故 $\bar{X}$ 是一致估計量。
\end{solution}

%=============================================================================
\section{最大概似估計法}
%=============================================================================

\subsection{概似函數}

\begin{definition}[概似函數]
設 $X_1, \ldots, X_n$ 為來自機率分配 $f(x; \theta)$ 的隨機樣本。\textbf{概似函數}(likelihood function)定義為:
\[
L(\theta) = L(\theta; x_1, \ldots, x_n) = \prod_{i=1}^{n} f(x_i; \theta)
\]

\textbf{對數概似函數}(log-likelihood function):
\[
\ell(\theta) = \ln L(\theta) = \sum_{i=1}^{n} \ln f(x_i; \theta)
\]
\end{definition}

\begin{remark}
概似函數是將樣本觀測值視為固定,將 $\theta$ 視為變數的函數。它衡量在不同參數值下,觀測到這組樣本的「可能性」。
\end{remark}

\subsection{最大概似估計量}

\begin{definition}[最大概似估計量]
使概似函數 $L(\theta)$(或等價地,對數概似函數 $\ell(\theta)$)達到最大的 $\theta$ 值,稱為\textbf{最大概似估計量}(Maximum Likelihood Estimator, MLE),記為 $\hat{\theta}_{\text{MLE}}$:
\[
\hat{\theta}_{\text{MLE}} = \arg\max_{\theta} L(\theta) = \arg\max_{\theta} \ell(\theta)
\]
\end{definition}

\begin{property}[求 MLE 的步驟]
\begin{enumerate}
  \item[]
    \item 寫出概似函數 $L(\theta) = \prod_{i=1}^{n} f(x_i; \theta)$
    \item 取對數得 $\ell(\theta) = \sum_{i=1}^{n} \ln f(x_i; \theta)$
    \item 對 $\theta$ 微分,令 $\dfrac{d\ell}{d\theta} = 0$(或 $\dfrac{\partial \ell}{\partial \theta_j} = 0$ 對所有參數)
    \item 解方程得 $\hat{\theta}_{\text{MLE}}$
    \item 確認為最大值(檢查二階導數或邊界條件)
\end{enumerate}
\end{property}

\begin{example}[常態分配的 MLE]
設 $X_1, \ldots, X_n \simark N(\mu, \sigma^2)$,求 $\mu$ 和 $\sigma^2$ 的 MLE。
\end{example}

\begin{solution}
\textbf{步驟 1}:概似函數
\[
L(\mu, \sigma^2) = \prod_{i=1}^{n} \frac{1}{\sqrt{2\pi\sigma^2}} \exp\left(-\frac{(x_i - \mu)^2}{2\sigma^2}\right)
\]

\textbf{步驟 2}:對數概似函數
\begin{align*}
\ell(\mu, \sigma^2) &= -\frac{n}{2}\ln(2\pi) - \frac{n}{2}\ln(\sigma^2) - \frac{1}{2\sigma^2}\sum_{i=1}^{n}(x_i - \mu)^2
\end{align*}

\textbf{步驟 3}:對 $\mu$ 微分
\[
\pdiff{\ell}{\mu} = \frac{1}{\sigma^2}\sum_{i=1}^{n}(x_i - \mu) = 0
\]
\[
\sum_{i=1}^{n}(x_i - \mu) = 0 \ie \sum_{i=1}^{n} x_i = n\mu \ie \hat{\mu}_{\text{MLE}} = \bar{x}
\]

\textbf{步驟 4}:對 $\sigma^2$ 微分
\[
\pdiff{\ell}{\sigma^2} = -\frac{n}{2\sigma^2} + \frac{1}{2(\sigma^2)^2}\sum_{i=1}^{n}(x_i - \mu)^2 = 0
\]

代入 $\hat{\mu} = \bar{x}$:
\[
\frac{n}{2\sigma^2} = \frac{\sum_{i=1}^{n}(x_i - \bar{x})^2}{2(\sigma^2)^2}
\]
\[
\hat{\sigma}^2_{\text{MLE}} = \frac{1}{n}\sum_{i=1}^{n}(x_i - \bar{x})^2
\]

\textbf{結論}:
\[
\hat{\mu}_{\text{MLE}} = \bar{X}, \quad \hat{\sigma}^2_{\text{MLE}} = \frac{1}{n}\sum_{i=1}^{n}(X_i - \bar{X})^2
\]

注意:$\hat{\sigma}^2_{\text{MLE}}$ 是有偏的(分母為 $n$ 而非 $n-1$),但為一致估計量。
\end{solution}

\begin{example}[指數分配的 MLE]
設 $X_1, \ldots, X_n \simark \text{Exp}(\lambda)$,PDF 為 $f(x; \lambda) = \lambda e^{-\lambda x}$,$x > 0$。求 $\lambda$ 的 MLE。
\end{example}

\begin{solution}
\textbf{概似函數}:
\[
L(\lambda) = \prod_{i=1}^{n} \lambda e^{-\lambda x_i} = \lambda^n e^{-\lambda \sum_{i=1}^{n} x_i}
\]

\textbf{對數概似函數}:
\[
\ell(\lambda) = n\ln\lambda - \lambda\sum_{i=1}^{n} x_i
\]

\textbf{微分}:
\[
\diff{\ell}{\lambda} = \frac{n}{\lambda} - \sum_{i=1}^{n} x_i = 0
\]
\[
\hat{\lambda}_{\text{MLE}} = \frac{n}{\sum_{i=1}^{n} x_i} = \frac{1}{\bar{x}}
\]

\textbf{驗證最大值}:$\dfrac{d^2\ell}{d\lambda^2} = -\dfrac{n}{\lambda^2} < 0$,確認為最大值。
\end{solution}

\begin{example}[Poisson 分配的 MLE]
設 $X_1, \ldots, X_n \simark \text{Poisson}(\lambda)$。求 $\lambda$ 的 MLE。
\end{example}

\begin{solution}
\textbf{概似函數}:
\[
L(\lambda) = \prod_{i=1}^{n} \frac{e^{-\lambda}\lambda^{x_i}}{x_i!} = \frac{e^{-n\lambda}\lambda^{\sum x_i}}{\prod_{i=1}^{n} x_i!}
\]

\textbf{對數概似函數}:
\[
\ell(\lambda) = -n\lambda + \left(\sum_{i=1}^{n} x_i\right)\ln\lambda - \sum_{i=1}^{n}\ln(x_i!)
\]

\textbf{微分}:
\[
\diff{\ell}{\lambda} = -n + \frac{\sum_{i=1}^{n} x_i}{\lambda} = 0
\]
\[
\hat{\lambda}_{\text{MLE}} = \frac{\sum_{i=1}^{n} x_i}{n} = \bar{x}
\]
\end{solution}

\begin{example}[伯努利分配的 MLE]
設 $X_1, \ldots, X_n \simark \text{Bernoulli}(p)$。求 $p$ 的 MLE。
\end{example}

\begin{solution}
\textbf{概似函數}:
\[
L(p) = \prod_{i=1}^{n} p^{x_i}(1-p)^{1-x_i} = p^{\sum x_i}(1-p)^{n - \sum x_i}
\]

\textbf{對數概似函數}:
\[
\ell(p) = \left(\sum_{i=1}^{n} x_i\right)\ln p + \left(n - \sum_{i=1}^{n} x_i\right)\ln(1-p)
\]

\textbf{微分}:
\[
\diff{\ell}{p} = \frac{\sum x_i}{p} - \frac{n - \sum x_i}{1-p} = 0
\]
\[
(1-p)\sum x_i = p(n - \sum x_i)
\]
\[
\sum x_i = np \ie \hat{p}_{\text{MLE}} = \frac{\sum_{i=1}^{n} x_i}{n} = \bar{x}
\]

即樣本比例 $\hat{p} = \dfrac{\text{成功次數}}{n}$。
\end{solution}

\subsection{MLE 的性質}

\begin{theorem}[MLE 的不變性]
若 $\hat{\theta}_{\text{MLE}}$ 是 $\theta$ 的 MLE,且 $g$ 是一對一函數,則 $g(\hat{\theta}_{\text{MLE}})$ 是 $g(\theta)$ 的 MLE。
\end{theorem}

\begin{example}
若 $\hat{\sigma}^2_{\text{MLE}}$ 是 $\sigma^2$ 的 MLE,則 $\hat{\sigma}_{\text{MLE}} = \sqrt{\hat{\sigma}^2_{\text{MLE}}}$ 是 $\sigma$ 的 MLE。
\end{example}

\begin{theorem}[MLE 的漸近性質]
在適當的正則條件下,MLE 具有以下漸近性質:
\begin{enumerate}[label=(\roman*)]
    \item \textbf{一致性}:當樣本數 $n \to \infty$ 時,$\hat{\theta}_{\text{MLE}}$ 依機率收斂至 $\theta$
    \item \textbf{漸近常態性}:當 $n$ 夠大時,MLE 近似服從常態分配
    \item \textbf{漸近有效性}:MLE 在大樣本下達到最小可能的變異數
\end{enumerate}
\end{theorem}

\begin{note}
上述「依機率收斂」的嚴格定義需要機率論課程的知識。直觀上,它表示 $\hat{\theta}_{\text{MLE}}$ 會越來越接近真實的 $\theta$。
\end{note}

%=============================================================================
\section{區間估計的基本概念}
%=============================================================================

\subsection{信賴區間的定義}

\begin{definition}[信賴區間]
設 $\theta$ 為未知母體參數,$X_1, \ldots, X_n$ 為隨機樣本。若統計量 $L = L(X_1, \ldots, X_n)$ 與 $U = U(X_1, \ldots, X_n)$ 滿足:
\[
\prb(L \leqslant \theta \leqslant U) = 1 - \alpha
\]
則稱隨機區間 $[L, U]$ 為 $\theta$ 的 \textbf{$(1-\alpha) \times 100\%$ 信賴區間}(confidence interval),$1 - \alpha$ 稱為\textbf{信賴水準}(confidence level)。
\end{definition}

\begin{remark}
\textbf{信賴區間的正確解釋}:
\begin{itemize}
    \item 信賴區間是\textbf{隨機的}——由隨機樣本計算得到。
    \item 母體參數 $\theta$ 是\textbf{固定的}(雖然未知)。
    \item 「95\% 信賴區間」的意義:若重複抽樣很多次,每次計算一個信賴區間,則約有 95\% 的區間會包含真正的 $\theta$。
    \item \textbf{錯誤解釋}:「$\theta$ 有 95\% 的機率落在此區間內」——這是錯的,因為 $\theta$ 是固定值,不是隨機變數。
\end{itemize}
\end{remark}

\subsection{樞紐量}

\begin{definition}[樞紐量]
\textbf{樞紐量}(pivotal quantity)是樣本與參數的函數,其分配不依賴任何未知參數。
\end{definition}

\begin{example}
設 $X_1, \ldots, X_n \simark N(\mu, \sigma^2)$,$\sigma^2$ 已知。則
\[
Z = \frac{\bar{X} - \mu}{\sigma/\sqrt{n}} \sim N(0, 1)
\]
是樞紐量,因為其分配(標準常態)不依賴未知參數 $\mu$。
\end{example}

\begin{property}[用樞紐量建構信賴區間]
\begin{enumerate}
  \item[]
    \item 找到包含 $\theta$ 的樞紐量 $Q$,其分配已知
    \item 找 $a$ 和 $b$ 使得 $\prb(a \leqslant Q \leqslant b) = 1 - \alpha$
    \item 將不等式 $a \leqslant Q \leqslant b$ 改寫為 $L \leqslant \theta \leqslant U$ 的形式
    \item $[L, U]$ 即為 $(1-\alpha)$ 信賴區間
\end{enumerate}
\end{property}

%=============================================================================
\section{母體平均數的信賴區間}
%=============================================================================

\subsection{$\sigma$ 已知的情況}

\begin{theorem}[$\sigma$ 已知時 $\mu$ 的信賴區間]\label{thm:CI-mu-sigma-known}
設 $X_1, \ldots, X_n$ 為來自 $N(\mu, \sigma^2)$ 的隨機樣本,$\sigma^2$ 已知。則 $\mu$ 的 $(1-\alpha) \times 100\%$ 信賴區間為:
\[
\bar{X} \pm z_{\alpha/2} \cdot \frac{\sigma}{\sqrt{n}}
\]
即 $\left(\bar{X} - z_{\alpha/2} \cdot \dfrac{\sigma}{\sqrt{n}}, \; \bar{X} + z_{\alpha/2} \cdot \dfrac{\sigma}{\sqrt{n}}\right)$
\end{theorem}

\begin{prf}
樞紐量:$Z = \dfrac{\bar{X} - \mu}{\sigma/\sqrt{n}} \sim N(0, 1)$

由標準常態分配的對稱性:
\[
P\left(-z_{\alpha/2} \leqslant \frac{\bar{X} - \mu}{\sigma/\sqrt{n}} \leqslant z_{\alpha/2}\right) = 1 - \alpha
\]

將中間的不等式改寫:
\[
-z_{\alpha/2} \leqslant \frac{\bar{X} - \mu}{\sigma/\sqrt{n}} \leqslant z_{\alpha/2}
\]
\[
-z_{\alpha/2} \cdot \frac{\sigma}{\sqrt{n}} \leqslant \bar{X} - \mu \leqslant z_{\alpha/2} \cdot \frac{\sigma}{\sqrt{n}}
\]
\[
\bar{X} - z_{\alpha/2} \cdot \frac{\sigma}{\sqrt{n}} \leqslant \mu \leqslant \bar{X} + z_{\alpha/2} \cdot \frac{\sigma}{\sqrt{n}}
\]
\end{prf}

\begin{definition}[誤差界限]
信賴區間的\textbf{誤差界限}(margin of error)或\textbf{半寬}:
\[
E = z_{\alpha/2} \cdot \frac{\sigma}{\sqrt{n}}
\]
信賴區間可寫為 $\bar{X} \pm E$。
\end{definition}

\begin{example}
某工廠生產的電池壽命服從常態分配,已知標準差 $\sigma = 10$ 小時。隨機抽取 25 顆電池,測得平均壽命 $\bar{x} = 48$ 小時。求 $\mu$ 的 95\% 信賴區間。
\end{example}

\begin{solution}
$n = 25$,$\bar{x} = 48$,$\sigma = 10$,$\alpha = 0.05$,$z_{0.025} = 1.96$

誤差界限:$E = 1.96 \times \dfrac{10}{\sqrt{25}} = 1.96 \times 2 = 3.92$

95\% 信賴區間:$48 \pm 3.92 = (44.08, 51.92)$

解釋:我們有 95\% 的信心認為母體平均壽命介於 44.08 到 51.92 小時之間。
\end{solution}

\subsection{$\sigma$ 未知的情況}

\begin{theorem}[$\sigma$ 未知時 $\mu$ 的信賴區間]\label{thm:CI-mu-sigma-unknown}
設 $X_1, \ldots, X_n$ 為來自 $N(\mu, \sigma^2)$ 的隨機樣本,$\sigma^2$ 未知。則 $\mu$ 的 $(1-\alpha) \times 100\%$ 信賴區間為:
\[
\bar{X} \pm t_{\alpha/2, n-1} \cdot \frac{S}{\sqrt{n}}
\]
其中 $t_{\alpha/2, n-1}$ 是自由度為 $n-1$ 的 t 分配的上 $\alpha/2$ 分位數。
\end{theorem}

\begin{prf}
樞紐量:$T = \dfrac{\bar{X} - \mu}{S/\sqrt{n}} \sim t_{n-1}$

由 t 分配的對稱性:
\[
P\left(-t_{\alpha/2, n-1} \leqslant \frac{\bar{X} - \mu}{S/\sqrt{n}} \leqslant t_{\alpha/2, n-1}\right) = 1 - \alpha
\]

改寫不等式:
\[
\bar{X} - t_{\alpha/2, n-1} \cdot \frac{S}{\sqrt{n}} \leqslant \mu \leqslant \bar{X} + t_{\alpha/2, n-1} \cdot \frac{S}{\sqrt{n}}
\]
\end{prf}

\begin{example}
隨機抽取 16 名學生,測得其統計學成績平均為 72 分,樣本標準差為 8 分。假設成績服從常態分配,求母體平均成績的 95\% 信賴區間。
\end{example}

\begin{solution}
$n = 16$,$\bar{x} = 72$,$s = 8$,$\alpha = 0.05$

自由度 $df = 16 - 1 = 15$,查表 $t_{0.025, 15} = 2.131$

95\% 信賴區間:
\[
72 \pm 2.131 \times \frac{8}{\sqrt{16}} = 72 \pm 2.131 \times 2 = 72 \pm 4.262
\]

即 $(67.74, 76.26)$
\end{solution}

\begin{table}[H]
\centering
\begin{tabular}{c|ccc}
\toprule
\textbf{信賴水準} & 90\% & 95\% & 99\% \\
\midrule
$\alpha$ & 0.10 & 0.05 & 0.01 \\
$z_{\alpha/2}$ & 1.645 & 1.96 & 2.576 \\
\bottomrule
\end{tabular}
\caption{常用的 $z_{\alpha/2}$ 值}
\end{table}

\subsection{大樣本情況}

\begin{theorem}[大樣本時 $\mu$ 的近似信賴區間]
當樣本量 $n$ 足夠大時(通常 $n \geqslant 30$),無論母體分配為何,$\mu$ 的 $(1-\alpha)$ 近似信賴區間為:
\[
\bar{X} \pm z_{\alpha/2} \cdot \frac{S}{\sqrt{n}}
\]
\end{theorem}

\begin{prf}
由中央極限定理,當 $n$ 大時 $\dfrac{\bar{X} - \mu}{S/\sqrt{n}} \stackrel{\text{approx}}{\sim} N(0, 1)$。
\end{prf}

%=============================================================================
\section{母體變異數的信賴區間}
%=============================================================================

\begin{theorem}[$\sigma^2$ 的信賴區間]
設 $X_1, \ldots, X_n \simark N(\mu, \sigma^2)$。則 $\sigma^2$ 的 $(1-\alpha) \times 100\%$ 信賴區間為:
\[
\left(\frac{(n-1)S^2}{\chi^2_{\alpha/2, n-1}}, \; \frac{(n-1)S^2}{\chi^2_{1-\alpha/2, n-1}}\right)
\]
其中 $\chi^2_{\alpha/2, n-1}$ 和 $\chi^2_{1-\alpha/2, n-1}$ 分別是 $\chi^2_{n-1}$ 分配的上 $\alpha/2$ 和上 $1-\alpha/2$ 分位數。
\end{theorem}

\begin{prf}
樞紐量:$\chi^2 = \dfrac{(n-1)S^2}{\sigma^2} \sim \chi^2_{n-1}$

\[
P\left(\chi^2_{1-\alpha/2, n-1} \leqslant \frac{(n-1)S^2}{\sigma^2} \leqslant \chi^2_{\alpha/2, n-1}\right) = 1 - \alpha
\]

注意 $\chi^2$ 分配不對稱,故 $\chi^2_{1-\alpha/2} < \chi^2_{\alpha/2}$。

改寫不等式(取倒數時不等號方向改變):
\[
\frac{(n-1)S^2}{\chi^2_{\alpha/2, n-1}} \leqslant \sigma^2 \leqslant \frac{(n-1)S^2}{\chi^2_{1-\alpha/2, n-1}}
\]
\end{prf}

\begin{example}
設 $n = 20$,$s^2 = 25$。求 $\sigma^2$ 的 95\% 信賴區間。
\end{example}

\begin{solution}
$n = 20$,$df = 19$,$\alpha = 0.05$

查表:$\chi^2_{0.025, 19} = 32.85$,$\chi^2_{0.975, 19} = 8.91$

95\% 信賴區間:
\[
\left(\frac{19 \times 25}{32.85}, \frac{19 \times 25}{8.91}\right) = \left(\frac{475}{32.85}, \frac{475}{8.91}\right) = (14.46, 53.31)
\]

$\sigma$ 的 95\% 信賴區間:$(\sqrt{14.46}, \sqrt{53.31}) = (3.80, 7.30)$
\end{solution}

%=============================================================================
\section{母體比例的信賴區間}
%=============================================================================

\begin{theorem}[母體比例的信賴區間]
設 $X$ 為 $n$ 次獨立伯努利試驗中成功的次數,$X \sim B(n, p)$。樣本比例 $\hat{p} = X/n$。當 $n$ 夠大($n\hat{p} \geqslant 5$ 且 $n(1-\hat{p}) \geqslant 5$)時,$p$ 的 $(1-\alpha)$ 近似信賴區間為:
\[
\hat{p} \pm z_{\alpha/2} \sqrt{\frac{\hat{p}(1-\hat{p})}{n}}
\]
\end{theorem}

\begin{prf}
由中央極限定理,當 $n$ 大時:
\[
\frac{\hat{p} - p}{\sqrt{p(1-p)/n}} \stackrel{\text{approx}}{\sim} N(0, 1)
\]

用 $\hat{p}$ 估計 $p$:
\[
\frac{\hat{p} - p}{\sqrt{\hat{p}(1-\hat{p})/n}} \stackrel{\text{approx}}{\sim} N(0, 1)
\]

故:
\[
P\left(-z_{\alpha/2} \leqslant \frac{\hat{p} - p}{\sqrt{\hat{p}(1-\hat{p})/n}} \leqslant z_{\alpha/2}\right) \approx 1 - \alpha
\]

改寫得信賴區間。
\end{prf}

\begin{example}
某民調隨機抽取 400 位選民,其中 220 人支持某候選人。求支持率的 95\% 信賴區間。
\end{example}

\begin{solution}
$n = 400$,$x = 220$,$\hat{p} = 220/400 = 0.55$

檢驗條件:$n\hat{p} = 220 \geqslant 5$,$n(1-\hat{p}) = 180 \geqslant 5$ \checkmark

標準誤:$\sqrt{\dfrac{0.55 \times 0.45}{400}} = \sqrt{\dfrac{0.2475}{400}} = 0.0249$

95\% 信賴區間:$0.55 \pm 1.96 \times 0.0249 = 0.55 \pm 0.049$

即 $(0.501, 0.599)$ 或約 $(50.1\%, 59.9\%)$
\end{solution}

%=============================================================================
\section{兩母體的信賴區間}
%=============================================================================

\subsection{兩母體平均數差的信賴區間}

\begin{theorem}[獨立樣本,$\sigma_1^2$、$\sigma_2^2$ 已知]
設兩獨立樣本 $X_1, \ldots, X_{n_1} \simark N(\mu_1, \sigma_1^2)$,$Y_1, \ldots, Y_{n_2} \simark N(\mu_2, \sigma_2^2)$。則 $\mu_1 - \mu_2$ 的 $(1-\alpha)$ 信賴區間為:
\[
(\bar{X} - \bar{Y}) \pm z_{\alpha/2} \sqrt{\frac{\sigma_1^2}{n_1} + \frac{\sigma_2^2}{n_2}}
\]
\end{theorem}

\begin{theorem}[獨立樣本,$\sigma_1^2 = \sigma_2^2 = \sigma^2$ 未知(合併變異數)]
若假設 $\sigma_1^2 = \sigma_2^2 = \sigma^2$(未知),則使用\textbf{合併樣本變異數}:
\[
S_p^2 = \frac{(n_1-1)S_1^2 + (n_2-1)S_2^2}{n_1 + n_2 - 2}
\]

$\mu_1 - \mu_2$ 的 $(1-\alpha)$ 信賴區間為:
\[
(\bar{X} - \bar{Y}) \pm t_{\alpha/2, n_1+n_2-2} \cdot S_p \sqrt{\frac{1}{n_1} + \frac{1}{n_2}}
\]
\end{theorem}

\begin{prf}
樞紐量:
\[
T = \frac{(\bar{X} - \bar{Y}) - (\mu_1 - \mu_2)}{S_p\sqrt{\frac{1}{n_1} + \frac{1}{n_2}}} \sim t_{n_1+n_2-2}
\]

由此推導信賴區間。
\end{prf}

\begin{example}
兩組獨立樣本:第一組 $n_1 = 10$,$\bar{x}_1 = 85$,$s_1^2 = 16$;第二組 $n_2 = 12$,$\bar{x}_2 = 78$,$s_2^2 = 20$。假設兩母體變異數相等,求 $\mu_1 - \mu_2$ 的 95\% 信賴區間。
\end{example}

\begin{solution}
合併變異數:
\[
S_p^2 = \frac{(10-1)(16) + (12-1)(20)}{10 + 12 - 2} = \frac{144 + 220}{20} = \frac{364}{20} = 18.2
\]

$S_p = \sqrt{18.2} = 4.266$

自由度 $df = 10 + 12 - 2 = 20$,$t_{0.025, 20} = 2.086$

標準誤:$S_p\sqrt{\dfrac{1}{10} + \dfrac{1}{12}} = 4.266 \times \sqrt{0.1 + 0.0833} = 4.266 \times 0.428 = 1.826$

95\% 信賴區間:
\[
(85 - 78) \pm 2.086 \times 1.826 = 7 \pm 3.81
\]

即 $(3.19, 10.81)$

由於區間不包含 0,可認為兩母體平均數有顯著差異。
\end{solution}

\subsection{兩母體比例差的信賴區間}

\begin{theorem}[兩母體比例差的信賴區間]
設兩獨立樣本的樣本比例為 $\hat{p}_1$ 和 $\hat{p}_2$,樣本量分別為 $n_1$ 和 $n_2$。當樣本量夠大時,$p_1 - p_2$ 的 $(1-\alpha)$ 近似信賴區間為:
\[
(\hat{p}_1 - \hat{p}_2) \pm z_{\alpha/2} \sqrt{\frac{\hat{p}_1(1-\hat{p}_1)}{n_1} + \frac{\hat{p}_2(1-\hat{p}_2)}{n_2}}
\]
\end{theorem}

%=============================================================================
\section{樣本大小的決定}
%=============================================================================

\subsection{估計平均數所需的樣本量}

\begin{theorem}[估計 $\mu$ 所需的樣本量]
若要使誤差界限不超過 $E$,即 $z_{\alpha/2} \cdot \dfrac{\sigma}{\sqrt{n}} \leqslant E$,則所需樣本量為:
\[
n \geqslant \left(\frac{z_{\alpha/2} \cdot \sigma}{E}\right)^2
\]
\end{theorem}

\begin{prf}
由 $z_{\alpha/2} \cdot \dfrac{\sigma}{\sqrt{n}} \leqslant E$:
\[
\sqrt{n} \geqslant \frac{z_{\alpha/2} \cdot \sigma}{E} \ie n \geqslant \left(\frac{z_{\alpha/2} \cdot \sigma}{E}\right)^2
\]
\end{prf}

\begin{example}
已知 $\sigma = 10$。若要以 95\% 信心水準估計 $\mu$,且誤差界限不超過 2,需要多少樣本?
\end{example}

\begin{solution}
$z_{0.025} = 1.96$,$\sigma = 10$,$E = 2$

\[
n \geqslant \left(\frac{1.96 \times 10}{2}\right)^2 = (9.8)^2 = 96.04
\]

故至少需要 $n = 97$ 個樣本。
\end{solution}

\subsection{估計比例所需的樣本量}

\begin{theorem}[估計 $p$ 所需的樣本量]
若要使誤差界限不超過 $E$,則所需樣本量為:
\[
n \geqslant \left(\frac{z_{\alpha/2}}{E}\right)^2 p(1-p)
\]

若 $p$ 未知,使用\textbf{保守估計} $p = 0.5$(使 $p(1-p)$ 最大):
\[
n \geqslant \frac{(z_{\alpha/2})^2}{4E^2}
\]
\end{theorem}

\begin{prf}
$p(1-p)$ 在 $p = 0.5$ 時達到最大值 $0.25$。使用此值可確保無論真實 $p$ 為何,所需樣本量都足夠。
\end{prf}

\begin{example}
某金融公司想調查顧客對新產品的偏好比例。若要以 95\% 信心水準,使誤差界限不超過 3\%,需要多少樣本?
\end{example}

\begin{solution}
$z_{0.025} = 1.96$,$E = 0.03$

使用保守估計 $p = 0.5$:
\[
n \geqslant \frac{(1.96)^2}{4 \times (0.03)^2} = \frac{3.8416}{0.0036} = 1067.1
\]

故至少需要 $n = 1068$ 個樣本。

\textbf{若有先驗資訊}:若預估 $p \approx 0.3$,則:
\[
n \geqslant \frac{(1.96)^2 \times 0.3 \times 0.7}{(0.03)^2} = \frac{3.8416 \times 0.21}{0.0009} = 896.4
\]

需要 897 個樣本(較少)。
\end{solution}

%=============================================================================
\section{本章習題}
%=============================================================================

\begin{exercise}
設 $Y_1, \ldots, Y_T$ 為 i.i.d. 樣本,$\expc[Y_i] = \mu$,$\var(Y_i) = \sigma^2$。令 $m = \frac{\sum g(i) Y_i}{\sum g(i)}$,其中 $g(i) = 1 + (i \mod 5)$,$T$ 為 5 的倍數。

(a) 證明 $m$ 為 $\mu$ 的不偏估計量。

(b) 求 $\var(m)$ 並與 $\var(\bar{Y})$ 比較。
\end{exercise}

\begin{solution}
\textbf{(a)} $g(i)$ 以週期 5 循環,取值 $\{2, 3, 4, 5, 1\}$,每週期和為 15。設 $T = 5N$,則 $\sum g(i) = 15N = 3T$。

$\expc[m] = \frac{1}{3T}\sum g(i)\expc[Y_i] = \frac{\mu}{3T} \cdot 3T = \mu$ $\checkmark$

\textbf{(b)} 每週期平方和 $= 4 + 9 + 16 + 25 + 1 = 55$,$\sum[g(i)]^2 = 55N = 11T$。

$\var(m) = \frac{\sigma^2}{9T^2} \cdot 11T = \frac{11\sigma^2}{9T}$

$\var(\bar{Y}) = \sigma^2/T$

$\var(m)/\var(\bar{Y}) = 11/9 \approx 1.22 > 1$

$m$ 效率較低,因為非等權重平均。
\end{solution}

\begin{exercise}
設 $X_1, \ldots, X_n \simark \text{Uniform}(0, \theta)$,PDF 為 $f(x; \theta) = 1/\theta$,$0 < x < \theta$。求 $\theta$ 的 MLE。
\end{exercise}

\begin{solution}
概似函數:
\[
L(\theta) = \prod_{i=1}^{n} \frac{1}{\theta} \cdot \mathbf{1}_{(0 < x_i < \theta)} = \frac{1}{\theta^n} \cdot \mathbf{1}_{(\theta > \max_i x_i)}
\]

對於 $\theta \geqslant \max_i x_i$,$L(\theta) = 1/\theta^n$ 是 $\theta$ 的遞減函數。

因此 $L(\theta)$ 在 $\theta = \max_i x_i$ 時達到最大。

$\hat{\theta}_{\text{MLE}} = X_{(n)} = \max\{X_1, \ldots, X_n\}$
\end{solution}

\begin{exercise}
設 $X_1, \ldots, X_n \simark N(\mu, \sigma^2)$。考慮 $\sigma^2$ 的兩個估計量:
\begin{itemize}
    \item $\hat{\sigma}_1^2 = S^2 = \dfrac{1}{n-1}\sum(X_i - \bar{X})^2$
    \item $\hat{\sigma}_2^2 = \dfrac{1}{n}\sum(X_i - \bar{X})^2$
\end{itemize}
\begin{enumerate}[label=(\alph*)]
    \item 證明 $\hat{\sigma}_1^2$ 是不偏的,$\hat{\sigma}_2^2$ 是有偏的
    \item 求 $\hat{\sigma}_2^2$ 的偏誤
\end{enumerate}
\end{exercise}

\begin{solution}
\begin{enumerate}[label=(\alph*)]
    \item 由第三部分已證 $\expc[S^2] = \sigma^2$,故 $\hat{\sigma}_1^2$ 不偏。
    
    $\hat{\sigma}_2^2 = \dfrac{n-1}{n} S^2$,故:
    \[
    \expc[\hat{\sigma}_2^2] = \frac{n-1}{n} \expc[S^2] = \frac{n-1}{n} \sigma^2 \neq \sigma^2
    \]
    
    故 $\hat{\sigma}_2^2$ 有偏。
    
    \item $\text{Bias}(\hat{\sigma}_2^2) = \expc[\hat{\sigma}_2^2] - \sigma^2 = \dfrac{n-1}{n}\sigma^2 - \sigma^2 = -\dfrac{\sigma^2}{n}$
\end{enumerate}
\end{solution}

\begin{exercise}
某廠商宣稱其產品平均重量為 500 克。品管人員隨機抽取 25 件產品,測得平均重量 496 克,樣本標準差 10 克。假設重量服從常態分配。
\begin{enumerate}[label=(\alph*)]
    \item 求母體平均重量的 95\% 信賴區間
    \item 根據信賴區間,廠商的宣稱是否可信?
\end{enumerate}
\end{exercise}

\begin{solution}
$n = 25$,$\bar{x} = 496$,$s = 10$

\begin{enumerate}[label=(\alph*)]
    \item $\sigma$ 未知,使用 t 分配。$df = 24$,$t_{0.025, 24} = 2.064$
    
    95\% 信賴區間:
    \[
    496 \pm 2.064 \times \frac{10}{\sqrt{25}} = 496 \pm 2.064 \times 2 = 496 \pm 4.13
    \]
    
    即 $(491.87, 500.13)$
    
    \item 500 落在信賴區間內,故在 95\% 信心水準下,廠商的宣稱可信。
\end{enumerate}
\end{solution}

\begin{exercise}
比較兩種教學法的效果。A 法:$n_1 = 15$,$\bar{x}_1 = 78$,$s_1 = 8$;B 法:$n_2 = 18$,$\bar{x}_2 = 72$,$s_2 = 10$。假設成績服從常態分配且變異數相等。求 $\mu_A - \mu_B$ 的 95\% 信賴區間。
\end{exercise}

\begin{solution}
合併變異數:
\[
S_p^2 = \frac{14 \times 64 + 17 \times 100}{15 + 18 - 2} = \frac{896 + 1700}{31} = \frac{2596}{31} = 83.74
\]

$S_p = 9.15$

$df = 31$,$t_{0.025, 31} \approx 2.04$

標準誤:$9.15 \times \sqrt{\dfrac{1}{15} + \dfrac{1}{18}} = 9.15 \times \sqrt{0.1222} = 9.15 \times 0.350 = 3.20$

95\% 信賴區間:
\[
(78 - 72) \pm 2.04 \times 3.20 = 6 \pm 6.53
\]

即 $(-0.53, 12.53)$

區間包含 0,故在 95\% 信心水準下,無法斷定兩種教學法有顯著差異。
\end{solution}

\begin{exercise}
某研究者想估計大學生每天使用手機的平均時間。預估標準差約為 1.5 小時。若要以 99\% 信心水準,使誤差界限不超過 0.5 小時,需要多少樣本?
\end{exercise}

\begin{solution}
$z_{0.005} = 2.576$,$\sigma = 1.5$,$E = 0.5$

\[
n \geqslant \left(\frac{2.576 \times 1.5}{0.5}\right)^2 = \left(\frac{3.864}{0.5}\right)^2 = (7.728)^2 = 59.72
\]

故至少需要 60 個樣本。
\end{solution}

%=============================================================================
%% \section*{本章重點整理}
%=============================================================================

% \begin{enumerate}
%     \item \textbf{點估計的性質}:
%     \begin{itemize}
%         \item 不偏性:$\expc[\hat{\theta}] = \theta$
%         \item 有效性:變異數最小
%         \item 一致性:$\hat{\theta}_n \pmark \theta$
%         \item MSE 分解:$\MSE = \var + \text{Bias}^2$
%     \end{itemize}
    
%     \item \textbf{最大概似估計法}:
%     \begin{itemize}
%         \item 概似函數:$L(\theta) = \prod f(x_i; \theta)$
%         \item 求法:令 $\dfrac{d\ell}{d\theta} = 0$
%         \item 常見 MLE:常態 $\to \bar{X}$,指數 $\to 1/\bar{X}$,Poisson $\to \bar{X}$,Bernoulli $\to \hat{p}$
%     \end{itemize}
    
%     \item \textbf{母體平均數 $\mu$ 的信賴區間}:
%     \begin{itemize}
%         \item $\sigma$ 已知:$\bar{X} \pm z_{\alpha/2} \cdot \sigma/\sqrt{n}$
%         \item $\sigma$ 未知:$\bar{X} \pm t_{\alpha/2, n-1} \cdot S/\sqrt{n}$
%     \end{itemize}
    
%     \item \textbf{母體變異數 $\sigma^2$ 的信賴區間}:
%     \[
%     \left(\frac{(n-1)S^2}{\chi^2_{\alpha/2}}, \frac{(n-1)S^2}{\chi^2_{1-\alpha/2}}\right)
%     \]
    
%     \item \textbf{母體比例 $p$ 的信賴區間}:
%     \[
%     \hat{p} \pm z_{\alpha/2} \sqrt{\frac{\hat{p}(1-\hat{p})}{n}}
%     \]
    
%     \item \textbf{樣本量決定}:
%     \begin{itemize}
%         \item 估計 $\mu$:$n \geqslant (z_{\alpha/2} \cdot \sigma / E)^2$
%         \item 估計 $p$(保守):$n \geqslant (z_{\alpha/2})^2 / (4E^2)$
%     \end{itemize}
% \end{enumerate}

\end{document}
