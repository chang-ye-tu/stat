\documentclass[12pt,a4paper]{article}
\usepackage[left=1cm,right=1cm,bottom=15mm,top=20mm]{geometry}
\usepackage[AutoFakeBold,AutoFakeSlant]{xeCJK}
\setCJKmainfont[AutoFakeSlant=.1,AutoFakeBold=2]{Noto Serif CJK TC}
\usepackage{amsmath,amsthm,amssymb,amsfonts}
\usepackage{graphicx,xcolor,float}
\usepackage{booktabs,tabularx,multirow,array}
\usepackage{enumitem}
\usepackage{parskip}
\setlist{itemsep=0pt,parsep=0pt}
\usepackage{hyperref}
\hypersetup{
    colorlinks=true,
    linkcolor=blue!70!black,
    urlcolor=blue!80!black
}

% 定理環境
\theoremstyle{definition}
\newtheorem{definition}{定義}[section]
\newtheorem{example}{例題}[section]
\newtheorem{exercise}{練習題}[section]
\newtheorem{theorem}{定理}[section]
\newtheorem{lemma}{引理}[section]
\newtheorem{corollary}{推論}[section]
\newtheorem{proposition}{命題}[section]
\newtheorem{property}{性質}[section]
\newtheorem*{remark}{註}
\newtheorem*{solution}{解答}
\newtheorem*{note}{說明}
\newtheorem*{prf}{證明}

% 常用指令
\newcommand{\ds}{\displaystyle}
\newcommand{\ie}{\;\Longrightarrow\;}
\newcommand{\ifff}{\;\Longleftrightarrow\;}
\newcommand\expc{\mathsf{E}}
\DeclareMathOperator\var{var}
\DeclareMathOperator\cov{cov}
\DeclareMathOperator\corr{corr}
\newcommand{\SE}{\mathrm{SE}}
\newcommand\prb{\mathsf{P}}
\newcommand{\Real}{\mathbb{R}}
\newcommand{\Nat}{\mathbb{N}}
\newcommand{\diff}[2]{\frac{\mathrm{d} #1}{\mathrm{d} #2}}
\newcommand{\pdiff}[2]{\frac{\partial #1}{\partial #2}}
\newcommand{\simark}{\stackrel{\text{i.i.d.}}{\sim}}
\newcommand{\pmark}{\stackrel{p}{\longrightarrow}}

% 頁面設定
\renewcommand{\figurename}{圖}
\renewcommand{\tablename}{表}

\usepackage{fancyhdr}
\pagestyle{fancy}
\fancyhf{}
\fancyhead[L]{統計學講義}
\fancyhead[R]{第五部分:假設檢定}
\fancyfoot[C]{\thepage}
\renewcommand{\headrulewidth}{0.4pt}
\renewcommand{\footrulewidth}{0.4pt}

\title{\vspace{-2cm}\textbf{統計學講義}\\[3mm] \Large 第五部分:假設檢定}
\author{}
\date{\vspace{-2cm}}

\begin{document}
\maketitle
\thispagestyle{fancy}

\begin{center}
\fbox{\parbox{0.9\textwidth}{\centering
\textbf{參考書籍}\\[2mm]
Jeffrey S. Rosenthal, \textit{Probability and Statistics: The Science of Uncertainty}, 2nd Edition\\
Chapter 8: Hypothesis Testing; Chapter 9: Further Topics\\[2mm]
\url{https://utstat.utoronto.ca/mikevans/jeffrosenthal/}
}}
\end{center}

%\tableofcontents

%=============================================================================
\section{假設檢定的基本概念}
%=============================================================================

\subsection{統計假設}

\begin{definition}[統計假設]
\textbf{統計假設}(statistical hypothesis)是關於母體參數或母體分配的陳述。
\begin{itemize}
  \item \textbf{虛無假設}(null hypothesis),記為 $H_0$:欲檢驗的假設,通常代表「無效果」、「無差異」或現狀,且包含等號。
  \item \textbf{對立假設}(alternative hypothesis),記為 $H_1$ 或 $H_a$:與 $H_0$ 相對的假設,通常代表研究者想要證明的主張。
\end{itemize}
\end{definition}

\begin{example}[假設的設定]
\begin{itemize}
  \item[]
  \item 檢驗藥物是否有效:$H_0$:藥物無效 vs $H_1$:藥物有效
  \item 檢驗產品平均重量:$H_0: \mu = 500$ vs $H_1: \mu \neq 500$
  \item 檢驗新製程是否提高良率:$H_0: p \leqslant 0.9$ vs $H_1: p > 0.9$
\end{itemize}
\end{example}

\subsection{檢定類型}

\begin{definition}[單尾與雙尾檢定]
設母體參數為 $\theta$,檢定值為 $\theta_0$:
\begin{enumerate}[label=(\roman*)]
  \item \textbf{雙尾檢定}(two-tailed test):
    \[
    H_0: \theta = \theta_0 \quad \text{vs} \quad H_1: \theta \neq \theta_0
    \]
  \item \textbf{右尾檢定}(right-tailed test):
    \[
    H_0: \theta \leqslant \theta_0 \quad \text{vs} \quad H_1: \theta > \theta_0
    \]
  \item \textbf{左尾檢定}(left-tailed test):
    \[
    H_0: \theta \geqslant \theta_0 \quad \text{vs} \quad H_1: \theta < \theta_0
    \]
\end{enumerate}
\end{definition}

\begin{remark}
在實務中,$H_0$ 常簡寫為 $\theta = \theta_0$,即使對立假設是單尾的。這是因為檢定統計量的分配通常在 $\theta = \theta_0$ 時計算。
\end{remark}

\subsection{檢定程序}

\begin{property}[假設檢定的步驟]
\begin{enumerate}
  \item[]
    \item \textbf{建立假設}:根據問題設定 $H_0$ 和 $H_1$
    \item \textbf{選擇顯著水準}:決定 $\alpha$(常用 0.05、0.01、0.10)
    \item \textbf{計算檢定統計量}:根據樣本資料計算適當的統計量
    \item \textbf{決定拒絕域或計算 p 值}
    \item \textbf{做出結論}:拒絕或不拒絕 $H_0$
\end{enumerate}
\end{property}

%=============================================================================
\section{檢定錯誤與檢定力}
%=============================================================================

\subsection{兩類錯誤}

\begin{definition}[型一錯誤與型二錯誤]
\begin{table}[H]
\centering
\renewcommand{\arraystretch}{1.3}
\begin{tabular}{ccc}
\toprule
& $H_0$ 為真 & $H_0$ 為假 \\
\midrule
不拒絕 $H_0$ & 正確決策 & \textbf{型二錯誤}($\beta$)\\
拒絕 $H_0$ & \textbf{型一錯誤}($\alpha$) & 正確決策(檢定力)\\
\bottomrule
\end{tabular}
\end{table}
\begin{itemize}
  \item[]
    \item \textbf{型一錯誤}(Type I Error):$H_0$ 為真卻拒絕 $H_0$(偽陽性;拒絕不該拒絕的)
    \[
    \alpha = \prb(\text{拒絕 } H_0 \mid H_0 \text{ 為真})
    \]
    $\alpha$ 稱為\textbf{顯著水準}(significance level)。
    
  \item \textbf{型二錯誤}(Type II Error):$H_0$ 為假卻不拒絕 $H_0$(偽陰性;接受不該接受的)
    \[
    \beta = \prb(\text{不拒絕 } H_0 \mid H_0 \text{ 為假})
    \]
\end{itemize}
\end{definition}

\begin{remark}
\begin{itemize}
  \item[]
    \item 型一錯誤與型二錯誤是\textbf{互相牽制}的:在固定樣本量下,降低 $\alpha$ 通常會增加 $\beta$。
    \item 一般優先控制型一錯誤(固定 $\alpha$),因為型一錯誤的後果通常較嚴重(如錯誤地宣稱新藥有效)。
    \item 增加樣本量可以同時降低 $\alpha$ 和 $\beta$。
\end{itemize}
\end{remark}

\subsection{檢定力}

\begin{definition}[檢定力]
\textbf{檢定力}(power)是當 $H_0$ 為假時,正確拒絕 $H_0$ 的機率:
\[
\text{Power} = 1 - \beta = \prb(\text{拒絕 } H_0 \mid H_0 \text{ 為假})
\]
\end{definition}

\begin{theorem}[檢定力的影響因素]
檢定力會隨以下因素而增加:
\begin{enumerate}[label=(\roman*)]
    \item 效應量(effect size)增加:真實參數值離 $H_0$ 假設值越遠
    \item 樣本量 $n$ 增加
    \item 顯著水準 $\alpha$ 增加
    \item 母體變異數 $\sigma^2$ 減少
\end{enumerate}
\end{theorem}

\subsection{型二錯誤機率的計算}

\begin{property}[計算 $\beta$ 的步驟]
\begin{enumerate}
  \item[]
    \item 在 $H_0$ 下,找出\textbf{不拒絕域}(接受域)
    \item 假設真實參數值為 $\theta_1$($H_1$ 下的某個特定值)
    \item 在 $\theta = \theta_1$ 下,計算統計量落在不拒絕域的機率
\end{enumerate}
\end{property}

\begin{example}\label{ex:beta}
設 $X_1, \ldots, X_{25} \simark N(\mu, 100)$(即 $\sigma = 10$)。檢定 $H_0: \mu = 50$ vs $H_1: \mu > 50$,顯著水準 $\alpha = 0.05$。若真實 $\mu = 54$,求型二錯誤機率 $\beta$。
\end{example}

\begin{solution}
\textbf{步驟 1}:在 $H_0: \mu = 50$ 下找不拒絕域

檢定統計量:$Z = \dfrac{\bar{X} - 50}{10/\sqrt{25}} = \dfrac{\bar{X} - 50}{2}$

右尾檢定,拒絕域:$Z > z_{0.05} = 1.645$

即 $\bar{X} > 50 + 1.645 \times 2 = 53.29$ 時拒絕 $H_0$

不拒絕域:$\bar{X} \leqslant 53.29$

\textbf{步驟 2}:在真實 $\mu = 54$ 下計算 $\beta$

當 $\mu = 54$ 時,$\bar{X} \sim N(54, 4)$(標準差 2)
\begin{align*}
\beta &= \prb(\bar{X} \leqslant 53.29 \mid \mu = 54)\\
&= P\left(Z \leqslant \frac{53.29 - 54}{2}\right)\\
&= \prb(Z \leqslant -0.355)\\
&= \Phi(-0.355) \approx 0.361
\end{align*}

\textbf{結論}:型二錯誤機率約為 36.1\%,檢定力 $= 1 - 0.361 = 0.639$(約 64\%)。
\end{solution}

\begin{example}
延續例 \ref{ex:beta},畫出檢定力函數 $\text{Power}(\mu)$,並求使檢定力達到 0.90 所需的樣本量。
\end{example}

\begin{solution}
\textbf{檢定力函數}:

對於任意 $\mu > 50$:
\begin{align*}
\text{Power}(\mu) &= \prb(\bar{X} > 53.29 \mid \mu)\\
&= P\left(Z > \frac{53.29 - \mu}{2}\right)\\
&= 1 - \Phi\left(\frac{53.29 - \mu}{2}\right)
\end{align*}

當 $\mu = 54$:$\text{Power} = 1 - \Phi(-0.355) = 0.639$

當 $\mu = 56$:$\text{Power} = 1 - \Phi(-1.355) = 0.912$

\textbf{求所需樣本量}(使 Power = 0.90 當 $\mu = 54$):

設樣本量為 $n$,標準誤 $= 10/\sqrt{n}$

拒絕域臨界值:$c = 50 + 1.645 \times \dfrac{10}{\sqrt{n}}$

\begin{align*}
\text{Power} &= P\left(Z > \frac{c - 54}{10/\sqrt{n}}\right) = 0.90
\end{align*}

需要 $\dfrac{c - 54}{10/\sqrt{n}} = -z_{0.10} = -1.282$

代入 $c = 50 + 1.645 \times \dfrac{10}{\sqrt{n}}$:
\[
\frac{50 + 1.645 \times \frac{10}{\sqrt{n}} - 54}{10/\sqrt{n}} = -1.282
\]
\[
\frac{-4 + \frac{16.45}{\sqrt{n}}}{10/\sqrt{n}} = -1.282
\]
\[
\frac{-4\sqrt{n} + 16.45}{10} = -1.282
\]
\[
-4\sqrt{n} + 16.45 = -12.82
\]
\[
\sqrt{n} = \frac{16.45 + 12.82}{4} = 7.32
\]
\[
n = 53.6
\]

故至少需要 $n = 54$ 個樣本。
\end{solution}

%=============================================================================
\section{p 值}
%=============================================================================

\subsection{p 值的定義}

\begin{definition}[p 值]
\textbf{p 值}(p-value)是在 $H_0$ 為真的假設下,觀察到與當前樣本統計量\textbf{一樣極端或更極端}的結果之機率。
\begin{itemize}
    \item 右尾檢定:$\text{p-value} = \prb(T \geqslant t_{\text{obs}} \mid H_0)$
    \item 左尾檢定:$\text{p-value} = \prb(T \leqslant t_{\text{obs}} \mid H_0)$
    \item 雙尾檢定:$\text{p-value} = 2 \times \prb(T \geqslant |t_{\text{obs}}| \mid H_0)$
\end{itemize}
其中 $t_{\text{obs}}$ 是觀察到的檢定統計量值。
\end{definition}

\begin{property}[p 值的決策規則]
\begin{itemize}
  \item[]
    \item 若 p-value $< \alpha$,則拒絕 $H_0$
    \item 若 p-value $\geqslant \alpha$,則不拒絕 $H_0$
\end{itemize}
\end{property}

\begin{theorem}[p 值與拒絕域的等價性]
  p-value $< \alpha$ $\ifff$ 檢定統計量落在拒絕域內。
\end{theorem}

\begin{prf}
以右尾檢定為例。設檢定統計量為 $T$,臨界值為 $c_\alpha$(滿足 $\prb(T > c_\alpha \mid H_0) = \alpha$)。

觀察到 $t_{\text{obs}}$,則:
\begin{align*}
\text{p-value} < \alpha &\ifff \prb(T \geqslant t_{\text{obs}} \mid H_0) < \alpha\\
&\ifff \prb(T \geqslant t_{\text{obs}} \mid H_0) < \prb(T > c_\alpha \mid H_0)\\
&\ifff t_{\text{obs}} > c_\alpha\\
&\ifff t_{\text{obs}} \text{ 落在拒絕域內}
\end{align*}
\end{prf}

\subsection{p 值的解釋}

\begin{remark}[p 值的正確與錯誤解釋]
\textbf{正確解釋}:
\begin{itemize}
    \item p 值是在 $H_0$ 為真時,觀察到目前結果(或更極端結果)的機率
    \item p 值越小,反對 $H_0$ 的證據越強
\end{itemize}
\textbf{錯誤解釋}(常見誤解):
\begin{itemize}
    \item \textbf{錯}:p 值是 $H_0$ 為真的機率
    \item \textbf{錯}:p 值是結果由隨機造成的機率
    \item \textbf{錯}:$1 - \text{p-value}$ 是效應存在的機率
\end{itemize}
\end{remark}

\begin{example}
設 $X_1, \ldots, X_{16} \simark N(\mu, 64)$($\sigma = 8$)。樣本平均 $\bar{x} = 53$。檢定 $H_0: \mu = 50$ vs $H_1: \mu \neq 50$。求 p 值並在 $\alpha = 0.05$ 下做結論。
\end{example}

\begin{solution}
檢定統計量:
\[
z = \frac{\bar{x} - \mu_0}{\sigma/\sqrt{n}} = \frac{53 - 50}{8/\sqrt{16}} = \frac{3}{2} = 1.5
\]

雙尾檢定的 p 值:
\[
\text{p-value} = 2 \times \prb(Z \geqslant 1.5) = 2 \times (1 - \Phi(1.5)) = 2 \times 0.0668 = 0.1336
\]

由於 p-value $= 0.1336 > 0.05 = \alpha$,不拒絕 $H_0$。

結論:在 $\alpha = 0.05$ 下,沒有足夠證據說明 $\mu \neq 50$。
\end{solution}

%=============================================================================
\section{單樣本平均數檢定}
%=============================================================================

\subsection{$z$ 檢定($\sigma$ 已知)}

\begin{theorem}[單樣本 $z$ 檢定]
設 $X_1, \ldots, X_n \simark N(\mu, \sigma^2)$,$\sigma^2$ 已知。檢定 $H_0: \mu = \mu_0$。

\textbf{檢定統計量}:
\[
Z = \frac{\bar{X} - \mu_0}{\sigma/\sqrt{n}} \sim N(0, 1) \quad \text{(在 $H_0$ 下)}
\]

\textbf{拒絕域}(顯著水準 $\alpha$):
\begin{itemize}
    \item 雙尾($H_1: \mu \neq \mu_0$):$|Z| > z_{\alpha/2}$
    \item 右尾($H_1: \mu > \mu_0$):$Z > z_{\alpha}$
    \item 左尾($H_1: \mu < \mu_0$):$Z < -z_{\alpha}$
\end{itemize}
\end{theorem}

\begin{note}[為何 $Z$ 服從標準常態?]
由於 $X_i \simark N(\mu, \sigma^2)$,樣本平均數 $\bar{X} \sim N(\mu, \sigma^2/n)$。標準化後:
\[
Z = \frac{\bar{X} - \mu}{\sigma/\sqrt{n}} \sim N(0, 1)
\]
在 $H_0: \mu = \mu_0$ 下,以 $\mu_0$ 代入 $\mu$,得到檢定統計量。
\end{note}

\subsection{$t$ 檢定($\sigma$ 未知)}

\begin{theorem}[單樣本 $t$ 檢定]
設 $X_1, \ldots, X_n \simark N(\mu, \sigma^2)$,$\sigma^2$ 未知。檢定 $H_0: \mu = \mu_0$。

\textbf{檢定統計量}:
\[
T = \frac{\bar{X} - \mu_0}{S/\sqrt{n}} \sim t_{n-1} \quad \text{(在 $H_0$ 下)}
\]

\textbf{拒絕域}(顯著水準 $\alpha$):
\begin{itemize}
    \item 雙尾($H_1: \mu \neq \mu_0$):$|T| > t_{\alpha/2, n-1}$
    \item 右尾($H_1: \mu > \mu_0$):$T > t_{\alpha, n-1}$
    \item 左尾($H_1: \mu < \mu_0$):$T < -t_{\alpha, n-1}$
\end{itemize}
\end{theorem}

\begin{note}[為何 $T$ 服從 $t$ 分配?]
將 $z$ 檢定中的 $\sigma$ 替換成樣本標準差 $S$。由 $t$ 分配的定義:
\[
T = \frac{\bar{X} - \mu}{S/\sqrt{n}} = \frac{(\bar{X} - \mu)/(\sigma/\sqrt{n})}{S/\sigma} = \frac{Z}{\sqrt{(n-1)S^2/\sigma^2/(n-1)}} = \frac{Z}{\sqrt{V/(n-1)}}
\]
其中 $Z \sim N(0,1)$,$V = (n-1)S^2/\sigma^2 \sim \chi^2_{n-1}$,且 $Z$ 與 $V$ 獨立(常態母體下 $\bar{X}$ 與 $S^2$ 獨立),故 $T \sim t_{n-1}$。
\end{note}

\begin{example}
  某便利商店宣稱每日庫存量平均為 500 件。品管人員隨機抽查 25 天,得平均庫存量 485 件,樣本標準差 30 件。在 $\alpha = 0.05$ 下,檢定庫存量是否低於宣稱值。
\end{example}

\begin{solution}
\textbf{步驟 1}:建立假設

$H_0: \mu \geqslant 500$ vs $H_1: \mu < 500$(左尾檢定)

\textbf{步驟 2}:計算檢定統計量

$\sigma$ 未知,使用 $t$ 檢定:
\[
t = \frac{\bar{x} - \mu_0}{s/\sqrt{n}} = \frac{485 - 500}{30/\sqrt{25}} = \frac{-15}{6} = -2.5
\]

\textbf{步驟 3}:確定拒絕域

$df = 24$,$\alpha = 0.05$,左尾檢定

$t_{0.05, 24} = 1.711$,拒絕域:$t < -1.711$

\textbf{步驟 4}:結論

$t = -2.5 < -1.711$,落在拒絕域內,拒絕 $H_0$。

\textbf{p 值計算}:$\text{p-value} = \prb(T_{24} < -2.5) \approx 0.01$

結論:在 $\alpha = 0.05$ 下,有足夠證據顯示庫存量低於宣稱的 500 件。
\end{solution}

%=============================================================================
\section{雙樣本平均數檢定}
%=============================================================================

\subsection{獨立樣本 $t$ 檢定}

\begin{theorem}[獨立樣本 $t$ 檢定(假設變異數相等)]
設兩獨立樣本 $X_1, \ldots, X_{n_1} \simark N(\mu_1, \sigma^2)$ 與 $Y_1, \ldots, Y_{n_2} \simark N(\mu_2, \sigma^2)$。

檢定 $H_0: \mu_1 - \mu_2 = \delta_0$(通常 $\delta_0 = 0$)。

\textbf{合併變異數估計}:
\[
S_p^2 = \frac{(n_1-1)S_1^2 + (n_2-1)S_2^2}{n_1 + n_2 - 2}
\]

\textbf{檢定統計量}:
\[
T = \frac{(\bar{X} - \bar{Y}) - \delta_0}{S_p\sqrt{\frac{1}{n_1} + \frac{1}{n_2}}} \sim t_{n_1+n_2-2} \quad \text{(在 $H_0$ 下)}
\]
\end{theorem}

\begin{prf}
由第三部分的抽樣分配結論,當 $\sigma_1^2 = \sigma_2^2 = \sigma^2$ 時:
\[
\frac{(\bar{X} - \bar{Y}) - (\mu_1 - \mu_2)}{\sigma\sqrt{\frac{1}{n_1} + \frac{1}{n_2}}} \sim N(0, 1)
\]
且 $\dfrac{(n_1-1)S_1^2 + (n_2-1)S_2^2}{\sigma^2} \sim \chi^2_{n_1+n_2-2}$。由 $t$ 分配的定義,將 $\sigma$ 用 $S_p$ 估計後得到 $t_{n_1+n_2-2}$ 分配。
\end{prf}

\begin{example}
  比較兩種教學法。A 法:$n_1 = 12$,$\bar{x}_1 = 78$,$s_1 = 8$;B 法:$n_2 = 15$,$\bar{x}_2 = 72$,$s_2 = 10$。假設成績服從常態分配且變異數相等。在 $\alpha = 0.05$ 下,檢定兩種教學法是否有顯著差異。
\end{example}

\begin{solution}
\textbf{假設}:$H_0: \mu_1 = \mu_2$ vs $H_1: \mu_1 \neq \mu_2$

\textbf{合併變異數}:
\[
S_p^2 = \frac{11 \times 64 + 14 \times 100}{12 + 15 - 2} = \frac{704 + 1400}{25} = \frac{2104}{25} = 84.16
\]

$S_p = 9.17$

\textbf{檢定統計量}:
\[
t = \frac{78 - 72}{9.17\sqrt{\frac{1}{12} + \frac{1}{15}}} = \frac{6}{9.17 \times 0.391} = \frac{6}{3.59} = 1.67
\]

\textbf{臨界值}:$df = 25$,$t_{0.025, 25} = 2.060$

\textbf{結論}:$|t| = 1.67 < 2.060$,不拒絕 $H_0$。

在 $\alpha = 0.05$ 下,沒有足夠證據顯示兩種教學法有顯著差異。
\end{solution}

\subsection{配對樣本 $t$ 檢定}

\begin{definition}[配對樣本]
當兩組觀測值存在自然配對關係時,應使用\textbf{配對樣本 $t$ 檢定}:
\begin{itemize}
    \item 同一受試者的前後測量(before-after)
    \item 雙胞胎研究
    \item 配對比較實驗
\end{itemize}
\end{definition}

\begin{theorem}[配對樣本 $t$ 檢定]
設配對觀測值為 $(X_1, Y_1), \ldots, (X_n, Y_n)$,令差異 $D_i = X_i - Y_i$。

假設 $D_i \simark N(\mu_D, \sigma_D^2)$。檢定 $H_0: \mu_D = 0$(即 $\mu_1 = \mu_2$)。

\textbf{檢定統計量}:
\[
T = \frac{\bar{D} - 0}{S_D/\sqrt{n}} \sim t_{n-1} \quad \text{(在 $H_0$ 下)}
\]

其中 $\bar{D} = \dfrac{1}{n}\sum_{i=1}^{n} D_i$,$S_D^2 = \dfrac{1}{n-1}\sum_{i=1}^{n}(D_i - \bar{D})^2$。
\end{theorem}

\begin{remark}
配對樣本 $t$ 檢定本質上是對「差異」進行單樣本 $t$ 檢定,自由度為 $n-1$(配對數減 1)。
\end{remark}

\begin{example}
  8 輛汽車分別使用舊引擎和新引擎,測量燃油效率(km/L)如下:
  \begin{center}
    \begin{tabular}{c|cccccccc}
    \toprule
    車輛 & 1 & 2 & 3 & 4 & 5 & 6 & 7 & 8 \\
    \midrule
    新引擎 & 12 & 15 & 14 & 13 & 16 & 14 & 15 & 13 \\
    舊引擎 & 10 & 13 & 12 & 12 & 14 & 12 & 13 & 11 \\
    \midrule
    差異 $D$ & 2 & 2 & 2 & 1 & 2 & 2 & 2 & 2 \\
    \bottomrule
    \end{tabular}
  \end{center}
  在 $\alpha = 0.01$ 下,檢定新引擎是否顯著提高燃油效率。
\end{example}

\begin{solution}
\textbf{假設}:$H_0: \mu_D \leqslant 0$ vs $H_1: \mu_D > 0$(右尾檢定)

\textbf{計算差異的統計量}:
\[
\bar{D} = \frac{2+2+2+1+2+2+2+2}{8} = \frac{15}{8} = 1.875
\]

\[
S_D^2 = \frac{\sum(D_i - \bar{D})^2}{7} = \frac{(0.125)^2 \times 7 + (0.875)^2}{7} = \frac{0.109 + 0.766}{7} = \frac{0.875}{7} = 0.125
\]

$S_D = 0.354$

\textbf{檢定統計量}:
\[
t = \frac{1.875 - 0}{0.354/\sqrt{8}} = \frac{1.875}{0.125} = 15.0
\]

\textbf{臨界值}:$df = 7$,$t_{0.01, 7} = 2.998$

\textbf{結論}:$t = 15.0 > 2.998$,拒絕 $H_0$。

有非常強的證據顯示新引擎顯著提高燃油效率(p-value $< 0.001$)。
\end{solution}

%=============================================================================
\section{比例檢定}
%=============================================================================

\subsection{單一比例檢定}

\begin{theorem}[單一比例 $z$ 檢定]
設 $X$ 為 $n$ 次獨立伯努利試驗中成功的次數,$X \sim B(n, p)$。

檢定 $H_0: p = p_0$。

\textbf{檢定統計量}($n$ 夠大時):
\[
Z = \frac{\hat{p} - p_0}{\sqrt{\frac{p_0(1-p_0)}{n}}} \stackrel{\text{approx}}{\sim} N(0, 1) \quad \text{(在 $H_0$ 下)}
\]

其中 $\hat{p} = X/n$。
\end{theorem}

\begin{remark}
注意:檢定統計量的分母使用 $H_0$ 下的 $p_0$,而非樣本估計值 $\hat{p}$。這與信賴區間不同。
\end{remark}

\begin{example}
某候選人宣稱其支持率為 50\%。隨機調查 400 位選民,有 220 人支持。在 $\alpha = 0.05$ 下,檢定支持率是否顯著高於 50\%。
\end{example}

\begin{solution}
\textbf{假設}:$H_0: p \leqslant 0.5$ vs $H_1: p > 0.5$

\textbf{樣本比例}:$\hat{p} = 220/400 = 0.55$

\textbf{檢定統計量}:
\[
z = \frac{0.55 - 0.50}{\sqrt{\frac{0.5 \times 0.5}{400}}} = \frac{0.05}{0.025} = 2.0
\]

\textbf{臨界值}:$z_{0.05} = 1.645$

\textbf{結論}:$z = 2.0 > 1.645$,拒絕 $H_0$。

p-value $= \prb(Z > 2.0) = 0.0228$

在 $\alpha = 0.05$ 下,有足夠證據顯示支持率高於 50\%。
\end{solution}

\subsection{兩母體比例差檢定}

\begin{theorem}[兩母體比例差檢定]
設兩獨立樣本的樣本比例為 $\hat{p}_1 = X_1/n_1$ 和 $\hat{p}_2 = X_2/n_2$。

檢定 $H_0: p_1 = p_2$。

\textbf{合併比例估計}:
\[
\hat{p} = \frac{X_1 + X_2}{n_1 + n_2}
\]

\textbf{檢定統計量}:
\[
Z = \frac{\hat{p}_1 - \hat{p}_2}{\sqrt{\hat{p}(1-\hat{p})\left(\frac{1}{n_1} + \frac{1}{n_2}\right)}} \stackrel{\text{approx}}{\sim} N(0, 1) \quad \text{(在 $H_0$ 下)}
\]
\end{theorem}

%=============================================================================
\section{變異數檢定}
%=============================================================================

\subsection{單一變異數檢定}

\begin{theorem}[單一變異數的 $\chi^2$ 檢定]
設 $X_1, \ldots, X_n \simark N(\mu, \sigma^2)$。檢定 $H_0: \sigma^2 = \sigma_0^2$。

\textbf{檢定統計量}:
\[
\chi^2 = \frac{(n-1)S^2}{\sigma_0^2} \sim \chi^2_{n-1} \quad \text{(在 $H_0$ 下)}
\]

\textbf{拒絕域}:
\begin{itemize}
    \item 雙尾($H_1: \sigma^2 \neq \sigma_0^2$):$\chi^2 < \chi^2_{1-\alpha/2, n-1}$ 或 $\chi^2 > \chi^2_{\alpha/2, n-1}$
    \item 右尾($H_1: \sigma^2 > \sigma_0^2$):$\chi^2 > \chi^2_{\alpha, n-1}$
    \item 左尾($H_1: \sigma^2 < \sigma_0^2$):$\chi^2 < \chi^2_{1-\alpha, n-1}$
\end{itemize}
\end{theorem}

\begin{note}[為何使用卡方分配?]
由第三部分定理,對常態母體:$(n-1)S^2/\sigma^2 \sim \chi^2_{n-1}$。在 $H_0: \sigma^2 = \sigma_0^2$ 下,以 $\sigma_0^2$ 代入即得檢定統計量。
\end{note}

\subsection{兩變異數比檢定}

\begin{theorem}[兩變異數比的 $F$ 檢定]
設兩獨立常態樣本的樣本變異數為 $S_1^2$ 和 $S_2^2$。檢定 $H_0: \sigma_1^2 = \sigma_2^2$。

\textbf{檢定統計量}:
\[
F = \frac{S_1^2}{S_2^2} \sim F_{n_1-1, n_2-1} \quad \text{(在 $H_0$ 下)}
\]

\textbf{拒絕域}(雙尾,$\alpha$):$F < F_{1-\alpha/2, n_1-1, n_2-1}$ 或 $F > F_{\alpha/2, n_1-1, n_2-1}$

利用 $F_{1-\alpha/2, k_1, k_2} = \dfrac{1}{F_{\alpha/2, k_2, k_1}}$ 簡化查表。
\end{theorem}

\begin{note}[為何使用 $F$ 分配?]
由 $F$ 分配定義:若 $U \sim \chi^2_{k_1}$ 與 $V \sim \chi^2_{k_2}$ 獨立,則 $(U/k_1)/(V/k_2) \sim F_{k_1, k_2}$。

因為 $(n_1-1)S_1^2/\sigma_1^2 \sim \chi^2_{n_1-1}$ 且 $(n_2-1)S_2^2/\sigma_2^2 \sim \chi^2_{n_2-1}$ 獨立,在 $H_0: \sigma_1^2 = \sigma_2^2$ 下:
\[
\frac{S_1^2/\sigma_1^2}{S_2^2/\sigma_2^2} = \frac{S_1^2}{S_2^2} \sim F_{n_1-1, n_2-1}
\]
\end{note}

\begin{example}
兩組獨立常態樣本:$n_1 = 10$,$s_1^2 = 25$;$n_2 = 8$,$s_2^2 = 10$。在 $\alpha = 0.10$ 下,檢定兩母體變異數是否相等。
\end{example}

\begin{solution}
\textbf{假設}:$H_0: \sigma_1^2 = \sigma_2^2$ vs $H_1: \sigma_1^2 \neq \sigma_2^2$

\textbf{檢定統計量}:
\[
F = \frac{s_1^2}{s_2^2} = \frac{25}{10} = 2.5
\]

\textbf{臨界值}:$df_1 = 9$,$df_2 = 7$

$F_{0.05, 9, 7} = 3.68$(右尾)

$F_{0.95, 9, 7} = 1/F_{0.05, 7, 9} = 1/3.29 = 0.304$(左尾)

\textbf{拒絕域}:$F < 0.304$ 或 $F > 3.68$

\textbf{結論}:$F = 2.5$ 不在拒絕域內,不拒絕 $H_0$。

在 $\alpha = 0.10$ 下,沒有足夠證據顯示兩母體變異數不相等。
\end{solution}

%=============================================================================
\section{卡方獨立性檢定}
%=============================================================================

\subsection{列聯表與獨立性}

\begin{definition}[列聯表]
\textbf{列聯表}(contingency table)用於呈現兩個類別變數的交叉分類次數。

設有 $r$ 個列類別和 $c$ 個行類別,$O_{ij}$ 表示第 $i$ 列第 $j$ 行的觀察次數。
\end{definition}

\begin{definition}[獨立性]
若兩類別變數獨立,則聯合機率等於邊際機率的乘積:
\[
\prb(X = i, Y = j) = \prb(X = i) \cdot \prb(Y = j)
\]
\end{definition}

\subsection{卡方獨立性檢定}

\begin{theorem}[卡方獨立性檢定]
檢定 $H_0$:兩變數獨立 vs $H_1$:兩變數不獨立。

\textbf{期望次數}:在 $H_0$ 下,
\[
E_{ij} = \frac{(\text{第 } i \text{ 列總和}) \times (\text{第 } j \text{ 行總和})}{n} = \frac{R_i \times C_j}{n}
\]

\textbf{檢定統計量}:
\[
\chi^2 = \sum_{i=1}^{r} \sum_{j=1}^{c} \frac{(O_{ij} - E_{ij})^2}{E_{ij}} \stackrel{\text{approx}}{\sim} \chi^2_{(r-1)(c-1)} \quad \text{(在 $H_0$ 下)}
\]

\textbf{拒絕域}:$\chi^2 > \chi^2_{\alpha, (r-1)(c-1)}$
\end{theorem}

\begin{prf}[證明概要]
在 $H_0$(獨立)下,$E_{ij} = n \cdot p_{i\cdot} \cdot p_{\cdot j}$,其中 $p_{i\cdot}$ 和 $p_{\cdot j}$ 是邊際機率。

估計這些機率後(用 $R_i/n$ 和 $C_j/n$),自由度減少為 $(r-1)(c-1)$:
\begin{itemize}
    \item 原有 $rc$ 個格子
    \item 估計 $(r-1)$ 個列邊際機率和 $(c-1)$ 個行邊際機率
    \item 加上總和 = 1 的限制
    \item $df = rc - 1 - (r-1) - (c-1) = (r-1)(c-1)$
\end{itemize}
\end{prf}

\begin{remark}
卡方檢定的適用條件:所有期望次數 $E_{ij} \geqslant 5$。若不滿足,可合併類別或使用 Fisher 精確檢定。
\end{remark}

\begin{example}
某公司調查員工工作級別與年終獎金等級的關係:

\begin{center}
\begin{tabular}{c|ccc|c}
\toprule
& 低獎金 & 中獎金 & 高獎金 & 列總和 \\
\midrule
初級 & 30 & 40 & 10 & 80 \\
中級 & 20 & 50 & 30 & 100 \\
高級 & 10 & 30 & 80 & 120 \\
\midrule
行總和 & 60 & 120 & 120 & 300 \\
\bottomrule
\end{tabular}
\end{center}

在 $\alpha = 0.05$ 下,檢定工作級別與年終獎金是否獨立。
\end{example}

\begin{solution}
\textbf{假設}:$H_0$:獨立 vs $H_1$:不獨立

\textbf{計算期望次數}:$E_{ij} = R_i \times C_j / 300$

\begin{center}
\begin{tabular}{c|ccc}
\toprule
& 低獎金 & 中獎金 & 高獎金 \\
\midrule
初級 & $80 \times 60/300 = 16$ & $80 \times 120/300 = 32$ & $80 \times 120/300 = 32$ \\
中級 & $100 \times 60/300 = 20$ & $100 \times 120/300 = 40$ & $100 \times 120/300 = 40$ \\
高級 & $120 \times 60/300 = 24$ & $120 \times 120/300 = 48$ & $120 \times 120/300 = 48$ \\
\bottomrule
\end{tabular}
\end{center}

\textbf{檢定統計量}:
\begin{align*}
\chi^2 &= \frac{(30-16)^2}{16} + \frac{(40-32)^2}{32} + \frac{(10-32)^2}{32}\\
&\quad + \frac{(20-20)^2}{20} + \frac{(50-40)^2}{40} + \frac{(30-40)^2}{40}\\
&\quad + \frac{(10-24)^2}{24} + \frac{(30-48)^2}{48} + \frac{(80-48)^2}{48}\\
&= \frac{196}{16} + \frac{64}{32} + \frac{484}{32} + 0 + \frac{100}{40} + \frac{100}{40}\\
&\quad + \frac{196}{24} + \frac{324}{48} + \frac{1024}{48}\\
&= 12.25 + 2 + 15.125 + 0 + 2.5 + 2.5 + 8.17 + 6.75 + 21.33\\
&= 70.625
\end{align*}

\textbf{臨界值}:$df = (3-1)(3-1) = 4$,$\chi^2_{0.05, 4} = 9.488$

\textbf{結論}:$\chi^2 = 70.625 > 9.488$,強烈拒絕 $H_0$。

有非常強的證據顯示工作級別與年終獎金不獨立(即有關聯)。
\end{solution}

%=============================================================================
\section{信賴區間與假設檢定的關係}
%=============================================================================

\begin{theorem}[雙尾檢定與信賴區間的等價性]
對於雙尾檢定 $H_0: \theta = \theta_0$ vs $H_1: \theta \neq \theta_0$:

在顯著水準 $\alpha$ 下拒絕 $H_0$ $\ifff$ $\theta_0$ 不在 $(1-\alpha)$ 信賴區間內
\end{theorem}

\begin{prf}
以 $\mu$ 的 $t$ 檢定為例。

$(1-\alpha)$ 信賴區間為 $\bar{X} \pm t_{\alpha/2, n-1} \cdot S/\sqrt{n}$

$\theta_0$ 不在區間內 $\ifff$ $|\bar{X} - \theta_0| > t_{\alpha/2, n-1} \cdot S/\sqrt{n}$

$\ifff$ $\left|\dfrac{\bar{X} - \theta_0}{S/\sqrt{n}}\right| > t_{\alpha/2, n-1}$

$\ifff$ 拒絕 $H_0$
\end{prf}

\begin{example}
例 4.1 中,95\% 信賴區間為 $485 \pm 2.064 \times 6 = (472.62, 497.38)$。

$\mu_0 = 500$ 不在此區間內,故在 $\alpha = 0.05$ 下拒絕 $H_0: \mu = 500$。

但本題是左尾檢定,所以需要計算單側信賴區間或直接用 p 值判斷。
\end{example}

%=============================================================================
\section{本章練習題}
%=============================================================================

\begin{exercise}
設 $X_1, \ldots, X_{36} \simark N(\mu, 36)$(即 $\sigma = 6$)。檢定 $H_0: \mu = 20$ vs $H_1: \mu < 20$,$\alpha = 0.05$。
\begin{enumerate}[label=(\alph*)]
    \item 求拒絕域
    \item 若真實 $\mu = 18$,求型二錯誤機率
    \item 若要使檢定力達到 0.80(當 $\mu = 18$),需要多大樣本?
\end{enumerate}
\end{exercise}

\begin{solution}
\begin{enumerate}[label=(\alph*)]
    \item 檢定統計量:$Z = \dfrac{\bar{X} - 20}{6/\sqrt{36}} = \dfrac{\bar{X} - 20}{1}$
    
    左尾檢定,$z_{0.05} = 1.645$
    
    拒絕域:$Z < -1.645$,即 $\bar{X} < 20 - 1.645 = 18.355$
    
    \item 當 $\mu = 18$ 時,$\bar{X} \sim N(18, 1)$
    
    $\beta = \prb(\bar{X} \geqslant 18.355 \mid \mu = 18) = \prb(Z \geqslant 0.355) = 1 - 0.639 = 0.361$
    
    \item 設樣本量 $n$,標準誤 $= 6/\sqrt{n}$
    
    臨界值:$c = 20 - 1.645 \times 6/\sqrt{n}$
    
    檢定力 $= \prb(\bar{X} < c \mid \mu = 18) = 0.80$
    
    $P\left(Z < \dfrac{c - 18}{6/\sqrt{n}}\right) = 0.80$
    
    $\dfrac{c - 18}{6/\sqrt{n}} = z_{0.20} = 0.842$
    
    代入 $c$:
    $\dfrac{20 - 1.645 \times 6/\sqrt{n} - 18}{6/\sqrt{n}} = 0.842$
    
    $\dfrac{2 - 9.87/\sqrt{n}}{6/\sqrt{n}} = 0.842$
    
    $\dfrac{2\sqrt{n} - 9.87}{6} = 0.842$
    
    $2\sqrt{n} = 5.052 + 9.87 = 14.922$
    
    $\sqrt{n} = 7.46$,$n = 55.7$
    
    需要至少 56 個樣本。
\end{enumerate}
\end{solution}

\begin{exercise}
某研究比較兩種藥物的效果。若採用配對設計(每位病人同時接受兩種藥物),樣本量為 $n$;若採用獨立設計,兩組各 $n$ 人。假設兩設計的母體標準差相同。
\begin{enumerate}[label=(\alph*)]
    \item 配對設計的自由度為何?
    \item 獨立設計的自由度為何?
    \item 哪種設計通常有較高的檢定力?為什麼?
\end{enumerate}
\end{exercise}

\begin{solution}
\begin{enumerate}[label=(\alph*)]
    \item 配對設計:$df = n - 1$
    
    \item 獨立設計:$df = 2n - 2$
    
    \item \textbf{配對設計通常有較高檢定力},原因:
    \begin{itemize}
        \item 配對設計消除了個體間差異的變異
        \item 差異 $D_i = X_i - Y_i$ 的變異數通常小於 $\var(\bar{X} - \bar{Y})$
        \item 雖然自由度較小,但標準誤的減少通常更顯著
    \end{itemize}
    
    定量比較:若 $\corr(X, Y) = \rho$,則
    \[
    \var(D) = \sigma_X^2 + \sigma_Y^2 - 2\rho\sigma_X\sigma_Y
    \]
    
    當 $\rho > 0$ 時,$\var(D) < \sigma_X^2 + \sigma_Y^2$,配對設計更有效。
\end{enumerate}
\end{solution}

\begin{exercise}
某製造商宣稱產品不良率不超過 5\%。檢驗 200 件產品,發現 15 件不良品。在 $\alpha = 0.05$ 下,是否有足夠證據拒絕製造商的宣稱?
\end{exercise}

\begin{solution}
\textbf{假設}:$H_0: p \leqslant 0.05$ vs $H_1: p > 0.05$

\textbf{樣本比例}:$\hat{p} = 15/200 = 0.075$

\textbf{檢定統計量}:
\[
z = \frac{0.075 - 0.05}{\sqrt{\frac{0.05 \times 0.95}{200}}} = \frac{0.025}{\sqrt{0.0002375}} = \frac{0.025}{0.0154} = 1.62
\]

\textbf{臨界值}:$z_{0.05} = 1.645$

\textbf{結論}:$z = 1.62 < 1.645$,不拒絕 $H_0$。

p-value $= \prb(Z > 1.62) = 0.0526 > 0.05$

在 $\alpha = 0.05$ 下,沒有足夠證據拒絕製造商的宣稱(但結果接近顯著邊界)。
\end{solution}

\begin{exercise}
擲一枚骰子 120 次,各點數出現次數如下:

\begin{center}
\begin{tabular}{c|cccccc}
點數 & 1 & 2 & 3 & 4 & 5 & 6 \\
\midrule
次數 & 25 & 17 & 15 & 23 & 24 & 16 \\
\end{tabular}
\end{center}

在 $\alpha = 0.05$ 下,檢定骰子是否公正。
\end{exercise}

\begin{solution}
\textbf{假設}:$H_0$:骰子公正(各點機率 = 1/6)vs $H_1$:骰子不公正

\textbf{期望次數}:$E_i = 120 \times 1/6 = 20$(每個點數)

\textbf{檢定統計量}:
\begin{align*}
\chi^2 &= \sum_{i=1}^{6} \frac{(O_i - E_i)^2}{E_i}\\
&= \frac{(25-20)^2}{20} + \frac{(17-20)^2}{20} + \frac{(15-20)^2}{20}\\
&\quad + \frac{(23-20)^2}{20} + \frac{(24-20)^2}{20} + \frac{(16-20)^2}{20}\\
&= \frac{25 + 9 + 25 + 9 + 16 + 16}{20} = \frac{100}{20} = 5
\end{align*}

\textbf{臨界值}:$df = 6 - 1 = 5$,$\chi^2_{0.05, 5} = 11.07$

\textbf{結論}:$\chi^2 = 5 < 11.07$,不拒絕 $H_0$。

沒有足夠證據顯示骰子不公正。
\end{solution}

\begin{exercise}
設母體服從 $N(\mu, 25)$。以 $n = 100$ 的樣本檢定 $H_0: \mu = 10$ vs $H_1: \mu \neq 10$,$\alpha = 0.05$。求當真實 $\mu = 11$ 時的檢定力。
\end{exercise}

\begin{solution}
\textbf{拒絕域}:$|Z| > 1.96$,其中 $Z = \dfrac{\bar{X} - 10}{5/\sqrt{100}} = \dfrac{\bar{X} - 10}{0.5}$

拒絕 $H_0$ 當 $\bar{X} < 10 - 1.96 \times 0.5 = 9.02$ 或 $\bar{X} > 10 + 1.96 \times 0.5 = 10.98$

\textbf{當 $\mu = 11$ 時},$\bar{X} \sim N(11, 0.25)$

\begin{align*}
\text{Power} &= \prb(\bar{X} < 9.02 \mid \mu = 11) + \prb(\bar{X} > 10.98 \mid \mu = 11)\\
&= P\left(Z < \frac{9.02 - 11}{0.5}\right) + P\left(Z > \frac{10.98 - 11}{0.5}\right)\\
&= \prb(Z < -3.96) + \prb(Z > -0.04)\\
&\approx 0 + 0.516 = 0.516
\end{align*}

當 $\mu = 11$ 時,檢定力約為 51.6\%。

(若只考慮右尾,$\text{Power} \approx \prb(Z > -0.04) \approx 0.516$)
\end{solution}

%=============================================================================
%% \section*{本章重點整理}
%=============================================================================

% \begin{enumerate}
%     \item \textbf{假設檢定基本概念}:
%     \begin{itemize}
%         \item $H_0$(虛無假設)vs $H_1$(對立假設)
%         \item 型一錯誤 $\alpha$:$H_0$ 真卻拒絕;型二錯誤 $\beta$:$H_0$ 假卻不拒絕
%         \item 檢定力 $= 1 - \beta$
%     \end{itemize}
    
%     \item \textbf{p 值}:
%     \begin{itemize}
%         \item 在 $H_0$ 下觀察到更極端結果的機率
%         \item p-value $< \alpha$ $\Rightarrow$ 拒絕 $H_0$
%         \item p 值\textbf{不是} $H_0$ 為真的機率
%     \end{itemize}
    
%     \item \textbf{單樣本檢定}:
%     \begin{itemize}
%         \item $\sigma$ 已知:$Z = \dfrac{\bar{X} - \mu_0}{\sigma/\sqrt{n}} \sim N(0, 1)$
%         \item $\sigma$ 未知:$T = \dfrac{\bar{X} - \mu_0}{S/\sqrt{n}} \sim t_{n-1}$
%     \end{itemize}
    
%     \item \textbf{雙樣本檢定}:
%     \begin{itemize}
%         \item 獨立樣本:使用合併變異數 $S_p^2$,$df = n_1 + n_2 - 2$
%         \item 配對樣本:對差異 $D_i$ 做單樣本 $t$ 檢定,$df = n - 1$
%     \end{itemize}
    
%     \item \textbf{比例檢定}:$Z = \dfrac{\hat{p} - p_0}{\sqrt{p_0(1-p_0)/n}}$
    
%     \item \textbf{變異數檢定}:
%     \begin{itemize}
%         \item 單一:$\chi^2 = \dfrac{(n-1)S^2}{\sigma_0^2} \sim \chi^2_{n-1}$
%         \item 兩樣本:$F = S_1^2/S_2^2 \sim F_{n_1-1, n_2-1}$
%     \end{itemize}
    
%     \item \textbf{卡方獨立性檢定}:
%     \begin{itemize}
%         \item $\chi^2 = \sum \dfrac{(O_{ij} - E_{ij})^2}{E_{ij}}$
%         \item $E_{ij} = \dfrac{R_i \times C_j}{n}$
%         \item $df = (r-1)(c-1)$
%     \end{itemize}
    
%     \item \textbf{型二錯誤計算}(考古題重點):
%     \begin{enumerate}
%         \item 在 $H_0$ 下找不拒絕域
%         \item 在 $H_1$(真實值)下計算落在不拒絕域的機率
%     \end{enumerate}
% \end{enumerate}

\end{document}
