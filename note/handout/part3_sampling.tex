\documentclass[12pt,a4paper]{article}
\usepackage[left=1cm,right=1cm,bottom=15mm,top=20mm]{geometry}
\usepackage[AutoFakeBold,AutoFakeSlant]{xeCJK}
\setCJKmainfont[AutoFakeSlant=.1,AutoFakeBold=2]{Noto Serif CJK TC}
\usepackage{amsmath,amsthm,amssymb,amsfonts}
\usepackage{graphicx,xcolor,float}
\usepackage{booktabs,tabularx,multirow,array}
\usepackage{enumitem}
\usepackage{parskip}
\setlist{itemsep=0pt,parsep=0pt}
\usepackage{hyperref}
\hypersetup{
    colorlinks=true,
    linkcolor=blue!70!black,
    urlcolor=blue!80!black
}

% 定理環境
\theoremstyle{definition}
\newtheorem{definition}{定義}[section]
\newtheorem{example}{例題}[section]
\newtheorem{exercise}{習題}[section]
\newtheorem{theorem}{定理}[section]
\newtheorem{lemma}{引理}[section]
\newtheorem{corollary}{推論}[section]
\newtheorem{proposition}{命題}[section]
\newtheorem{property}{性質}[section]
\newtheorem*{remark}{註}
\newtheorem*{solution}{解答}
\newtheorem*{note}{說明}
\newtheorem*{prf}{證明}

% 常用指令
\newcommand{\ds}{\displaystyle}
\newcommand{\ie}{\;\Longrightarrow\;}
\newcommand{\ifff}{\;\Longleftrightarrow\;}
\newcommand\expc{\mathsf{E}}
\DeclareMathOperator\var{var}
\DeclareMathOperator\cov{cov}
\DeclareMathOperator\corr{corr}
\newcommand{\SE}{\mathrm{SE}}
\newcommand\prb{\mathsf{P}}
\newcommand{\Real}{\mathbb{R}}
\newcommand{\Nat}{\mathbb{N}}
\newcommand{\diff}[2]{\frac{\mathrm{d} #1}{\mathrm{d} #2}}
\newcommand{\pdiff}[2]{\frac{\partial #1}{\partial #2}}
\newcommand{\simark}{\stackrel{\text{i.i.d.}}{\sim}}

% 頁面設定
\renewcommand{\figurename}{圖}
\renewcommand{\tablename}{表}

\usepackage{fancyhdr}
\pagestyle{fancy}
\fancyhf{}
\fancyhead[L]{統計學講義}
\fancyhead[R]{第三部分:抽樣分配}
\fancyfoot[C]{\thepage}
\renewcommand{\headrulewidth}{0.4pt}
\renewcommand{\footrulewidth}{0.4pt}

\title{\vspace{-2cm}\textbf{統計學講義}\\[3mm] \Large 第三部分:抽樣分配}
\author{}
\date{\vspace{-2cm}}

\begin{document}
\maketitle
\thispagestyle{fancy}

\begin{center}
\fbox{\parbox{0.9\textwidth}{\centering
\textbf{參考書籍}\\[2mm]
Jeffrey S. Rosenthal, \textit{Probability and Statistics: The Science of Uncertainty}, 2nd Edition\\
Chapter 6: Sampling Distributions\\[2mm]
\url{https://utstat.utoronto.ca/mikevans/jeffrosenthal/}
}}
\end{center}

%\tableofcontents

%=============================================================================
\section{母體、樣本與統計量}
%=============================================================================

\subsection{母體與樣本}

\begin{definition}[母體與樣本]
\begin{itemize}
  \item[]
    \item \textbf{母體}(population):研究對象的全體,通常以機率分配描述。
    \item \textbf{樣本}(sample):從母體中抽取的部分個體。
    \item \textbf{母體參數}(population parameter):描述母體特徵的數值,如母體平均數 $\mu$、母體變異數 $\sigma^2$。
    \item \textbf{樣本統計量}(sample statistic):由樣本計算出的數值,如樣本平均數 $\bar{X}$、樣本變異數 $S^2$。
\end{itemize}
\end{definition}

\begin{definition}[隨機樣本]
若 $X_1, X_2, \ldots, X_n$ 為從母體中抽取的 $n$ 個觀測值,且滿足:
\begin{enumerate}[label=(\roman*)]
    \item 每個 $X_i$ 與母體有相同的分配
    \item $X_1, X_2, \ldots, X_n$ 相互獨立
\end{enumerate}
則稱 $X_1, X_2, \ldots, X_n$ 為一組大小為 $n$ 的\textbf{隨機樣本}(random sample),記為
\[
X_1, X_2, \ldots, X_n \simark F
\]
其中 $F$ 為母體分配,i.i.d. 表示「獨立且同分配」(independent and identically distributed)。
\end{definition}

\subsection{常用樣本統計量}

\begin{definition}[樣本平均數]
設 $X_1, X_2, \ldots, X_n$ 為隨機樣本,則\textbf{樣本平均數}(sample mean)定義為:
\[
\bar{X} = \frac{1}{n} \sum_{i=1}^{n} X_i = \frac{X_1 + X_2 + \cdots + X_n}{n}
\]
\end{definition}

\begin{definition}[樣本變異數與樣本標準差]
\textbf{樣本變異數}(sample variance)定義為:
\[
S^2 = \frac{1}{n-1} \sum_{i=1}^{n} (X_i - \bar{X})^2
\]

\textbf{樣本標準差}(sample standard deviation)定義為:
\[
S = \sqrt{S^2} = \sqrt{\frac{1}{n-1} \sum_{i=1}^{n} (X_i - \bar{X})^2}
\]
\end{definition}

\begin{remark}
樣本變異數的分母為 $n-1$ 而非 $n$,這是為了使 $S^2$ 成為 $\sigma^2$ 的\textbf{不偏估計量}(unbiased estimator)。$n-1$ 稱為\textbf{自由度}(degrees of freedom)。
\end{remark}

\begin{property}[樣本變異數的計算公式]
\[
S^2 = \frac{1}{n-1} \left( \sum_{i=1}^{n} X_i^2 - n\bar{X}^2 \right) = \frac{1}{n-1} \left( \sum_{i=1}^{n} X_i^2 - \frac{\left(\sum_{i=1}^{n} X_i\right)^2}{n} \right)
\]
\end{property}

\subsection{統計量的抽樣分配}

\begin{definition}[抽樣分配]
\textbf{抽樣分配}(sampling distribution)是指統計量在所有可能樣本下的機率分配。

由於樣本是隨機的,由樣本計算的統計量(如 $\bar{X}$、$S^2$)也是隨機變數,有其自己的分配。
\end{definition}

\begin{theorem}[樣本平均數的期望值與變異數]\label{thm:xbar-properties}
設 $X_1, X_2, \ldots, X_n$ 為來自母體($\expc[X_i] = \mu$,$\var(X_i) = \sigma^2$)的隨機樣本。則:
\begin{enumerate}[label=(\roman*)]
    \item $\expc[\bar{X}] = \mu$($\bar{X}$ 是 $\mu$ 的不偏估計量)
    \item $\var(\bar{X}) = \dfrac{\sigma^2}{n}$
    \item $\SE(\bar{X}) = \dfrac{\sigma}{\sqrt{n}}$(標準誤)
\end{enumerate}
\end{theorem}

\begin{prf}
\textbf{(i) 期望值:}
\[
\expc[\bar{X}] = \expc\left[\frac{1}{n} \sum_{i=1}^{n} X_i\right] = \frac{1}{n} \sum_{i=1}^{n} \expc[X_i] = \frac{1}{n} \cdot n\mu = \mu
\]

\textbf{(ii) 變異數:}

由於 $X_1, \ldots, X_n$ 獨立:
\[
\var(\bar{X}) = \var\left(\frac{1}{n} \sum_{i=1}^{n} X_i\right) = \frac{1}{n^2} \sum_{i=1}^{n} \var(X_i) = \frac{1}{n^2} \cdot n\sigma^2 = \frac{\sigma^2}{n}
\]
\end{prf}

\begin{definition}[標準誤]
統計量的\textbf{標準誤}(standard error, SE)是該統計量抽樣分配的標準差。

對於樣本平均數:
\[
\SE(\bar{X}) = \sqrt{\var(\bar{X})} = \frac{\sigma}{\sqrt{n}}
\]

若 $\sigma$ 未知,以 $S$ 估計:
\[
\widehat{\SE}(\bar{X}) = \frac{S}{\sqrt{n}}
\]
\end{definition}

\begin{theorem}[樣本變異數的期望值]
設 $X_1, X_2, \ldots, X_n$ 為來自母體(變異數為 $\sigma^2$)的隨機樣本。則:
\[
\expc[S^2] = \sigma^2
\]
即 $S^2$ 是 $\sigma^2$ 的不偏估計量。
\end{theorem}

\begin{prf}
\begin{align*}
\sum_{i=1}^{n}(X_i - \bar{X})^2 &= \sum_{i=1}^{n}[(X_i - \mu) - (\bar{X} - \mu)]^2\\
&= \sum_{i=1}^{n}(X_i - \mu)^2 - 2(\bar{X} - \mu)\sum_{i=1}^{n}(X_i - \mu) + n(\bar{X} - \mu)^2
\end{align*}

注意 $\sum_{i=1}^{n}(X_i - \mu) = n(\bar{X} - \mu)$,故:
\[
\sum_{i=1}^{n}(X_i - \bar{X})^2 = \sum_{i=1}^{n}(X_i - \mu)^2 - n(\bar{X} - \mu)^2
\]

取期望值:
\[
\expc\left[\sum_{i=1}^{n}(X_i - \bar{X})^2\right] = \sum_{i=1}^{n}\expc[(X_i - \mu)^2] - n\expc[(\bar{X} - \mu)^2] = n\sigma^2 - n \cdot \frac{\sigma^2}{n} = (n-1)\sigma^2
\]

因此 $\expc[S^2] = \dfrac{1}{n-1} \cdot (n-1)\sigma^2 = \sigma^2$。
\end{prf}

%=============================================================================
\section{中央極限定理}
%=============================================================================

\subsection{定理陳述}

\begin{theorem}[中央極限定理 (Central Limit Theorem, CLT)]\label{thm:CLT}
設 $X_1, X_2, \ldots, X_n$ 為 i.i.d. 隨機變數,$\expc[X_i] = \mu$,$\var(X_i) = \sigma^2 < \infty$。則當 $n \to \infty$ 時:
\[
\frac{\bar{X} - \mu}{\sigma/\sqrt{n}} \stackrel{d}{\longrightarrow} N(0, 1)
\]

或等價地:
\[
\frac{\sum_{i=1}^{n} X_i - n\mu}{\sigma\sqrt{n}} \stackrel{d}{\longrightarrow} N(0, 1)
\]
\end{theorem}

\begin{remark}
中央極限定理的重要性:
\begin{itemize}
    \item \textbf{不需假設母體為常態分配}——無論母體是什麼分配,只要變異數有限,樣本平均數近似常態。
    \item 這解釋了為什麼常態分配在統計學中如此重要。
    \item 經驗法則:當 $n \geqslant 30$ 時,近似通常足夠好。若母體本身接近常態,較小的 $n$ 也可接受。
\end{itemize}
\end{remark}

\begin{corollary}[CLT 的實用形式]
當 $n$ 夠大時:
\[
\bar{X} \stackrel{\text{approx}}{\sim} N\left(\mu, \frac{\sigma^2}{n}\right)
\]

或等價地:
\[
Z = \frac{\bar{X} - \mu}{\sigma/\sqrt{n}} \stackrel{\text{approx}}{\sim} N(0, 1)
\]
\end{corollary}

\subsection{連續性校正}

\begin{definition}[連續性校正]
當使用連續型分配(如常態分配)近似離散型分配(如二項分配)時,由於離散變數只取整數值,而連續變數可取任意實數值,需要進行\textbf{連續性校正}(continuity correction)以提高近似精度。

具體做法:將離散變數的整數值 $k$ 對應到連續變數的區間 $(k - 0.5, k + 0.5)$。
\end{definition}

\begin{property}[連續性校正規則]
設 $X$ 為離散型隨機變數,以連續型隨機變數 $Y$ 近似:
\begin{enumerate}[label=(\roman*)]
    \item $\prb(X = k) \approx P(k - 0.5 < Y < k + 0.5)$
    \item $\prb(X \leqslant k) \approx \prb(Y < k + 0.5)$
    \item $\prb(X < k) \approx \prb(Y < k - 0.5)$
    \item $\prb(X \geqslant k) \approx \prb(Y > k - 0.5)$
    \item $\prb(X > k) \approx \prb(Y > k + 0.5)$
    \item $\prb(a \leqslant X \leqslant b) \approx \prb(a - 0.5 < Y < b + 0.5)$
\end{enumerate}
\end{property}

\begin{remark}
連續性校正的直觀理解:整數 $k$ 在數線上是一個點,但在連續近似中,我們將它「擴展」為寬度為 1 的區間 $(k-0.5, k+0.5)$,使得該區間的機率可以更準確地近似離散點的機率。
\end{remark}

\subsection{中央極限定理的應用}

\begin{example}
某地區人口的 IQ 分數平均為 100,標準差為 15。隨機抽取 36 人。
\begin{enumerate}[label=(\alph*)]
    \item 求樣本平均 IQ 超過 105 的機率
    \item 求樣本平均 IQ 介於 97 與 103 之間的機率
\end{enumerate}
\end{example}

\begin{solution}
設 $X_i$ = 第 $i$ 個人的 IQ,$\mu = 100$,$\sigma = 15$,$n = 36$。

由中央極限定理,$\bar{X} \stackrel{\text{approx}}{\sim} N\left(100, \dfrac{15^2}{36}\right) = N(100, 6.25)$

標準誤:$\SE(\bar{X}) = \dfrac{15}{\sqrt{36}} = 2.5$

\begin{enumerate}[label=(\alph*)]
    \item 
    \[
    \prb(\bar{X} > 105) = P\left(Z > \frac{105 - 100}{2.5}\right) = \prb(Z > 2) = 1 - \Phi(2) \approx 1 - 0.9772 = 0.0228
    \]
    
    \item 
    \begin{align*}
    \prb(97 < \bar{X} < 103) &= P\left(\frac{97-100}{2.5} < Z < \frac{103-100}{2.5}\right)\\
    &= \prb(-1.2 < Z < 1.2)\\
    &= \Phi(1.2) - \Phi(-1.2)\\
    &= 2\Phi(1.2) - 1 \approx 2(0.8849) - 1 = 0.7698
    \end{align*}
\end{enumerate}
\end{solution}

\begin{example}
擲一枚公正硬幣 100 次,求正面次數介於 45 到 55 之間的機率。
\end{example}

\begin{solution}
設 $X$ = 正面次數,$X \sim B(100, 0.5)$。

$\mu = np = 50$,$\sigma = \sqrt{np(1-p)} = \sqrt{25} = 5$。

檢驗近似條件:$np = 50 > 5$,$n(1-p) = 50 > 5$ \checkmark

\textbf{使用連續性校正}:由於我們要計算 $\prb(45 \leqslant X \leqslant 55)$,根據規則 (vi),應近似為 $\prb(44.5 < Y < 55.5)$:
\begin{align*}
\prb(45 \leqslant X \leqslant 55) &\approx \prb(44.5 < Y < 55.5) \quad \text{(連續性校正)}\\
&= P\left(\frac{44.5 - 50}{5} < Z < \frac{55.5 - 50}{5}\right)\\
&= \prb(-1.1 < Z < 1.1)\\
&= 2\Phi(1.1) - 1 \approx 2(0.8643) - 1 = 0.7286
\end{align*}

\textbf{比較}:若不使用連續性校正,直接計算 $\prb(-1 < Z < 1) = 0.6826$,誤差較大。
\end{solution}

%=============================================================================
\section{隨機變數變換與 Jacobian}
%=============================================================================

本節介紹推導抽樣分配所需的數學工具。

\subsection{單變數變換}

\begin{theorem}[單變數變換公式]\label{thm:univariate-transform}
設連續型隨機變數 $X$ 的 PDF 為 $f_X(x)$。令 $Y = g(X)$,其中 $g$ 為嚴格單調函數,反函數為 $X = g^{-1}(Y) = h(Y)$。則 $Y$ 的 PDF 為:
\[
f_Y(y) = f_X(h(y)) \cdot \left| h'(y) \right| = f_X(h(y)) \cdot \left| \frac{dx}{dy} \right|
\]
\end{theorem}

\begin{prf}
設 $g$ 為嚴格遞增函數(遞減情況類似)。則:
\[
F_Y(y) = \prb(Y \leqslant y) = P(g(X) \leqslant y) = \prb(X \leqslant g^{-1}(y)) = F_X(h(y))
\]

對 $y$ 微分:
\[
f_Y(y) = f_X(h(y)) \cdot h'(y)
\]

若 $g$ 為遞減函數,則 $F_Y(y) = \prb(X \geqslant h(y)) = 1 - F_X(h(y))$,微分後 $f_Y(y) = f_X(h(y)) \cdot (-h'(y))$。

綜合兩種情況,$f_Y(y) = f_X(h(y)) \cdot |h'(y)|$。
\end{prf}

\subsection{多變數變換與 Jacobian}

\begin{theorem}[雙變數變換公式]\label{thm:bivariate-transform}
設 $(X, Y)$ 的聯合 PDF 為 $f_{X,Y}(x, y)$。令
\[
U = g_1(X, Y), \quad V = g_2(X, Y)
\]
為一對一變換,反變換為
\[
X = h_1(U, V), \quad Y = h_2(U, V)
\]

則 $(U, V)$ 的聯合 PDF 為:
\[
f_{U,V}(u, v) = f_{X,Y}(h_1(u,v), h_2(u,v)) \cdot |J|
\]

其中 $J$ 為 \textbf{Jacobian 行列式}:
\[
J = \frac{\partial(x, y)}{\partial(u, v)} = \begin{vmatrix} \dfrac{\partial x}{\partial u} & \dfrac{\partial x}{\partial v} \\[3mm] \dfrac{\partial y}{\partial u} & \dfrac{\partial y}{\partial v} \end{vmatrix} = \frac{\partial x}{\partial u} \cdot \frac{\partial y}{\partial v} - \frac{\partial x}{\partial v} \cdot \frac{\partial y}{\partial u}
\]
\end{theorem}

\begin{remark}
Jacobian 的絕對值 $|J|$ 代表變換的「面積縮放因子」。在單變數情況下,$|J| = |dx/dy|$ 就是定理 \ref{thm:univariate-transform} 中的 $|h'(y)|$。
\end{remark}

%=============================================================================
\section{卡方分配}
%=============================================================================

\subsection{卡方分配的定義與 PDF 推導}

\begin{definition}[卡方分配]
若 $Z_1, Z_2, \ldots, Z_k$ 為 i.i.d. $N(0, 1)$ 隨機變數,則
\[
\chi^2 = Z_1^2 + Z_2^2 + \cdots + Z_k^2 = \sum_{i=1}^{k} Z_i^2
\]
服從自由度為 $k$ 的\textbf{卡方分配}(chi-square distribution),記為 $\chi^2 \sim \chi^2_k$ 或 $\chi^2(k)$。
\end{definition}

\begin{theorem}[$\chi^2_1$ 的 PDF]\label{thm:chi1-pdf}
若 $Z \sim N(0,1)$,令 $Y = Z^2$,則 $Y \sim \chi^2_1$,其 PDF 為:
\[
f_Y(y) = \frac{1}{\sqrt{2\pi}} y^{-1/2} e^{-y/2} = \frac{1}{2^{1/2} \Gamma(1/2)} y^{1/2 - 1} e^{-y/2}, \quad y > 0
\]
\end{theorem}

\begin{prf}
設 $Z \sim N(0,1)$,$f_Z(z) = \dfrac{1}{\sqrt{2\pi}} e^{-z^2/2}$。

令 $Y = Z^2$。對於 $y > 0$,使用 CDF 方法:
\[
F_Y(y) = \prb(Y \leqslant y) = \prb(Z^2 \leqslant y) = P(-\sqrt{y} \leqslant Z \leqslant \sqrt{y}) = \Phi(\sqrt{y}) - \Phi(-\sqrt{y})
\]

由標準常態的對稱性:
\[
F_Y(y) = 2\Phi(\sqrt{y}) - 1
\]

對 $y$ 微分得 PDF:
\[
f_Y(y) = 2 \cdot \phi(\sqrt{y}) \cdot \frac{1}{2\sqrt{y}} = \frac{\phi(\sqrt{y})}{\sqrt{y}}
\]

代入 $\phi(\sqrt{y}) = \dfrac{1}{\sqrt{2\pi}} e^{-y/2}$:
\[
f_Y(y) = \frac{1}{\sqrt{y}} \cdot \frac{1}{\sqrt{2\pi}} e^{-y/2} = \frac{1}{\sqrt{2\pi}} y^{-1/2} e^{-y/2}
\]

利用 $\Gamma(1/2) = \sqrt{\pi}$,可改寫為:
\[
f_Y(y) = \frac{1}{2^{1/2} \Gamma(1/2)} y^{1/2 - 1} e^{-y/2}, \quad y > 0
\]
\end{prf}

\begin{theorem}[一般 $\chi^2_k$ 的 PDF]\label{thm:chik-pdf}
若 $X \sim \chi^2_k$,則其 PDF 為:
\[
f(x) = \frac{1}{2^{k/2} \Gamma(k/2)} x^{k/2 - 1} e^{-x/2}, \quad x > 0
\]

這是參數為 $\alpha = k/2$、$\lambda = 1/2$ 的 \textbf{Gamma 分配}。
\end{theorem}

\begin{prf}[概要]
由卡方分配的可加性(定理 \ref{thm:chi-additive}),$\chi^2_k = \chi^2_1 + \chi^2_1 + \cdots + \chi^2_1$($k$ 個獨立的 $\chi^2_1$)。

獨立隨機變數之和的 PDF 可由卷積得到。由於 $\chi^2_1$ 是 $\text{Gamma}(1/2, 1/2)$ 分配,而 Gamma 分配具有可加性:
\[
\text{Gamma}(\alpha_1, \lambda) + \text{Gamma}(\alpha_2, \lambda) \sim \text{Gamma}(\alpha_1 + \alpha_2, \lambda)
\]

故 $\chi^2_k \sim \text{Gamma}(k/2, 1/2)$,其 PDF 如上所述。

(完整的卷積證明需要較長的計算,此處從略。)
\end{prf}

\subsection{卡方分配的性質}

\begin{lemma}[標準常態的四階動差]\label{lem:Z4}
若 $Z \sim N(0,1)$,則 $\expc[Z^4] = 3$。
\end{lemma}

\begin{prf}
\begin{align*}
\expc[Z^4] &= \int_{-\infty}^{\infty} z^4 \cdot \frac{1}{\sqrt{2\pi}} e^{-z^2/2} \, dz
\end{align*}

使用分部積分。令 $u = z^3$,$dv = z \cdot \dfrac{1}{\sqrt{2\pi}} e^{-z^2/2} dz$。

則 $du = 3z^2 dz$,$v = -\dfrac{1}{\sqrt{2\pi}} e^{-z^2/2}$。

\begin{align*}
\expc[Z^4] &= \left[ -z^3 \cdot \frac{1}{\sqrt{2\pi}} e^{-z^2/2} \right]_{-\infty}^{\infty} + \int_{-\infty}^{\infty} 3z^2 \cdot \frac{1}{\sqrt{2\pi}} e^{-z^2/2} \, dz\\
&= 0 + 3 \expc[Z^2] = 3 \cdot 1 = 3
\end{align*}

(邊界項為 0 是因為 $z^3 e^{-z^2/2} \to 0$ 當 $z \to \pm\infty$。)
\end{prf}

\begin{lemma}[卡方分配的負一次動差]\label{lem:chi-inverse}
若 $V \sim \chi^2_k$ 且 $k > 2$,則:
\[
\expc\left[\frac{1}{V}\right] = \frac{1}{k-2}
\]
\end{lemma}

\begin{prf}
$V$ 的 PDF 為 $f_V(v) = \dfrac{1}{2^{k/2} \Gamma(k/2)} v^{k/2 - 1} e^{-v/2}$,$v > 0$。

\begin{align*}
\expc\left[\frac{1}{V}\right] &= \int_0^{\infty} \frac{1}{v} \cdot \frac{1}{2^{k/2} \Gamma(k/2)} v^{k/2 - 1} e^{-v/2} \, dv\\
&= \frac{1}{2^{k/2} \Gamma(k/2)} \int_0^{\infty} v^{k/2 - 2} e^{-v/2} \, dv
\end{align*}

令 $u = v/2$,則 $v = 2u$,$dv = 2\,du$:
\begin{align*}
\expc\left[\frac{1}{V}\right] &= \frac{1}{2^{k/2} \Gamma(k/2)} \int_0^{\infty} (2u)^{k/2 - 2} e^{-u} \cdot 2 \, du\\
&= \frac{2^{k/2 - 2} \cdot 2}{2^{k/2} \Gamma(k/2)} \int_0^{\infty} u^{k/2 - 2} e^{-u} \, du\\
&= \frac{2^{k/2 - 1}}{2^{k/2} \Gamma(k/2)} \cdot \Gamma\left(\frac{k}{2} - 1\right)\\
&= \frac{1}{2 \Gamma(k/2)} \cdot \Gamma\left(\frac{k}{2} - 1\right)
\end{align*}

利用 Gamma 函數的遞推公式 $\Gamma(k/2) = (k/2 - 1) \Gamma(k/2 - 1)$:
\[
\expc\left[\frac{1}{V}\right] = \frac{\Gamma(k/2 - 1)}{2 \cdot (k/2 - 1) \cdot \Gamma(k/2 - 1)} = \frac{1}{k - 2}
\]

此結果需要 $k/2 - 1 > 0$,即 $k > 2$。
\end{prf}

\begin{theorem}[卡方分配的可加性]\label{thm:chi-additive}
若 $X_1 \sim \chi^2_{k_1}$ 與 $X_2 \sim \chi^2_{k_2}$ 獨立,則:
\[
X_1 + X_2 \sim \chi^2_{k_1 + k_2}
\]
\end{theorem}

\begin{prf}
設 $X_1 = \sum_{i=1}^{k_1} Z_i^2$,$X_2 = \sum_{j=1}^{k_2} W_j^2$,其中所有 $Z_i$、$W_j$ 皆為獨立的 $N(0,1)$。

則:
\[
X_1 + X_2 = \sum_{i=1}^{k_1} Z_i^2 + \sum_{j=1}^{k_2} W_j^2 = \sum_{m=1}^{k_1+k_2} U_m^2 \sim \chi^2_{k_1+k_2}
\]
其中 $U_1, \ldots, U_{k_1+k_2}$ 為 $k_1 + k_2$ 個獨立的 $N(0,1)$ 隨機變數。
\end{prf}

\begin{property}[卡方分配的期望值與變異數]
若 $X \sim \chi^2_k$,則:
\begin{enumerate}[label=(\roman*)]
    \item $\expc[X] = k$
    \item $\var(X) = 2k$
\end{enumerate}
\end{property}

\begin{prf}
設 $Z \sim N(0,1)$,則 $Z^2 \sim \chi^2_1$。

\textbf{(i)} 由 $\var(Z) = \expc[Z^2] - (\expc[Z])^2 = 1$,得 $\expc[Z^2] = 1$。

對於 $\chi^2_k = Z_1^2 + Z_2^2 + \cdots + Z_k^2$($Z_i$ i.i.d. $N(0,1)$):
\[
\expc[\chi^2_k] = \sum_{i=1}^{k} \expc[Z_i^2] = k \cdot 1 = k
\]

\textbf{(ii)} 由引理 \ref{lem:Z4},$\expc[Z^4] = 3$,故:
\[
\var(Z^2) = \expc[Z^4] - (\expc[Z^2])^2 = 3 - 1^2 = 2
\]

由獨立性:
\[
\var(\chi^2_k) = \sum_{i=1}^{k} \var(Z_i^2) = k \cdot 2 = 2k
\]
\end{prf}

\subsection{樣本變異數與卡方分配}

\begin{theorem}[樣本變異數與卡方分配]\label{thm:S2-chisq}
若 $X_1, X_2, \ldots, X_n \simark N(\mu, \sigma^2)$,則:
\[
\frac{(n-1)S^2}{\sigma^2} = \frac{\sum_{i=1}^{n}(X_i - \bar{X})^2}{\sigma^2} \sim \chi^2_{n-1}
\]

且 $\bar{X}$ 與 $S^2$ \textbf{獨立}。
\end{theorem}

\begin{prf}[證明概要]
令 $Z_i = (X_i - \mu)/\sigma \sim N(0,1)$ i.i.d.,$\bar{Z} = (\bar{X} - \mu)/\sigma$。則:
\[
\sum_{i=1}^{n} Z_i^2 = \sum_{i=1}^{n} \frac{(X_i - \mu)^2}{\sigma^2} \sim \chi^2_n
\]

可分解為:
\[
\sum_{i=1}^{n} Z_i^2 = \sum_{i=1}^{n} (Z_i - \bar{Z})^2 + n\bar{Z}^2
\]

(這是因為 $\sum (Z_i - \bar{Z})^2 = \sum Z_i^2 - n\bar{Z}^2$)

由於 $\bar{Z} \sim N(0, 1/n)$,故 $n\bar{Z}^2 = (\sqrt{n}\bar{Z})^2 \sim \chi^2_1$。

\textbf{為何 $\bar{Z}$ 與 $\sum (Z_i - \bar{Z})^2$ 獨立?}

直觀理解:$\bar{Z}$ 代表「樣本平均數離 0 有多遠」,而 $\sum (Z_i - \bar{Z})^2$ 代表「各觀測值圍繞樣本平均數的分散程度」。對於常態分配,樣本的「中心位置」與「分散程度」是獨立的資訊。

嚴格來說,可用正交變換來證明:令 $\mathbf{Z} = (Z_1, \ldots, Z_n)^T$,存在正交矩陣 $P$(第一列為 $(1/\sqrt{n}, \ldots, 1/\sqrt{n})$)使得 $\mathbf{W} = P\mathbf{Z}$ 滿足 $W_1 = \sqrt{n}\bar{Z}$,$W_2, \ldots, W_n$ 為其餘正交分量。由多元常態分配的性質,$W_1, \ldots, W_n$ 獨立且均為 $N(0,1)$,而 $\sum (Z_i - \bar{Z})^2 = \sum_{j=2}^{n} W_j^2$,故與 $W_1^2 = n\bar{Z}^2$ 獨立。

由卡方分配的可加性,$\sum_{j=2}^{n} W_j^2 \sim \chi^2_{n-1}$。

因此:
\[
\frac{(n-1)S^2}{\sigma^2} = \frac{\sum_{i=1}^{n}(X_i - \bar{X})^2}{\sigma^2} = \sum_{i=1}^{n} (Z_i - \bar{Z})^2 \sim \chi^2_{n-1}
\]

且由於 $\bar{X}$ 僅與 $\bar{Z}$ 有關,$S^2$ 僅與 $\sum (Z_i - \bar{Z})^2$ 有關,故 $\bar{X}$ 與 $S^2$ 獨立。
\end{prf}

\begin{remark}
自由度為 $n-1$ 而非 $n$,因為 $(X_i - \bar{X})$ 滿足限制條件 $\sum_{i=1}^{n}(X_i - \bar{X}) = 0$,只有 $n-1$ 個自由的項。從分解 $\chi^2_n = \chi^2_{n-1} + \chi^2_1$ 也可看出。
\end{remark}

%=============================================================================
\section{t 分配}
%=============================================================================

\subsection{t 分配的定義與 PDF 推導}

\begin{definition}[t 分配]
若 $Z \sim N(0, 1)$ 且 $V \sim \chi^2_k$ 獨立,則
\[
T = \frac{Z}{\sqrt{V/k}}
\]
服從自由度為 $k$ 的 \textbf{t 分配}(Student's t-distribution),記為 $T \sim t_k$ 或 $t(k)$。
\end{definition}

\begin{theorem}[t 分配的 PDF]\label{thm:t-pdf}
若 $T \sim t_k$,則其 PDF 為:
\[
f_T(t) = \frac{\Gamma\left(\frac{k+1}{2}\right)}{\sqrt{k\pi} \, \Gamma\left(\frac{k}{2}\right)} \left(1 + \frac{t^2}{k}\right)^{-\frac{k+1}{2}}, \quad -\infty < t < \infty
\]
\end{theorem}

\begin{prf}
設 $Z \sim N(0,1)$,$V \sim \chi^2_k$ 獨立。其聯合 PDF 為:
\[
f_{Z,V}(z, v) = \frac{1}{\sqrt{2\pi}} e^{-z^2/2} \cdot \frac{1}{2^{k/2} \Gamma(k/2)} v^{k/2-1} e^{-v/2}, \quad -\infty < z < \infty, \; v > 0
\]

令變換 $(Z, V) \to (T, V)$:
\[
T = \frac{Z}{\sqrt{V/k}}, \quad V = V
\]

反變換:
\[
Z = T \sqrt{V/k}, \quad V = V
\]

計算 Jacobian:
\[
J = \frac{\partial(z, v)}{\partial(t, v)} = \begin{vmatrix} \dfrac{\partial z}{\partial t} & \dfrac{\partial z}{\partial v} \\[3mm] \dfrac{\partial v}{\partial t} & \dfrac{\partial v}{\partial v} \end{vmatrix} = \begin{vmatrix} \sqrt{v/k} & \dfrac{t}{2\sqrt{kv}} \\[2mm] 0 & 1 \end{vmatrix} = \sqrt{\frac{v}{k}}
\]

$(T, V)$ 的聯合 PDF:
\begin{align*}
f_{T,V}(t, v) &= f_{Z,V}\left(t\sqrt{v/k}, v\right) \cdot |J|\\
&= \frac{1}{\sqrt{2\pi}} e^{-t^2 v/(2k)} \cdot \frac{v^{k/2-1} e^{-v/2}}{2^{k/2} \Gamma(k/2)} \cdot \sqrt{\frac{v}{k}}\\
&= \frac{1}{\sqrt{2\pi k}} \cdot \frac{1}{2^{k/2} \Gamma(k/2)} \cdot v^{k/2 - 1/2} e^{-v(1 + t^2/k)/2}
\end{align*}

$T$ 的邊際 PDF(對 $v$ 積分):
\begin{align*}
f_T(t) &= \int_0^{\infty} f_{T,V}(t, v) \, dv\\
&= \frac{1}{\sqrt{2\pi k} \cdot 2^{k/2} \Gamma(k/2)} \int_0^{\infty} v^{(k+1)/2 - 1} e^{-v(1 + t^2/k)/2} \, dv
\end{align*}

令 $u = v(1 + t^2/k)/2$,則 $v = \dfrac{2u}{1 + t^2/k}$,$dv = \dfrac{2}{1 + t^2/k} du$:
\begin{align*}
f_T(t) &= \frac{1}{\sqrt{2\pi k} \cdot 2^{k/2} \Gamma(k/2)} \cdot \left(\frac{2}{1 + t^2/k}\right)^{(k+1)/2} \int_0^{\infty} u^{(k+1)/2 - 1} e^{-u} \, du\\
&= \frac{1}{\sqrt{2\pi k} \cdot 2^{k/2} \Gamma(k/2)} \cdot \frac{2^{(k+1)/2}}{(1 + t^2/k)^{(k+1)/2}} \cdot \Gamma\left(\frac{k+1}{2}\right)\\
&= \frac{2^{1/2} \cdot \Gamma\left(\frac{k+1}{2}\right)}{\sqrt{2\pi k} \cdot \Gamma(k/2)} \cdot \left(1 + \frac{t^2}{k}\right)^{-(k+1)/2}\\
&= \frac{\Gamma\left(\frac{k+1}{2}\right)}{\sqrt{k\pi} \, \Gamma\left(\frac{k}{2}\right)} \left(1 + \frac{t^2}{k}\right)^{-(k+1)/2}
\end{align*}
\end{prf}

\subsection{t 分配的性質}

\begin{property}[t 分配的性質]
若 $T \sim t_k$,則:
\begin{enumerate}[label=(\roman*)]
    \item $\expc[T] = 0$(當 $k > 1$)
    \item $\var(T) = \dfrac{k}{k-2}$(當 $k > 2$)
    \item $t$ 分配對稱於 0
    \item $t$ 分配比標準常態分配有較厚的尾部(heavier tails)
    \item 當 $k \to \infty$ 時,$t_k \to N(0, 1)$
\end{enumerate}
\end{property}

\begin{prf}
設 $T = \dfrac{Z}{\sqrt{V/k}}$,其中 $Z \sim N(0,1)$,$V \sim \chi^2_k$ 獨立。

\textbf{(iii) 對稱性}:由於 $Z$ 對稱於 0,且 $\sqrt{V/k} > 0$,故 $T = Z/\sqrt{V/k}$ 也對稱於 0。從 PDF 也可見 $f_T(-t) = f_T(t)$。

\textbf{(i) 期望值}:由於 $Z$ 與 $V$ 獨立,
\[
\expc[T] = \expc\left[\frac{Z}{\sqrt{V/k}}\right] = \expc[Z] \cdot \expc\left[\frac{1}{\sqrt{V/k}}\right] = 0 \cdot \expc\left[\sqrt{\frac{k}{V}}\right] = 0
\]
(當 $k > 1$ 時,$\expc[1/\sqrt{V}]$ 存在且有限。)

\textbf{(ii) 變異數}:當 $k > 2$ 時,
\[
\var(T) = \expc[T^2] - (\expc[T])^2 = \expc[T^2] = \expc\left[\frac{Z^2}{V/k}\right] = k \cdot \expc[Z^2] \cdot \expc\left[\frac{1}{V}\right]
\]

由於 $\expc[Z^2] = 1$,且由引理 \ref{lem:chi-inverse},$\expc[1/V] = \dfrac{1}{k-2}$(當 $k > 2$):
\[
\var(T) = k \cdot 1 \cdot \frac{1}{k-2} = \frac{k}{k-2}
\]

\textbf{(v) 當 $k \to \infty$ 時}:由大數法則,$V/k \to \expc[V/k] = 1$(機率收斂)。因此:
\[
T = \frac{Z}{\sqrt{V/k}} \to \frac{Z}{\sqrt{1}} = Z \sim N(0,1)
\]

也可從 PDF 驗證:當 $k \to \infty$ 時,$\left(1 + \dfrac{t^2}{k}\right)^{-(k+1)/2} \to e^{-t^2/2}$。
\end{prf}

\begin{remark}
\begin{itemize}
  \item[]
    \item 當 $k = 1$ 時,$\expc[T]$ 不存在(Cauchy 分配)。
    \item 當 $k \leqslant 2$ 時,$\var(T)$ 不存在(無限或未定義)。
    \item 當 $k = 2$ 時,$\var(T) = \infty$。
    \item 當 $k > 2$ 時,$\var(T) = k/(k-2) > 1$,比標準常態的變異數 1 大,這對應於較厚的尾部。
\end{itemize}
\end{remark}

\subsection{樣本平均數的 t 統計量}

\begin{theorem}[樣本平均數的 t 統計量]\label{thm:t-stat}
若 $X_1, X_2, \ldots, X_n \simark N(\mu, \sigma^2)$,則:
\[
T = \frac{\bar{X} - \mu}{S/\sqrt{n}} \sim t_{n-1}
\]
\end{theorem}

\begin{prf}
由定理 \ref{thm:S2-chisq}:
\[
Z = \frac{\bar{X} - \mu}{\sigma/\sqrt{n}} \sim N(0, 1), \quad V = \frac{(n-1)S^2}{\sigma^2} \sim \chi^2_{n-1}
\]
且 $Z$ 與 $V$ 獨立。

\begin{align*}
T &= \frac{Z}{\sqrt{V/(n-1)}} = \frac{\dfrac{\bar{X} - \mu}{\sigma/\sqrt{n}}}{\sqrt{\dfrac{(n-1)S^2}{\sigma^2(n-1)}}}\\[2mm]
&= \frac{\bar{X} - \mu}{\sigma/\sqrt{n}} \cdot \frac{\sigma}{S} = \frac{\bar{X} - \mu}{S/\sqrt{n}} \sim t_{n-1}
\end{align*}
\end{prf}

\begin{remark}
\textbf{t 分配的重要性}:當母體變異數 $\sigma^2$ 未知時(實務中幾乎總是如此),我們用 $S^2$ 估計 $\sigma^2$。此時使用 t 分配而非常態分配來進行推論。
\end{remark}

\begin{table}[H]
\centering
\begin{tabular}{c|cccccc}
\toprule
df & $t_{0.10}$ & $t_{0.05}$ & $t_{0.025}$ & $t_{0.01}$ & $t_{0.005}$ \\
\midrule
5  & 1.476 & 2.015 & 2.571 & 3.365 & 4.032 \\
10 & 1.372 & 1.812 & 2.228 & 2.764 & 3.169 \\
15 & 1.341 & 1.753 & 2.131 & 2.602 & 2.947 \\
20 & 1.325 & 1.725 & 2.086 & 2.528 & 2.845 \\
25 & 1.316 & 1.708 & 2.060 & 2.485 & 2.787 \\
30 & 1.310 & 1.697 & 2.042 & 2.457 & 2.750 \\
$\infty$ & 1.282 & 1.645 & 1.960 & 2.326 & 2.576 \\
\bottomrule
\end{tabular}
\caption{t 分配的常用臨界值($t_{\alpha, df}$:右尾機率為 $\alpha$ 的分位數)}
\end{table}

%=============================================================================
\section{F 分配}
%=============================================================================

\subsection{F 分配的定義與 PDF 推導}

\begin{definition}[F 分配]
若 $U \sim \chi^2_{k_1}$ 與 $V \sim \chi^2_{k_2}$ 獨立,則
\[
F = \frac{U/k_1}{V/k_2}
\]
服從自由度為 $(k_1, k_2)$ 的 \textbf{F 分配},記為 $F \sim F_{k_1, k_2}$ 或 $F(k_1, k_2)$。
\end{definition}

\begin{theorem}[F 分配的 PDF]\label{thm:F-pdf}
若 $F \sim F_{k_1, k_2}$,則其 PDF 為:
\[
f_F(x) = \frac{\Gamma\left(\frac{k_1+k_2}{2}\right)}{\Gamma\left(\frac{k_1}{2}\right)\Gamma\left(\frac{k_2}{2}\right)} \left(\frac{k_1}{k_2}\right)^{k_1/2} \frac{x^{k_1/2 - 1}}{\left(1 + \frac{k_1 x}{k_2}\right)^{(k_1+k_2)/2}}, \quad x > 0
\]
\end{theorem}

\begin{prf}
設 $U \sim \chi^2_{k_1}$,$V \sim \chi^2_{k_2}$ 獨立。其聯合 PDF 為:
\[
f_{U,V}(u, v) = \frac{u^{k_1/2-1} e^{-u/2}}{2^{k_1/2} \Gamma(k_1/2)} \cdot \frac{v^{k_2/2-1} e^{-v/2}}{2^{k_2/2} \Gamma(k_2/2)}, \quad u, v > 0
\]

令變換 $(U, V) \to (F, V)$:
\[
F = \frac{U/k_1}{V/k_2} = \frac{k_2 U}{k_1 V}, \quad V = V
\]

反變換:
\[
U = \frac{k_1}{k_2} F V, \quad V = V
\]

計算 Jacobian:
\[
J = \frac{\partial(u, v)}{\partial(f, v)} = \begin{vmatrix} \dfrac{\partial u}{\partial f} & \dfrac{\partial u}{\partial v} \\[3mm] \dfrac{\partial v}{\partial f} & \dfrac{\partial v}{\partial v} \end{vmatrix} = \begin{vmatrix} \dfrac{k_1 v}{k_2} & \dfrac{k_1 f}{k_2} \\[2mm] 0 & 1 \end{vmatrix} = \frac{k_1 v}{k_2}
\]

$(F, V)$ 的聯合 PDF:
\begin{align*}
f_{F,V}(f, v) &= f_{U,V}\left(\frac{k_1 f v}{k_2}, v\right) \cdot |J|\\
&= \frac{\left(\frac{k_1 f v}{k_2}\right)^{k_1/2-1} e^{-k_1 f v/(2k_2)}}{2^{k_1/2} \Gamma(k_1/2)} \cdot \frac{v^{k_2/2-1} e^{-v/2}}{2^{k_2/2} \Gamma(k_2/2)} \cdot \frac{k_1 v}{k_2}
\end{align*}

整理:
\begin{align*}
f_{F,V}(f, v) &= \frac{1}{2^{(k_1+k_2)/2} \Gamma(k_1/2) \Gamma(k_2/2)} \left(\frac{k_1}{k_2}\right)^{k_1/2} f^{k_1/2-1} v^{(k_1+k_2)/2-1} e^{-v(1 + k_1 f/k_2)/2}
\end{align*}

$F$ 的邊際 PDF(對 $v$ 積分):
\begin{align*}
f_F(f) &= \int_0^{\infty} f_{F,V}(f, v) \, dv
\end{align*}

令 $w = v(1 + k_1 f/k_2)/2$,則積分變為 $\Gamma$ 函數:
\begin{align*}
f_F(f) &= \frac{\left(\frac{k_1}{k_2}\right)^{k_1/2} f^{k_1/2-1}}{2^{(k_1+k_2)/2} \Gamma(k_1/2) \Gamma(k_2/2)} \cdot \frac{2^{(k_1+k_2)/2} \Gamma\left(\frac{k_1+k_2}{2}\right)}{\left(1 + \frac{k_1 f}{k_2}\right)^{(k_1+k_2)/2}}\\
&= \frac{\Gamma\left(\frac{k_1+k_2}{2}\right)}{\Gamma\left(\frac{k_1}{2}\right)\Gamma\left(\frac{k_2}{2}\right)} \left(\frac{k_1}{k_2}\right)^{k_1/2} \frac{f^{k_1/2 - 1}}{\left(1 + \frac{k_1 f}{k_2}\right)^{(k_1+k_2)/2}}
\end{align*}
\end{prf}

\subsection{F 分配的性質}

\begin{property}[F 分配的性質]
若 $F \sim F_{k_1, k_2}$,則:
\begin{enumerate}[label=(\roman*)]
    \item $F \geqslant 0$
    \item $\expc[F] = \dfrac{k_2}{k_2 - 2}$(當 $k_2 > 2$)
    \item 若 $F \sim F_{k_1, k_2}$,則 $\dfrac{1}{F} \sim F_{k_2, k_1}$
    \item $F_{1-\alpha, k_1, k_2} = \dfrac{1}{F_{\alpha, k_2, k_1}}$
    \item 若 $T \sim t_k$,則 $T^2 \sim F_{1, k}$
\end{enumerate}
\end{property}

\begin{prf}
\textbf{(ii) 期望值}:設 $F = \dfrac{U/k_1}{V/k_2}$,其中 $U \sim \chi^2_{k_1}$,$V \sim \chi^2_{k_2}$ 獨立。

\[
F = \frac{k_2}{k_1} \cdot \frac{U}{V}
\]

由於 $U$ 與 $V$ 獨立:
\[
\expc[F] = \frac{k_2}{k_1} \cdot \expc[U] \cdot \expc\left[\frac{1}{V}\right] = \frac{k_2}{k_1} \cdot k_1 \cdot \frac{1}{k_2 - 2} = \frac{k_2}{k_2 - 2}
\]
(利用 $\expc[U] = k_1$ 及引理 \ref{lem:chi-inverse}:$\expc[1/V] = 1/(k_2-2)$,當 $k_2 > 2$。)

\textbf{(iii) 倒數性質}:若 $F = \dfrac{U/k_1}{V/k_2} \sim F_{k_1, k_2}$,則:
\[
\frac{1}{F} = \frac{V/k_2}{U/k_1} \sim F_{k_2, k_1}
\]
這直接由 F 分配的定義得到。

\textbf{(iv) 分位數關係}:由性質 (iii),若 $F \sim F_{k_1, k_2}$,則 $1/F \sim F_{k_2, k_1}$。

設 $F_{\alpha, k_1, k_2}$ 為 $F_{k_1, k_2}$ 的上 $\alpha$ 分位數,即 $\prb(F > F_{\alpha, k_1, k_2}) = \alpha$。

則 $\prb(F < F_{\alpha, k_1, k_2}) = 1 - \alpha$,等價於 $\prb(1/F > 1/F_{\alpha, k_1, k_2}) = 1 - \alpha$。

由於 $1/F \sim F_{k_2, k_1}$,這意味著 $1/F_{\alpha, k_1, k_2}$ 是 $F_{k_2, k_1}$ 的上 $(1-\alpha)$ 分位數:
\[
\frac{1}{F_{\alpha, k_1, k_2}} = F_{1-\alpha, k_2, k_1}
\]

移項得 $F_{1-\alpha, k_1, k_2} = \dfrac{1}{F_{\alpha, k_2, k_1}}$。

\textbf{(v) 與 t 分配的關係}:設 $T = \dfrac{Z}{\sqrt{V/k}} \sim t_k$,其中 $Z \sim N(0,1)$,$V \sim \chi^2_k$ 獨立。

則 $Z^2 \sim \chi^2_1$(由定理 \ref{thm:chi1-pdf}),故:
\[
T^2 = \frac{Z^2}{V/k} = \frac{Z^2/1}{V/k} \sim F_{1, k}
\]
\end{prf}

\subsection{兩樣本變異數比}

\begin{theorem}[兩樣本變異數比的分配]\label{thm:F-ratio}
設 $X_1, \ldots, X_{n_1} \simark N(\mu_1, \sigma_1^2)$ 與 $Y_1, \ldots, Y_{n_2} \simark N(\mu_2, \sigma_2^2)$ 為兩組獨立樣本,樣本變異數分別為 $S_1^2$ 與 $S_2^2$。則:
\[
F = \frac{S_1^2 / \sigma_1^2}{S_2^2 / \sigma_2^2} \sim F_{n_1-1, n_2-1}
\]

特別地,若 $\sigma_1^2 = \sigma_2^2$:
\[
F = \frac{S_1^2}{S_2^2} \sim F_{n_1-1, n_2-1}
\]
\end{theorem}

\begin{prf}
由定理 \ref{thm:S2-chisq}:
\[
\frac{(n_1-1)S_1^2}{\sigma_1^2} \sim \chi^2_{n_1-1}, \quad \frac{(n_2-1)S_2^2}{\sigma_2^2} \sim \chi^2_{n_2-1}
\]

且由於兩組樣本獨立,這兩個卡方隨機變數也獨立。

由 F 分配的定義:
\begin{align*}
F &= \frac{\left[\dfrac{(n_1-1)S_1^2}{\sigma_1^2}\right] / (n_1-1)}{\left[\dfrac{(n_2-1)S_2^2}{\sigma_2^2}\right] / (n_2-1)} = \frac{S_1^2/\sigma_1^2}{S_2^2/\sigma_2^2} \sim F_{n_1-1, n_2-1}
\end{align*}

若 $\sigma_1^2 = \sigma_2^2$,則 $F = \dfrac{S_1^2}{S_2^2} \sim F_{n_1-1, n_2-1}$。
\end{prf}

%=============================================================================
\section{常用抽樣分配摘要}
%=============================================================================

\begin{table}[H]
\centering
\renewcommand{\arraystretch}{1.8}
\begin{tabular}{p{5cm}|c|c}
\toprule
\textbf{情況} & \textbf{統計量} & \textbf{分配} \\
\midrule
母體 $N(\mu, \sigma^2)$,$\sigma^2$ 已知 & $\ds Z = \frac{\bar{X} - \mu}{\sigma/\sqrt{n}}$ & $N(0, 1)$ \\
\midrule
母體 $N(\mu, \sigma^2)$,$\sigma^2$ 未知 & $\ds T = \frac{\bar{X} - \mu}{S/\sqrt{n}}$ & $t_{n-1}$ \\
\midrule
母體 $N(\mu, \sigma^2)$ & $\ds \chi^2 = \frac{(n-1)S^2}{\sigma^2}$ & $\chi^2_{n-1}$ \\
\midrule
兩獨立常態母體,$\sigma_1^2 = \sigma_2^2$ & $\ds F = \frac{S_1^2}{S_2^2}$ & $F_{n_1-1, n_2-1}$ \\
\midrule
任意母體,$n$ 大(CLT) & $\ds Z = \frac{\bar{X} - \mu}{\sigma/\sqrt{n}}$ & $\approx N(0, 1)$ \\
\bottomrule
\end{tabular}
\caption{常用抽樣分配摘要}
\end{table}

\begin{table}[H]
\centering
\renewcommand{\arraystretch}{1.6}
\begin{tabular}{l|c|c|c}
\toprule
\textbf{分配} & \textbf{PDF} & $\expc[X]$ & $\var(X)$ \\
\midrule
$\chi^2_k$ & $\ds\frac{x^{k/2-1} e^{-x/2}}{2^{k/2} \Gamma(k/2)}$ & $k$ & $2k$ \\
\midrule
$t_k$ & $\ds\frac{\Gamma(\frac{k+1}{2})}{\sqrt{k\pi}\Gamma(\frac{k}{2})}\left(1+\frac{t^2}{k}\right)^{-\frac{k+1}{2}}$ & $0$ $(k>1)$ & $\dfrac{k}{k-2}$ $(k>2)$ \\
\midrule
$F_{k_1,k_2}$ & $\ds\frac{\Gamma(\frac{k_1+k_2}{2})}{\Gamma(\frac{k_1}{2})\Gamma(\frac{k_2}{2})}\left(\frac{k_1}{k_2}\right)^{k_1/2}\frac{x^{k_1/2-1}}{(1+\frac{k_1 x}{k_2})^{(k_1+k_2)/2}}$ & $\dfrac{k_2}{k_2-2}$ $(k_2>2)$ & — \\
\bottomrule
\end{tabular}
\caption{三大抽樣分配的 PDF 與動差}
\end{table}

%=============================================================================
\section{本章習題}
%=============================================================================

\begin{exercise}
設母體平均數 $\mu = 50$,母體變異數 $\sigma^2 = 100$。從此母體抽取 $n = 25$ 的隨機樣本。
\begin{enumerate}[label=(\alph*)]
    \item 求 $\expc[\bar{X}]$ 與 $\var(\bar{X})$
    \item 求 $\SE(\bar{X})$
    \item 若改為 $n = 100$,標準誤變為多少?
\end{enumerate}
\end{exercise}

\begin{solution}
\begin{enumerate}[label=(\alph*)]
    \item $\expc[\bar{X}] = \mu = 50$
    
    $\var(\bar{X}) = \dfrac{\sigma^2}{n} = \dfrac{100}{25} = 4$
    
    \item $\SE(\bar{X}) = \dfrac{\sigma}{\sqrt{n}} = \dfrac{10}{\sqrt{25}} = 2$
    
    \item 若 $n = 100$:$\SE(\bar{X}) = \dfrac{10}{\sqrt{100}} = 1$
    
    樣本量增加 4 倍,標準誤減半。
\end{enumerate}
\end{solution}

\begin{exercise}
某零件壽命的平均為 1000 小時,標準差為 200 小時(不假設常態分配)。隨機抽取 64 個零件。
\begin{enumerate}[label=(\alph*)]
    \item 求樣本平均壽命超過 1050 小時的機率
    \item 求樣本平均壽命介於 950 與 1030 小時之間的機率
\end{enumerate}
\end{exercise}

\begin{solution}
$\mu = 1000$,$\sigma = 200$,$n = 64$。

由 CLT,$\bar{X} \stackrel{\text{approx}}{\sim} N\left(1000, \dfrac{200^2}{64}\right) = N(1000, 625)$

$\SE(\bar{X}) = \dfrac{200}{8} = 25$

\begin{enumerate}[label=(\alph*)]
    \item $\prb(\bar{X} > 1050) = P\left(Z > \dfrac{1050-1000}{25}\right) = \prb(Z > 2) \approx 0.0228$
    
    \item $\prb(950 < \bar{X} < 1030) = P\left(\dfrac{950-1000}{25} < Z < \dfrac{1030-1000}{25}\right)$
    
    $= \prb(-2 < Z < 1.2) = \Phi(1.2) - \Phi(-2) = 0.8849 - 0.0228 = 0.8621$
\end{enumerate}
\end{solution}

\begin{exercise}
設 $X_1, \ldots, X_{10} \simark N(\mu, 16)$(即 $\sigma^2 = 16$)。求 $\prb(S^2 > 25)$。
\end{exercise}

\begin{solution}
由定理 \ref{thm:S2-chisq}:
\[
\frac{(n-1)S^2}{\sigma^2} = \frac{9S^2}{16} \sim \chi^2_9
\]

\begin{align*}
\prb(S^2 > 25) &= P\left(\frac{9S^2}{16} > \frac{9 \times 25}{16}\right) = P(\chi^2_9 > 14.0625)
\end{align*}

查 $\chi^2$ 表:$\chi^2_{9, 0.10} = 14.68$,$\chi^2_{9, 0.15} \approx 13.29$

故 $\prb(S^2 > 25) \approx 0.12$(介於 0.10 與 0.15 之間)。
\end{solution}

\begin{exercise}
設 $X_1, \ldots, X_9 \simark N(20, \sigma^2)$($\sigma^2$ 未知),$S = 4$。求使 $\prb(|\bar{X} - 20| \leqslant c) = 0.95$ 的 $c$ 值。
\end{exercise}

\begin{solution}
$\sigma^2$ 未知,使用 t 分配。

$T = \dfrac{\bar{X} - 20}{S/\sqrt{9}} = \dfrac{\bar{X} - 20}{4/3} \sim t_8$

$\prb(|\bar{X} - 20| \leqslant c) = 0.95$

$P\left(\left|\dfrac{\bar{X} - 20}{4/3}\right| \leqslant \dfrac{c}{4/3}\right) = 0.95$

查表:$t_{0.025, 8} = 2.306$

故 $\dfrac{c}{4/3} = 2.306 \ie c = 2.306 \times \dfrac{4}{3} = 3.075$
\end{solution}

\begin{exercise}
設兩組獨立常態樣本:第一組 $n_1 = 13$,$S_1^2 = 24$;第二組 $n_2 = 10$,$S_2^2 = 12$。假設兩母體變異數相等。求 $\prb(S_1^2/S_2^2 > 2)$。
\end{exercise}

\begin{solution}
若 $\sigma_1^2 = \sigma_2^2$,則 $F = S_1^2/S_2^2 \sim F_{12, 9}$。

$F = \dfrac{24}{12} = 2$

$\prb(F_{12, 9} > 2)$

查表:$F_{0.10, 12, 9} = 2.38$,$F_{0.25, 12, 9} \approx 1.56$

故 $\prb(F_{12, 9} > 2) \approx 0.15$(介於 0.10 與 0.25 之間)。
\end{solution}

%=============================================================================
%% \section*{本章重點整理}
%=============================================================================

% \begin{enumerate}
%     \item \textbf{樣本統計量}:
%     \begin{itemize}
%         \item 樣本平均數:$\bar{X} = \dfrac{1}{n}\sum_{i=1}^{n} X_i$
%         \item 樣本變異數:$S^2 = \dfrac{1}{n-1}\sum_{i=1}^{n}(X_i - \bar{X})^2$
%         \item $\expc[\bar{X}] = \mu$,$\var(\bar{X}) = \sigma^2/n$,$\expc[S^2] = \sigma^2$
%     \end{itemize}
    
%     \item \textbf{中央極限定理}:無論母體分配為何,當 $n$ 夠大時:
%     \[
%     \frac{\bar{X} - \mu}{\sigma/\sqrt{n}} \stackrel{\text{approx}}{\sim} N(0, 1)
%     \]
    
%     \item \textbf{連續性校正}:用連續分配近似離散分配時,將整數 $k$ 對應到區間 $(k-0.5, k+0.5)$。
    
%     \item \textbf{卡方分配} $\chi^2_k$:
%     \begin{itemize}
%         \item 定義:$k$ 個獨立 $N(0,1)^2$ 之和
%         \item PDF:$\dfrac{x^{k/2-1} e^{-x/2}}{2^{k/2} \Gamma(k/2)}$
%         \item $\expc = k$,$\var = 2k$
%         \item $(n-1)S^2/\sigma^2 \sim \chi^2_{n-1}$
%     \end{itemize}
    
%     \item \textbf{t 分配} $t_k$:
%     \begin{itemize}
%         \item 定義:$Z / \sqrt{V/k}$,$Z \sim N(0,1)$,$V \sim \chi^2_k$ 獨立
%         \item PDF:$\dfrac{\Gamma(\frac{k+1}{2})}{\sqrt{k\pi}\Gamma(\frac{k}{2})}\left(1+\dfrac{t^2}{k}\right)^{-(k+1)/2}$
%         \item $\expc = 0$($k>1$),$\var = k/(k-2)$($k>2$)
%         \item $(\bar{X} - \mu)/(S/\sqrt{n}) \sim t_{n-1}$
%     \end{itemize}
    
%     \item \textbf{F 分配} $F_{k_1,k_2}$:
%     \begin{itemize}
%         \item 定義:$(U/k_1)/(V/k_2)$,$U \sim \chi^2_{k_1}$,$V \sim \chi^2_{k_2}$ 獨立
%         \item $1/F_{k_1,k_2} \sim F_{k_2,k_1}$
%         \item $T^2 \sim F_{1,k}$ 若 $T \sim t_k$
%         \item $S_1^2/S_2^2 \sim F_{n_1-1,n_2-1}$(若 $\sigma_1^2 = \sigma_2^2$)
%     \end{itemize}
    
%     \item \textbf{重要引理}:
%     \begin{itemize}
%         \item $\expc[Z^4] = 3$($Z \sim N(0,1)$)
%         \item $\expc[1/V] = 1/(k-2)$($V \sim \chi^2_k$,$k > 2$)
%     \end{itemize}
% \end{enumerate}

\end{document}
