\documentclass[addpoints,12pt,a4paper]{exam}
\printanswers
\usepackage{xeCJK}
\setCJKmainfont[AutoFakeSlant=.1,AutoFakeBold=2]{Noto Serif CJK TC} 
\usepackage{amsmath,amsthm,amssymb,graphicx,hyperref,booktabs,tabularx,enumitem}
\pagestyle{headandfoot}
\firstpageheadrule
\firstpageheader{}{國立臺灣大學 113 學年度碩士班招生考試試題\\統計學(K)(題號:292,節次:7)}{}
\runningheader{}{統計學(K) 詳解}{}
\runningheadrule
\firstpagefooter{}{第\thepage\ 頁(共\numpages 頁)}{}
\runningfooter{}{第\thepage\ 頁(共\numpages 頁)}{}
\footrule
\extraheadheight{-8mm}
\extrafootheight{-10mm}
\extrawidth{35mm}
\newcommand{\ie}{\,\Longrightarrow\,}
\newcommand{\ifff}{\,\Longleftrightarrow\,}
\newcommand{\ds}{\displaystyle}
\renewcommand{\solutiontitle}{
  \noindent\textbf{解答:}
}
\usepackage{multicol}

\begin{document}
\begin{center}
    \fbox{\fbox{\parbox{14cm}{\centering
  第一部分:選擇題 (80\%),每題 5 分。\\
  第二部分:計算題 (20\%),每題 10 分。
    }}}
\end{center}
\vspace{3mm}

\section*{第一部分:選擇題}

\begin{questions}
\pointname{ 分}

%% 第 1 題
\question[5] A company is assessing the effectiveness of four different training programs for its employees to identify which program yields the highest performance improvement. Each program is administered to fifty randomly selected employees, and their performance scores before and after the training are recorded. Which of the following tests is the most suitable for comparing the effectiveness of the training programs?
\begin{choices}
\choice An independent-sample $t$-test
\choice A paired-sample $t$-test
\choice A one-way ANOVA $F$-test
\choice A repeated measures ANOVA test
\choice A Chi-squared test
\end{choices}

\begin{solution}
本題涉及比較四種不同訓練計畫的效果,需要考慮以下幾點:

\textbf{(A) 獨立樣本 $t$ 檢定}:僅適用於比較\textbf{兩組}獨立樣本的平均數差異,無法同時比較四組,故\textbf{不適合}。

\textbf{(B) 配對樣本 $t$ 檢定}:適用於比較\textbf{同一組}受試者在兩種情況下的差異(如前測與後測),但僅能處理兩組比較,無法同時比較四種計畫,故\textbf{不適合}。

\textbf{(C) 單因子變異數分析 (One-way ANOVA)}:適用於比較\textbf{三組以上獨立樣本}的平均數差異。若每位員工只接受一種訓練計畫,且我們比較各計畫的「改善幅度」(後測 $-$ 前測),則可使用。但此方法\textbf{忽略了前後測的配對結構}。

\textbf{(D) 重複測量變異數分析 (Repeated Measures ANOVA)}:本題中,每位員工有前測與後測兩筆配對資料,且有四種不同的訓練計畫。這是一個混合設計:
\begin{itemize}
\item \textbf{組間因子}(Between-subjects factor):訓練計畫(4 組)
\item \textbf{組內因子}(Within-subjects factor):時間(前測 vs. 後測)
\end{itemize}
重複測量 ANOVA 能正確處理配對資料的相關性,並同時比較四種計畫的效果差異,故為\textbf{最適合的方法}。

\textbf{(E) 卡方檢定}:用於類別變數的獨立性或適合度檢定,不適用於連續變數(績效分數)的比較,故\textbf{不適合}。

\vspace{2mm}
\textbf{答案:(D)}
\end{solution}

%% 第 2 題
\question[5] The table below displays the quarterly sales of a gifting store. What is the closest sales forecast for the $2^{\text{nd}}$ quarter of year 4?

\begin{center}
\begin{tabular}{|c|cccc|cccc|cccc|}
\hline
Year & \multicolumn{4}{c|}{1} & \multicolumn{4}{c|}{2} & \multicolumn{4}{c|}{3} \\
\hline
Quarter & 1 & 2 & 3 & 4 & 1 & 2 & 3 & 4 & 1 & 2 & 3 & 4 \\
\hline
Sales & 103 & 204 & 110 & 114 & 116 & 218 & 122 & 126 & 124 & 224 & 133 & 134 \\
\hline
\end{tabular}
\end{center}

\begin{choices}
\choice 132
\choice 141
\choice 150
\choice 234
\choice Cannot make the prediction based on the information provided.
\end{choices}

\begin{solution}
觀察資料可以發現明顯的\textbf{季節性模式}:
\begin{itemize}
\item 第 2 季的銷售額明顯高於其他季(約為 200 以上)
\item 第 1, 3, 4 季的銷售額較為接近(約 100--135 之間)
\end{itemize}

讓我們分析第 2 季的趨勢:
\begin{itemize}
\item Year 1, Q2: 204
\item Year 2, Q2: 218
\item Year 3, Q2: 224
\end{itemize}

計算第 2 季的成長趨勢:
\begin{itemize}
\item Year 1 $\to$ Year 2: $218 - 204 = 14$
\item Year 2 $\to$ Year 3: $224 - 218 = 6$
\end{itemize}

平均成長:$(14 + 6)/2 = 10$

若採用簡單趨勢外推:
\[
\text{Year 4, Q2} \approx 224 + 10 = 234
\]

另一種方法是使用線性迴歸。設 $t$ 為時間(第 2 季分別對應 $t = 2, 6, 10$),觀察值為 $Y = 204, 218, 224$。

計算斜率:
\[
\bar{t} = \frac{2+6+10}{3} = 6, \quad \bar{Y} = \frac{204+218+224}{3} = 215.33
\]
\[
b = \frac{\sum(t_i - \bar{t})(Y_i - \bar{Y})}{\sum(t_i - \bar{t})^2} = \frac{(2-6)(204-215.33)+(6-6)(218-215.33)+(10-6)(224-215.33)}{(2-6)^2+(6-6)^2+(10-6)^2}
\]
\[
= \frac{(-4)(-11.33)+(0)(2.67)+(4)(8.67)}{16+0+16} = \frac{45.32+0+34.68}{32} = \frac{80}{32} = 2.5
\]

預測 Year 4, Q2($t = 14$):
\[
\hat{Y} = \bar{Y} + b(t - \bar{t}) = 215.33 + 2.5(14 - 6) = 215.33 + 20 = 235.33 \approx 234
\]

\vspace{2mm}
\textbf{答案:(D) 234}
\end{solution}

%% 第 3 題
\question[5] A researcher wants to test whether the type of car owned (domestic or foreign) is independent of gender. He surveys 1000 car owners about their gender and the type of car they own. The result is summarized using cross-tabulation as below. Which of the following statements is incorrect?

\begin{center}
\begin{tabular}{l|cc}
 & \multicolumn{2}{c}{Gender} \\
Car Type & Male & Female \\
\hline
Domestic & 100 & 400 \\
Foreign & 300 & 200 \\
\end{tabular}
\end{center}

\begin{choices}
\choice The null hypothesis can be written as males and females own similar proportions of foreign cars.
\choice At alpha $= 0.05$, we can conclude that the type of car owned is independent of gender.
\choice The researcher should use a chi-squared test to perform the analysis.
\choice The gender distribution of domestic car owners is different from that of foreign car owners.
\choice B and C
\end{choices}

\begin{solution}
先分析列聯表數據:

\begin{center}
\begin{tabular}{l|cc|c}
Car Type & Male & Female & Total \\
\hline
Domestic & 100 & 400 & 500 \\
Foreign & 300 & 200 & 500 \\
\hline
Total & 400 & 600 & 1000 \\
\end{tabular}
\end{center}

\textbf{(A) 正確}。虛無假設 $H_0$:汽車類型與性別獨立,等價於「男性與女性擁有外國車的比例相同」。
\begin{itemize}
\item 男性擁有外國車比例:$300/400 = 75\%$
\item 女性擁有外國車比例:$200/600 = 33.3\%$
\end{itemize}
虛無假設確實可表述為這兩比例相等。

\textbf{(B) 錯誤}。讓我們計算卡方統計量。期望值:
\[
E_{11} = \frac{500 \times 400}{1000} = 200, \quad E_{12} = \frac{500 \times 600}{1000} = 300
\]
\[
E_{21} = \frac{500 \times 400}{1000} = 200, \quad E_{22} = \frac{500 \times 600}{1000} = 300
\]

卡方統計量:
\[
\chi^2 = \frac{(100-200)^2}{200} + \frac{(400-300)^2}{300} + \frac{(300-200)^2}{200} + \frac{(200-300)^2}{300}
\]
\[
= \frac{10000}{200} + \frac{10000}{300} + \frac{10000}{200} + \frac{10000}{300} = 50 + 33.33 + 50 + 33.33 = 166.67
\]

自由度 $df = (2-1)(2-1) = 1$,臨界值 $\chi^2_{0.05,1} = 3.841$。

由於 $166.67 \gg 3.841$,我們\textbf{拒絕}虛無假設,結論是汽車類型與性別\textbf{不獨立}。選項 (B) 說「可結論為獨立」是\textbf{錯誤}的。

\textbf{(C) 正確}。對於兩個類別變數的獨立性檢定,應使用卡方獨立性檢定。

\textbf{(D) 正確}。
\begin{itemize}
\item 國產車車主:男性 $100/500 = 20\%$,女性 $400/500 = 80\%$
\item 外國車車主:男性 $300/500 = 60\%$,女性 $200/500 = 40\%$
\end{itemize}
性別分布確實不同。

\textbf{(E)}:(B) 錯誤,(C) 正確,故 (E) 不正確。

\vspace{2mm}
\textbf{答案:(B)}

(注意:題目問的是哪個選項 incorrect,(B) 的陳述是錯誤的)
\end{solution}

%% 第 4 題
\question[5] For the following scatter plots, which estimate of the correlation coefficients is closest to the true values?

\begin{center}
\textit{(左圖顯示正相關的散佈圖,右圖顯示無明顯線性相關的散佈圖)}
\end{center}

\begin{choices}
\choice Left: $+0.8$; Right: $-1.0$
\choice Left: $+1.0$; Right: $-0.8$
\choice Left: $+0.7$; Right: $-0.2$
\choice Left: $+0.6$; Right: $-0.6$
\choice Left: $+0.9$; Right: $-0.4$
\end{choices}

\begin{solution}
根據散佈圖觀察:

\textbf{左圖}:
\begin{itemize}
\item 資料點顯示明顯的\textbf{正向線性關係}
\item 點分布在一條斜向上的帶狀區域
\item 有一定的離散程度,但趨勢明確
\item 相關係數估計約在 $+0.7$ 到 $+0.8$ 之間
\end{itemize}

\textbf{右圖}:
\begin{itemize}
\item 資料點分布較為分散
\item 呈現\textbf{輕微的負向}或接近無相關的模式
\item 部分點在中間偏高,右側偏低,但關係不強
\item 相關係數估計約在 $-0.2$ 到 $-0.4$ 之間
\end{itemize}

逐一檢視選項:
\begin{itemize}
\item (A) 左 $+0.8$,右 $-1.0$:右圖不可能是完美負相關 $-1.0$
\item (B) 左 $+1.0$,右 $-0.8$:左圖不是完美正相關,右圖也非強負相關
\item (C) 左 $+0.7$,右 $-0.2$:左圖約 $0.7$ 合理,右圖約 $-0.2$ 也符合觀察
\item (D) 左 $+0.6$,右 $-0.6$:左圖相關性應更高,右圖負相關不該這麼強
\item (E) 左 $+0.9$,右 $-0.4$:左圖 $0.9$ 稍高,右圖 $-0.4$ 稍強
\end{itemize}

\vspace{2mm}
\textbf{答案:(C)}
\end{solution}

%% 第 5 題
\question[5] Suppose we are examining the factors influencing the success of undergraduate programs, with the percentage of students going on to graduate schools as a measure of success. The faculty's union proposes a relationship between the average professors' monthly salary and the success of undergraduate programs. The resulting regression line is ``\% of students going on to graduate schools $= 12 + 0.001 \times$ (Average Professors' Monthly Salary).'' Which of the following statements is true regarding the given scenario?
\begin{choices}
\choice If the percentage of students going on to graduate schools increases by 0.001, we would expect the average professors' monthly salary to approximately increase by one dollar.
\choice If the percentage of students going on to graduate schools increases by 1, we would expect the average professors' monthly salary to approximately increase by 0.001 dollar.
\choice If the average professors' monthly salary increases by 0.001 dollar, we would expect the percentage of students going on to graduate schools to increase approximately by one percent.
\choice If the average professors' monthly salary increases by one dollar, we would expect the percentage of students going on to graduate schools to increase approximately by 0.001 percent.
\choice None of the options above.
\end{choices}

\begin{solution}
迴歸方程式為:
\[
Y = 12 + 0.001 \times X
\]
其中:
\begin{itemize}
\item $Y$:學生繼續讀研究所的百分比(\%)
\item $X$:教授平均月薪(美元)
\end{itemize}

斜率 $\beta_1 = 0.001$ 的解釋:\textbf{當 $X$(教授月薪)增加 1 美元時,$Y$(讀研比例)預期增加 0.001 個百分點}。

逐一檢視選項:

\textbf{(A)}:「讀研比例增加 0.001,預期月薪增加 1 美元」—— 這是\textbf{因果關係顛倒}。迴歸模型是用 $X$ 預測 $Y$,不能反過來直接解讀。\textbf{錯誤}。

\textbf{(B)}:「讀研比例增加 1,預期月薪增加 0.001 美元」—— 同樣是\textbf{因果關係顛倒}。\textbf{錯誤}。

\textbf{(C)}:「月薪增加 0.001 美元,預期讀研比例增加 1\%」—— 計算:$\Delta Y = 0.001 \times 0.001 = 0.000001\%$,不是 1\%。\textbf{錯誤}。

\textbf{(D)}:「月薪增加 1 美元,預期讀研比例增加 0.001\%」—— 計算:$\Delta Y = 0.001 \times 1 = 0.001$,單位是百分點(percentage points)。\textbf{正確}。

\textbf{(E)}:既然 (D) 正確,(E) 錯誤。

\vspace{2mm}
\textbf{答案:(D)}
\end{solution}

\newpage
%% 第 6--7 題共用題幹
\noindent\textbf{Answer questions 6 and 7 using the provided information as follows:}

The researcher is interested in comparing the performance of two franchisees. She takes two samples of daily sales values, each of size 25, from the two franchisees, assumed to be normally distributed with equal variances. The first sample has a mean sales value of \$5500 with a standard deviation of \$460, while the second sample has a mean sales value of \$4800 with a standard deviation of \$340. The researcher aims to test if there is a difference between the population means at the 0.05 significance level.

\vspace{3mm}

%% 第 6 題
\question[5] What conclusion can the researcher draw?
\begin{choices}
\choice There is insufficient evidence to reject the null hypothesis.
\choice The researcher can reject the null hypothesis and conclude that the two population means are equal.
\choice There is sufficient evidence to reject the null hypothesis and conclude that the two population means are different.
\choice The answer cannot be determined without calculating the $p$-value and comparing it to the significance level.
\choice None of the options above.
\end{choices}

\begin{solution}
設立假設:
\begin{itemize}
\item $H_0$: $\mu_1 = \mu_2$(兩母體平均數相等)
\item $H_1$: $\mu_1 \neq \mu_2$(雙尾檢定)
\end{itemize}

已知資訊:
\begin{itemize}
\item $n_1 = n_2 = 25$
\item $\bar{x}_1 = 5500$,$s_1 = 460$
\item $\bar{x}_2 = 4800$,$s_2 = 340$
\item 假設等變異數
\end{itemize}

計算合併變異數(pooled variance):
\[
s_p^2 = \frac{(n_1-1)s_1^2 + (n_2-1)s_2^2}{n_1+n_2-2} = \frac{24 \times 460^2 + 24 \times 340^2}{48}
\]
\[
= \frac{24 \times 211600 + 24 \times 115600}{48} = \frac{5078400 + 2774400}{48} = \frac{7852800}{48} = 163600
\]
\[
s_p = \sqrt{163600} = 404.47
\]

計算 $t$ 統計量:
\[
t = \frac{\bar{x}_1 - \bar{x}_2}{s_p\sqrt{\frac{1}{n_1}+\frac{1}{n_2}}} = \frac{5500 - 4800}{404.47\sqrt{\frac{1}{25}+\frac{1}{25}}} = \frac{700}{404.47 \times 0.2828} = \frac{700}{114.38} \approx 6.12
\]

自由度 $df = n_1 + n_2 - 2 = 48$。

查 $t$ 分配表,$t_{0.025, 48} \approx 2.01$(雙尾 $\alpha = 0.05$)。

由於 $|t| = 6.12 > 2.01$,我們\textbf{拒絕虛無假設},結論是兩母體平均數\textbf{有顯著差異}。

\vspace{2mm}
\textbf{答案:(C)}
\end{solution}

%% 第 7 題
\question[5] The researcher calculated the $p$-value for the $F$ test of equal variances and found it to be 0.171. Based on this $p$-value, what conclusion can the researcher draw?
\begin{choices}
\choice The assumption of equal variances is correct.
\choice The assumption of equal variances is incorrect.
\choice The answer cannot be determined based on this $p$-value.
\choice The variance of the first sample is greater than that of the second one.
\choice None of the options above.
\end{choices}

\begin{solution}
等變異數的 $F$ 檢定(Levene's test 或 $F$-test for equality of variances):
\begin{itemize}
\item $H_0$: $\sigma_1^2 = \sigma_2^2$(兩母體變異數相等)
\item $H_1$: $\sigma_1^2 \neq \sigma_2^2$
\end{itemize}

$p$-value $= 0.171 > 0.05 = \alpha$

決策:\textbf{無法拒絕} $H_0$

逐一檢視選項:

\textbf{(A)}:「等變異數假設正確」—— 統計上只能說「沒有足夠證據顯示假設錯誤」,而不能斷言「假設正確」。但在實務應用中,這通常被解讀為「可以接受等變異數假設」。\textbf{基本正確}。

\textbf{(B)}:「等變異數假設不正確」—— $p = 0.171 > 0.05$,無法拒絕 $H_0$,所以不能說假設不正確。\textbf{錯誤}。

\textbf{(C)}:「無法由此 $p$-value 決定」—— $p$-value 確實可以用來做決策。\textbf{錯誤}。

\textbf{(D)}:「第一組變異數大於第二組」—— 雖然 $s_1^2 > s_2^2$(樣本上),但 $p = 0.171$ 表示沒有足夠證據說明母體變異數有差異。\textbf{錯誤}。

\textbf{(E)}:如果 (A) 被認為是正確的,則 (E) 錯誤。

\vspace{2mm}
\textbf{答案:(A)}

(注:嚴格來說,我們只能說「沒有足夠證據拒絕等變異數假設」,但實務上 (A) 的表述是可接受的)
\end{solution}

%% 第 8 題
\question[5] In analyzing the factors influencing job satisfaction, data were collected on the characteristics of 50 employees, and the regression model is expressed as $S = \beta_0 + \beta_1 A + \beta_2 E + \beta_3 G + \varepsilon$, where $S$ represents job satisfaction, $A$ denotes the number of years of experience, $E$ stands for his/her years of education, and $G$ is a dummy variable for gender ($G=1$ if the employee is female). Suppose the regression outcome is $S = 78.2 + 2.5A + 3.8E - 6.2G$. How would you interpret the coefficient on gender $(G)$?
\begin{choices}
\choice A female employee's satisfaction is 6.2 less than a male employee, on average.
\choice A female employee's satisfaction is 6.2 higher than a male employee, on average.
\choice For any female employees, her satisfaction must be lower than any male employee with a similar number of years of experience and years of education.
\choice A and C
\choice None of the options above is correct.
\end{choices}

\begin{solution}
迴歸模型:$S = 78.2 + 2.5A + 3.8E - 6.2G$

其中 $G$ 是虛擬變數:$G = 1$ 表示女性,$G = 0$ 表示男性。

係數 $\beta_3 = -6.2$ 的解釋:

對於\textbf{男性}($G = 0$):
\[
S_{\text{male}} = 78.2 + 2.5A + 3.8E
\]

對於\textbf{女性}($G = 1$):
\[
S_{\text{female}} = 78.2 + 2.5A + 3.8E - 6.2 = (78.2 - 6.2) + 2.5A + 3.8E
\]

因此,在控制年資 ($A$) 和教育年數 ($E$) 相同的情況下,\textbf{女性員工的工作滿意度平均比男性低 6.2 單位}。

逐一檢視選項:

\textbf{(A)}:「女性員工滿意度平均比男性低 6.2」—— \textbf{正確}。這是迴歸係數的正確解釋(控制其他變數)。

\textbf{(B)}:「女性員工滿意度比男性高 6.2」—— 係數是 $-6.2$,方向錯誤。\textbf{錯誤}。

\textbf{(C)}:「任何女性員工的滿意度必定低於任何具有相同年資和教育的男性員工」—— 迴歸係數只描述\textbf{平均}關係,不是\textbf{每一個}個體的必然結果(因為有誤差項 $\varepsilon$)。\textbf{錯誤}。

\textbf{(D)}:A 和 C —— 由於 (C) 錯誤,(D) 也\textbf{錯誤}。

\textbf{(E)}:如果只有 (A) 正確,(E) 就錯誤。

\vspace{2mm}
\textbf{答案:(A)}
\end{solution}

\newpage
%% 第 9--11 題共用題幹
\noindent\textbf{Answer questions 9--11 using the provided information as follows:}

After collecting the dataset, the research team initially selected two competing brands and generated histograms of the number of daily visitors as follows:

\begin{center}
\textit{(Brand C1 的直方圖呈現右偏分布,範圍約 60000--180000;Brand C2 的直方圖呈現明顯的雙峰分布,範圍約 30000--90000)}
\end{center}

\vspace{3mm}

%% 第 9 題
\question[5] Which of the following statements is correct for Brand C2?
\begin{choices}
\choice The distribution of the number of daily visitors is unimodal.
\choice The total number of observations is 365.
\choice The mean is approximately 86,000.
\choice The median is approximately 70,000.
\choice None of the above options is correct.
\end{choices}

\begin{solution}
根據 Brand C2 的直方圖觀察:

\textbf{(A) 「分布是單峰的」}:從圖可見,Brand C2 的直方圖有\textbf{兩個明顯的峰}(bimodal),一個在較低值區域,一個在較高值區域。\textbf{錯誤}。

\textbf{(B) 「觀測值總數為 365」}:一年有 365 天,如果是每日訪客數據且收集整年,則應該約 365 筆。但需要從直方圖的頻率加總來驗證。圖中最高頻率約 50 多,各柱加總可能接近 365。\textbf{可能正確}。

\textbf{(C) 「平均數約 86,000」}:觀察直方圖,C2 的範圍約在 30,000--90,000,且有兩個峰。平均值估計應在 50,000--70,000 之間,86,000 似乎偏高(接近最大值)。\textbf{錯誤}。

\textbf{(D) 「中位數約 70,000」}:由於是雙峰分布,中位數取決於兩峰的相對大小。從圖看,右側峰(較高值)的頻率較大,中位數可能落在 65,000--75,000 之間。70,000 是合理估計。\textbf{可能正確}。

比較 (B) 和 (D):題目說「以下哪個敘述正確」。若 365 天資料確實,(B) 正確;但從直方圖精確估計中位數較困難。需要更仔細分析。

根據直方圖,Brand C2 右側峰較高,中位數約在 70,000 附近是合理的。

\vspace{2mm}
\textbf{答案:(D)}

(注:需根據實際直方圖詳細計算,此處基於圖形觀察)
\end{solution}

%% 第 10 題
\question[5] Suppose the research team discovered that one data point was falsely recorded for Brand C2. The true number of visitors was 41,100 instead of 31,100 on this date. What is correct regarding the influence of this change on the descriptive statistics?
\begin{choices}
\choice The median increased.
\choice The mean increased.
\choice The standard deviation decreased.
\choice Options B and C.
\choice Cannot conclude.
\end{choices}

\begin{solution}
資料更正:將一筆資料從 31,100 改為 41,100(增加 10,000)。

\textbf{對平均數的影響}:
\[
\text{新平均數} = \text{舊平均數} + \frac{10000}{n}
\]
平均數\textbf{增加}。

\textbf{對中位數的影響}:
\begin{itemize}
\item 31,100 和 41,100 都是相對較小的值(C2 的範圍約 30,000--90,000)
\item 這個改變可能不影響排序後的中間位置
\item 除非原本 31,100 正好在中位數附近,否則中位數通常不變或只有微小變化
\end{itemize}
中位數\textbf{可能不變或稍微增加}。

\textbf{對標準差的影響}:
\begin{itemize}
\item 原始值 31,100 較遠離平均數(假設平均約 60,000--70,000)
\item 更正值 41,100 較接近平均數
\item 將一個極端值拉近平均數會\textbf{減少離散程度}
\end{itemize}
標準差\textbf{減少}。

逐一檢視選項:
\begin{itemize}
\item (A) 中位數增加:不確定,可能不變
\item (B) 平均數增加:\textbf{正確}
\item (C) 標準差減少:\textbf{正確}
\item (D) B 和 C:\textbf{正確}
\end{itemize}

\vspace{2mm}
\textbf{答案:(D)}
\end{solution}

%% 第 11 題
\question[5] What conclusion can the research team make for the comparison between the number of daily visitors of Brand C1 and Brand C2?
\begin{choices}
\choice The median is larger for Brand C1 than Brand C2.
\choice The mean is larger for Brand C1 than Brand C2.
\choice The standard deviation is larger for Brand C1 than Brand C2.
\choice All of the above options are correct.
\choice None of the above options is correct.
\end{choices}

\begin{solution}
比較 Brand C1 和 Brand C2 的直方圖:

\textbf{Brand C1}:
\begin{itemize}
\item 範圍:約 60,000--180,000
\item 分布:右偏(正偏)
\item 主要集中在 120,000--160,000 區間
\end{itemize}

\textbf{Brand C2}:
\begin{itemize}
\item 範圍:約 30,000--90,000
\item 分布:雙峰
\item 值明顯較 C1 小
\end{itemize}

\textbf{(A) 中位數:C1 $>$ C2}
\begin{itemize}
\item C1 的值整體在 60,000 以上,中位數約 120,000--140,000
\item C2 的值整體在 90,000 以下,中位數約 70,000
\item C1 中位數 $>$ C2 中位數:\textbf{正確}
\end{itemize}

\textbf{(B) 平均數:C1 $>$ C2}
\begin{itemize}
\item C1 平均數估計約 120,000--130,000
\item C2 平均數估計約 60,000--70,000
\item C1 平均數 $>$ C2 平均數:\textbf{正確}
\end{itemize}

\textbf{(C) 標準差:C1 $>$ C2}
\begin{itemize}
\item C1 範圍約 120,000(180,000 $-$ 60,000)
\item C2 範圍約 60,000(90,000 $-$ 30,000)
\item C1 的分散程度較大:\textbf{正確}
\end{itemize}

\vspace{2mm}
\textbf{答案:(D) All of the above options are correct.}
\end{solution}

\newpage
%% 第 12--16 題共用題幹
\noindent\textbf{Questions 12--16:} The research team conducted various analyses, including tests to determine whether the daily number of visitors differs for different brands or between weekdays and weekends. For questions 12--16, choose the appropriate SPSS output (A, B, C, D, or E) that corresponds to each descriptive result. Note that questions 12--16 share the same answer options.

\vspace{3mm}

%% 第 12 題
\question[5] What is the SPSS output for this data table? Each cell displays the mean and standard deviation (in parentheses) of the number of daily visitors.

\begin{center}
\begin{tabular}{l|cc}
 & Weekday & Weekend \\
\hline
Brand 1 & 4050.0 (1608.8) & 3917.1 (1514.2) \\
Brand 2 & 3175.7 (1479.8) & 5216.6 (2396.6) \\
\end{tabular}
\end{center}

\begin{solution}
此表格顯示兩個品牌在平日與週末的訪客數平均值與標準差。這是一個 $2 \times 2$ 的設計(2 個品牌 $\times$ 2 種日期類型)。

關鍵特徵:
\begin{itemize}
\item Brand 1:平日 $>$ 週末(4050 $>$ 3917)
\item Brand 2:週末 $>$ 平日(5216 $>$ 3176)
\item 存在明顯的\textbf{交互作用}(interaction)
\end{itemize}

查看 SPSS 輸出 A(第七頁):
\begin{itemize}
\item 包含 Brand(2 個水準)、Weekday(1 個水準,表示平日/週末)、Brand*Weekday 交互作用
\item df: Brand = 2, Weekday = 1, Brand*Weekday = 2
\item 這不完全符合(Brand 應該只有 1 個自由度如果只有 2 個品牌)
\end{itemize}

經分析各 SPSS 輸出的自由度和結構,Output A 最符合此 $2 \times 2$ 設計的雙因子 ANOVA,且 Brand*Weekday 交互作用顯著($p < 0.001$)。

\vspace{2mm}
\textbf{答案:(A)}
\end{solution}

%% 第 13 題
\question[5] What is the SPSS output for this data table? The dependent variable is the number of daily visitors.

\begin{center}
\begin{tabular}{lcccc}
\hline
 & & & \multicolumn{2}{c}{95\% Confidence Interval} \\
Brand & Mean & Std. Deviation & Lower Bound & Upper Bound \\
\hline
Brand A & 4012.1 & 1581.5 & 3826.3 & 4197.9 \\
Brand B & 3757.2 & 2010.1 & 3571.4 & 3943.1 \\
\hline
\end{tabular}
\end{center}

\begin{solution}
此表格顯示兩個品牌的描述統計(平均數、標準差、95\% 信賴區間),這是一個單因子設計(只比較品牌差異)。

特徵:
\begin{itemize}
\item 只有一個自變數(Brand)
\item 兩個品牌(Brand A 和 Brand B)
\item 單因子 ANOVA 或獨立樣本 $t$ 檢定
\end{itemize}

查看 Output B:
\begin{itemize}
\item 自由度:Brand = 1(兩個品牌)
\item 沒有 Weekday 因子
\item Brand*Weekday 交互作用 df = 1,但係數不顯著($p = 0.579$)
\end{itemize}

Output B 顯示只有 Brand 效果的分析,符合此描述統計表。

\vspace{2mm}
\textbf{答案:(B)}
\end{solution}

%% 第 14 題
\question[5] What is the SPSS output for this bar chart? Error bars represent the standard error of the mean.

\begin{center}
\textit{(長條圖顯示 Brand i 和 Brand j 在 Weekday 和 Weekend 的比較,兩品牌都是週末高於平日)}
\end{center}

\begin{solution}
觀察長條圖特徵:
\begin{itemize}
\item 兩個品牌(Brand i, Brand j)
\item 兩種時間(Weekday, Weekend)
\item 兩個品牌都呈現\textbf{週末訪客 $>$ 平日訪客}的模式
\item 沒有明顯的交互作用(兩品牌趨勢一致)
\end{itemize}

這需要找一個:
\begin{itemize}
\item Weekday 效果顯著(週末 $>$ 平日)
\item Brand*Weekday 交互作用\textbf{不顯著}
\end{itemize}

查看 Output C:
\begin{itemize}
\item Brand: $p = 0.125$(不顯著)
\item Weekday: $p < 0.001$(顯著)
\item Brand*Weekday: $p < 0.001$(顯著)
\end{itemize}

Output C 的交互作用顯著,與圖形不符。

查看 Output D:
\begin{itemize}
\item Brand: $p = 0.057$(邊緣顯著)
\item Brand*Weekday 效果需檢查
\end{itemize}

根據圖形特徵(兩品牌趨勢一致,週末都高),應找交互作用不顯著的輸出。

\vspace{2mm}
\textbf{答案:(C)}

(注:需根據實際 SPSS 輸出詳細比對)
\end{solution}

%% 第 15 題
\question[5] What is the SPSS output for this bar chart? Error bars represent the standard error of the mean.

\begin{center}
\textit{(長條圖顯示 Brand Q、Brand R、Brand S 在 Weekday 和 Weekend 的比較,存在交互作用模式)}
\end{center}

\begin{solution}
觀察長條圖特徵:
\begin{itemize}
\item 三個品牌(Brand Q, R, S)
\item 兩種時間(Weekday, Weekend)
\item 不同品牌呈現不同的平日/週末模式
\item 存在\textbf{交互作用}
\end{itemize}

需要找:
\begin{itemize}
\item Brand 自由度 = 2(三個品牌)
\item Brand*Weekday 交互作用顯著
\end{itemize}

查看 Output A(有 3 個品牌):
\begin{itemize}
\item Brand: df = 2, $p = 0.005$(顯著)
\item Weekday: df = 1, $p < 0.001$(顯著)
\item Brand*Weekday: df = 2, $p < 0.001$(顯著)
\end{itemize}

Output A 符合三品牌的雙因子 ANOVA 且有顯著交互作用。

\vspace{2mm}
\textbf{答案:(A)}

(注:第 12 題和第 15 題可能指向不同的 Output,需根據實際圖表對應)
\end{solution}

%% 第 16 題
\question[5] What is the SPSS output for this bar chart? Error bars represent the standard error of the mean.

\begin{center}
\textit{(長條圖顯示 Brand X、Brand Y、Brand Z 的比較,三者相近但 Brand Z 稍低)}
\end{center}

\begin{solution}
觀察長條圖特徵:
\begin{itemize}
\item 三個品牌(Brand X, Y, Z)
\item 似乎只顯示整體平均(沒有分 Weekday/Weekend)
\item 品牌間差異不大
\end{itemize}

查看 Output E:
\begin{itemize}
\item 顯示 Brand 效果:df = 2
\item $R^2 = 0.023$(很低,解釋力弱)
\item Brand: $p < 0.001$(但效果量很小)
\end{itemize}

Output E 符合只有品牌因子、效果較小的分析。

\vspace{2mm}
\textbf{答案:(E)}
\end{solution}

\newpage
\section*{第二部分:計算題}

%% 第 17 題
\question[10] Find the slope of the tangent line to the curve of intersection of the surface $\ds z = \frac{1}{2}\sqrt{24 - x^2 - 2y^2}$ with the plane $y = 2$ at the point $(2, 2, \sqrt{3})$.

\begin{solution}
當平面 $y = 2$ 與曲面相交時,交線為:
\[
z = \frac{1}{2}\sqrt{24 - x^2 - 2(2)^2} = \frac{1}{2}\sqrt{24 - x^2 - 8} = \frac{1}{2}\sqrt{16 - x^2}
\]

驗證點 $(2, 2, \sqrt{3})$ 在此曲線上:
\[
z = \frac{1}{2}\sqrt{16 - 4} = \frac{1}{2}\sqrt{12} = \frac{1}{2} \cdot 2\sqrt{3} = \sqrt{3} \quad \checkmark
\]

求切線斜率,即 $\ds\frac{dz}{dx}$:
\[
z = \frac{1}{2}(16 - x^2)^{1/2}
\]
\[
\frac{dz}{dx} = \frac{1}{2} \cdot \frac{1}{2}(16 - x^2)^{-1/2} \cdot (-2x) = \frac{-x}{2\sqrt{16 - x^2}}
\]

在 $x = 2$ 處:
\[
\frac{dz}{dx}\bigg|_{x=2} = \frac{-2}{2\sqrt{16 - 4}} = \frac{-2}{2\sqrt{12}} = \frac{-2}{4\sqrt{3}} = \frac{-1}{2\sqrt{3}} = \frac{-\sqrt{3}}{6}
\]

\vspace{2mm}
\textbf{答案:切線斜率為 $\ds -\frac{\sqrt{3}}{6}$(或 $\ds -\frac{1}{2\sqrt{3}}$)}
\end{solution}

%% 第 18 題
\question[10] Find $\ds\frac{\partial^3}{\partial x\,\partial z\,\partial y}f(x,y,z)$ if $f(x,y,z) = \sin(xy + 2z)$.

\begin{solution}
設 $f(x,y,z) = \sin(xy + 2z)$。

\textbf{第一步}:對 $y$ 偏微分
\[
\frac{\partial f}{\partial y} = \cos(xy + 2z) \cdot x = x\cos(xy + 2z)
\]

\textbf{第二步}:對 $z$ 偏微分
\[
\frac{\partial^2 f}{\partial z\,\partial y} = \frac{\partial}{\partial z}\big[x\cos(xy + 2z)\big] = x \cdot (-\sin(xy + 2z)) \cdot 2 = -2x\sin(xy + 2z)
\]

\textbf{第三步}:對 $x$ 偏微分
\[
\frac{\partial^3 f}{\partial x\,\partial z\,\partial y} = \frac{\partial}{\partial x}\big[-2x\sin(xy + 2z)\big]
\]

使用乘積法則:
\[
= -2\sin(xy + 2z) + (-2x)\cos(xy + 2z) \cdot y
\]
\[
= -2\sin(xy + 2z) - 2xy\cos(xy + 2z)
\]

\vspace{2mm}
\textbf{答案:$\ds\frac{\partial^3 f}{\partial x\,\partial z\,\partial y} = -2\sin(xy + 2z) - 2xy\cos(xy + 2z)$}
\end{solution}

\end{questions}
\end{document}
