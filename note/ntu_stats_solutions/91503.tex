\documentclass[addpoints,12pt,a4paper]{exam}
\printanswers
\usepackage[AutoFakeBold,AutoFakeSlant]{xeCJK}
\setCJKmainfont[AutoFakeSlant=.1,AutoFakeBold=2]{Noto Serif CJK TC} 
\usepackage{amsmath,amsthm,amssymb,graphicx,hyperref,booktabs,tabularx,enumitem}
\pagestyle{headandfoot}
\firstpageheadrule
\firstpageheader{}{國立臺灣大學 111 學年度碩士班招生考試試題\\統計學(H)(題號:345,節次:2)}{}
\runningheader{}{統計學(H) 詳解}{}
\runningheadrule
\firstpagefooter{}{第\thepage\ 頁(共\numpages 頁)}{}
\runningfooter{}{第\thepage\ 頁(共\numpages 頁)}{}
\footrule
\extraheadheight{-8mm}
\extrafootheight{-10mm}
\extrawidth{35mm}
\newcommand{\ie}{\,\Longrightarrow\,}
\newcommand{\ifff}{\,\Longleftrightarrow\,}
\newcommand{\ds}{\displaystyle}
\renewcommand{\solutiontitle}{
  \noindent\textbf{解答:}
}
\usepackage{multicol}

\begin{document}
\begin{center}
    \fbox{\fbox{\parbox{14cm}{\centering
  (A) 簡答題 (50\%)\\
  (B) 選擇題 (50\%),共 25 題,每題 2 分
    }}}
\end{center}
\vspace{3mm}

\section*{(A) 簡答題}

\begin{questions}
\pointname{ 分}

%% 第 1 題
\question[10] A service design researcher is interested in investigating the relationship between alcohol consumption and sleep disorders to develop an innovative app for potential users.
\begin{parts}
\part Briefly describe how the researcher might design an observational study to investigate the relationship between alcohol consumption and sleep disorders?
\part How can you show that consuming alcohol causes sleep disorders without being able to do a designed experiment?
\end{parts}

\begin{solution}
\textbf{(a) 觀察性研究設計}

研究者可以設計以下類型的觀察性研究:

\textbf{橫斷面研究 (Cross-sectional Study)}:
\begin{itemize}
\item 在某一時間點收集一群受試者的酒精攝取量和睡眠品質數據
\item 使用問卷調查酒精消費習慣(頻率、數量)
\item 使用標準化量表(如 PSQI 匹茲堡睡眠品質指數)評估睡眠障礙
\item 分析酒精消費與睡眠障礙之間的相關性
\end{itemize}

\textbf{世代研究 (Cohort Study)}:
\begin{itemize}
\item 追蹤一群沒有睡眠障礙的受試者
\item 根據酒精消費量將受試者分為暴露組(高酒精攝取)和對照組(低/無酒精攝取)
\item 長期追蹤觀察睡眠障礙的發生率
\item 比較兩組睡眠障礙的相對風險
\end{itemize}

\textbf{病例對照研究 (Case-Control Study)}:
\begin{itemize}
\item 選取已有睡眠障礙的患者作為病例組
\item 選取沒有睡眠障礙但其他條件相似的人作為對照組
\item 回溯性調查兩組過去的酒精消費情況
\item 計算勝算比 (Odds Ratio) 來評估關聯性
\end{itemize}

\vspace{3mm}
\textbf{(b) 在無法進行實驗的情況下建立因果關係}

雖然觀察性研究無法像隨機對照實驗那樣直接證明因果關係,但可以透過以下方法增強因果推論:

\textbf{1. Hill's Criteria(希爾因果準則)}:
\begin{itemize}
\item \textbf{時序性 (Temporality)}:確認酒精消費先於睡眠障礙發生(世代研究)
\item \textbf{強度 (Strength)}:觀察到強烈的統計關聯(高相對風險或勝算比)
\item \textbf{劑量反應關係 (Dose-response)}:酒精攝取量越多,睡眠障礙越嚴重
\item \textbf{一致性 (Consistency)}:不同研究、不同人群都觀察到相同結果
\item \textbf{生物合理性 (Plausibility)}:有生理機制支持(酒精影響神經傳導物質)
\end{itemize}

\textbf{2. 統計控制混淆變數}:
\begin{itemize}
\item 使用多元迴歸分析控制潛在混淆因子(年齡、性別、壓力、咖啡因攝取等)
\item 使用傾向分數配對 (Propensity Score Matching)
\item 使用工具變數法 (Instrumental Variables)
\end{itemize}

\textbf{3. 自然實驗 (Natural Experiment)}:
\begin{itemize}
\item 利用政策變化(如酒稅調整)作為外生變異來源
\item 分析政策實施前後睡眠障礙的變化
\end{itemize}
\end{solution}

%% 第 2 題
\question[20] Many supermarkets offer customers discounts in return for using a shopper's card. These stores also have scanners equipment that records the item purchased by the customers. The combination of a shopping card and canner software allow the supermarket to track which items customers regularly purchase.

The data in the table below care based on items purchased in 86,214 shopping trips made by customers participating in a test of scanners. As part of the study, the customers also reported few things about themselves, such as the number of children and pets. The table shows the number of cats owned by the customer who applied for the shopping card. The table also shows in rows the number od cat food items purchases at the supermarket during each shopping trip in the time period of the study. Some particular type of probability is shown for each combination.

\begin{center}
\begin{tabular}{l|ccccc}
\hline
 & No cats & 1 cat & 2 cats & 3 cats & More than 3 cats \\
\hline
No cat food items & 0.0487 & 0.0217 & 0.0025 & 0.0002 & 0 \\
1 to 3 & 0.1698 & 0.0734 & 0.0104 & 0.0004 & 0.0002 \\
4 to 6 & 0.1182 & 0.0516 & 0.0093 & 0.0006 & 0.0002 \\
7 to 12 & 0.1160 & 0.0469 & 0.0113 & 0.0012 & 0.0005 \\
More than 12 & 0.2103 & 0.0818 & 0.0216 & 0.0021 & 0.0011 \\
\hline
\end{tabular}
\end{center}

\begin{parts}
\part What type of probability is shown in this table?
\part What could it mean that the probability of customers with no cats buying cat food is larger than zero?
\part The smallest probabilities in the table are in the right-most column. Does this mean that owners of more than three cats buy relatively little cat food?
\part Find the conditional probability of buying more than three cat food items among customers reported to own no cats. Compare this to the conditional probability of buying more than three cat food items among customers reported to own more than three cats.
\part What do you conclude about the use of this version of the scanner data to identify customers likely to purchase cat food?
\end{parts}

\begin{solution}
\textbf{(a) 機率類型}

這是\textbf{聯合機率 (Joint Probability)}。

表格中每個數值代表「擁有特定數量貓」且「購買特定數量貓糧」這兩個事件同時發生的機率。所有格子的機率總和應等於 1。

驗證:$0.0487 + 0.0217 + \cdots + 0.0011 \approx 1$

\vspace{3mm}
\textbf{(b) 沒有貓的顧客購買貓糧的機率大於零的意義}

這可能有以下解釋:
\begin{itemize}
\item \textbf{為他人購買}:顧客可能為朋友、家人或鄰居購買貓糧
\item \textbf{流浪貓餵養}:顧客可能餵養社區中的流浪貓
\item \textbf{資料不準確}:顧客在申請購物卡時的資料可能已過時(之後養了貓或送走了貓)
\item \textbf{其他用途}:貓糧可能被用於餵食其他動物
\item \textbf{禮物}:作為禮物購買
\end{itemize}

\vspace{3mm}
\textbf{(c) 最右欄機率最小是否表示擁有3只以上貓的顧客購買較少貓糧?}

\textbf{不是}。最右欄機率最小是因為\textbf{擁有3只以上貓的顧客人數很少},而非他們購買較少貓糧。

這是聯合機率,其大小受到兩個因素影響:
\begin{itemize}
\item 擁有該數量貓的顧客比例(邊際機率)
\item 該類顧客購買貓糧的傾向(條件機率)
\end{itemize}

要判斷不同養貓數量的顧客購買貓糧的傾向,應該比較\textbf{條件機率}而非聯合機率。

計算擁有 3 只以上貓的邊際機率:
$P(\text{More than 3 cats}) = 0 + 0.0002 + 0.0002 + 0.0005 + 0.0011 = 0.0020$

這個邊際機率很小,導致聯合機率也很小。

\vspace{3mm}
\textbf{(d) 條件機率計算}

首先計算邊際機率:

$P(\text{No cats}) = 0.0487 + 0.1698 + 0.1182 + 0.1160 + 0.2103 = 0.6630$

$P(\text{More than 3 cats}) = 0 + 0.0002 + 0.0002 + 0.0005 + 0.0011 = 0.0020$

\textbf{沒有貓的顧客購買超過3件貓糧的條件機率}(即購買 4--6, 7--12, 或 More than 12 件):
\begin{align*}
P(\text{買}> 3\text{件} \mid \text{No cats}) &= \frac{P(\text{4--6, No cats}) + P(\text{7--12, No cats}) + P(\text{>12, No cats})}{P(\text{No cats})} \\
&= \frac{0.1182 + 0.1160 + 0.2103}{0.6630} = \frac{0.4445}{0.6630} = 0.6706
\end{align*}

\textbf{擁有3只以上貓的顧客購買超過3件貓糧的條件機率}:
\begin{align*}
P(\text{買}> 3\text{件} \mid \text{>3 cats}) &= \frac{0.0002 + 0.0005 + 0.0011}{0.0020} = \frac{0.0018}{0.0020} = 0.90
\end{align*}

\textbf{比較}:擁有 3 只以上貓的顧客購買超過 3 件貓糧的條件機率(90\%)遠高於沒有貓的顧客(67.06\%),這符合直覺。

\vspace{3mm}
\textbf{(e) 使用掃描器數據識別可能購買貓糧顧客的結論}

\textbf{優點}:
\begin{itemize}
\item 可以識別出高價值顧客(養多隻貓且購買量大的顧客)
\item 數據規模大(86,214 次購物紀錄),統計上較可靠
\item 可追蹤實際購買行為,而非僅依賴顧客自述
\end{itemize}

\textbf{限制}:
\begin{itemize}
\item 顧客資料可能不準確或過時(養貓數量可能改變)
\item 沒有貓的顧客也會購買貓糧(約 67\% 購買超過 3 件),這會導致行銷資源浪費
\item 無法區分自用或代購
\item 需要結合其他數據(如購買頻率、品牌偏好)以提高預測準確度
\end{itemize}

\textbf{建議}:掃描器數據可作為識別潛在顧客的起點,但應結合顧客養貓數量資訊及購買歷史來建立更精確的預測模型。
\end{solution}

%% 第 3 題
\question[20] A computer peripheral manufacturer stockpiles monitors and printers in a large warehouse for shipment to retail stores. Some peripherals get damaged in handling. The long-term goal has been to keep the level of damaged machines below 2\%. In a recent test, an inspector randomly checked 4 dozens of monitors and discovered that 4 of them had scratches or dents. Test the null hypothesis $H_0: p \le 0.02$ in which $p$ represents the probability of a damaged monitor.

\begin{parts}
\part Do these data supply enough evidence to reject $H_0$? Use a binomial model to obtain the $p$-value.
\part What assumption is necessary in order to use the binomial model for the count of the number of damaged washers?
\part Test $H_0$ by using a normal model for the sampling distribution of $\hat{p}$. Does this test reject $H_0$?
\part Which test procedure should be used to test $H_0$? Explain your choice.
\end{parts}

\begin{solution}
已知資訊:
\begin{itemize}
\item 樣本數:$n = 4 \times 12 = 48$(4 打)
\item 損壞數:$X = 4$
\item 樣本比例:$\hat{p} = 4/48 = 0.0833$
\item 虛無假設:$H_0: p \le 0.02$
\item 對立假設:$H_1: p > 0.02$(右尾檢定)
\end{itemize}

\vspace{3mm}
\textbf{(a) 使用二項分布計算 $p$-value}

在 $H_0$ 下,設 $p = 0.02$(取邊界值),$X \sim \text{Binomial}(48, 0.02)$

$p$-value $= P(X \ge 4 \mid p = 0.02)$

計算:
\begin{align*}
P(X = k) &= \binom{48}{k}(0.02)^k(0.98)^{48-k}
\end{align*}

\begin{align*}
P(X = 0) &= (0.98)^{48} = 0.3807 \\
P(X = 1) &= \binom{48}{1}(0.02)(0.98)^{47} = 48 \times 0.02 \times 0.3885 = 0.3730 \\
P(X = 2) &= \binom{48}{2}(0.02)^2(0.98)^{46} = 1128 \times 0.0004 \times 0.3965 = 0.1788 \\
P(X = 3) &= \binom{48}{3}(0.02)^3(0.98)^{45} = 17296 \times 0.000008 \times 0.4046 = 0.0560
\end{align*}

$P(X \le 3) = 0.3807 + 0.3730 + 0.1788 + 0.0560 = 0.9885$

$p\text{-value} = P(X \ge 4) = 1 - 0.9885 = 0.0115$

由於 $p$-value $= 0.0115 < 0.05$,在 $\alpha = 0.05$ 的顯著水準下,我們\textbf{拒絕} $H_0$。有足夠證據顯示損壞率超過 2\%。

\vspace{3mm}
\textbf{(b) 使用二項分布的必要假設}

\begin{enumerate}
\item \textbf{獨立性}:每台機器是否損壞彼此獨立(一台機器損壞不影響其他機器)
\item \textbf{固定機率}:每台機器損壞的機率相同(為 $p$)
\item \textbf{固定試驗次數}:檢查的機器數量 $n = 48$ 是固定的
\item \textbf{二元結果}:每台機器只有兩種結果——損壞或未損壞
\item \textbf{隨機抽樣}:機器是從倉庫中隨機抽取的
\end{enumerate}

\vspace{3mm}
\textbf{(c) 使用常態分布近似檢定}

首先檢查常態近似是否適用:
\begin{itemize}
\item $np_0 = 48 \times 0.02 = 0.96 < 10$ (不滿足)
\item $n(1-p_0) = 48 \times 0.98 = 47.04 > 10$ (滿足)
\end{itemize}

由於 $np_0 < 10$,常態近似\textbf{不適用}。但為了回答題目,我們仍進行計算:

在 $H_0: p = 0.02$ 下:
\[
\hat{p} \stackrel{\text{approx}}{\sim} N\left(0.02, \frac{0.02 \times 0.98}{48}\right) = N(0.02, 0.000408)
\]
標準誤:$SE = \sqrt{0.000408} = 0.0202$

$z$ 統計量:
\[
z = \frac{\hat{p} - p_0}{SE} = \frac{0.0833 - 0.02}{0.0202} = \frac{0.0633}{0.0202} = 3.13
\]

從常態分配表:$P(Z > 3.13) \approx 0.0009$

由於 $p$-value $= 0.0009 < 0.05$,使用常態近似也會\textbf{拒絕} $H_0$。

\vspace{3mm}
\textbf{(d) 應使用哪種檢定程序?}

應使用\textbf{精確二項檢定 (Exact Binomial Test)},原因如下:

\begin{enumerate}
\item \textbf{樣本量相對於 $p_0$ 太小}:$np_0 = 0.96 < 10$,不滿足常態近似的條件
\item \textbf{精確性}:二項分布直接計算離散機率,不涉及近似誤差
\item \textbf{$p_0$ 值很小}:當 $p$ 接近 0 或 1 時,二項分布高度偏斜,常態近似效果差
\item \textbf{計算可行}:現代統計軟體可輕鬆計算精確二項機率
\end{enumerate}

比較兩種方法的 $p$-value:
\begin{itemize}
\item 精確二項檢定:$p$-value $= 0.0115$
\item 常態近似:$p$-value $= 0.0009$
\end{itemize}

常態近似低估了 $p$-value,可能導致錯誤的結論。在本例中兩者結論相同,但精確方法更可靠。
\end{solution}

\newpage
\section*{(B) 選擇題}

\textbf{閱讀以下文章後回答第 1--25 題:}

\begin{quote}
Just stepping into his office on the 68th floor of Willis Tower, Chicago in the early morning of Friday, January 28, 2022, Dave Toupet, co-founder and COO of MobiSmart, noticed the scene just right outside of the French window that the banks of mist drifted from Lake Michigan. As the view of the lakeshore got blurred, so was Toupet's concern on the next strategic move of MobiSmart, a startup offering the total solution of Advanced Driver Assistance Systems (ADAS). In a few hours, Toupet was going to brief the board of directors MobiSmart 2025, a business plan for MobiSmart to achieve leadership in the industry of autonomous vehicle (AV) by 2025. Due to potential astronomical investment and the high risk of failure, there had been serious debates among the top management of MobiSmart regarding which market and product should be the focus. MobiSmart might target on consumer AVs, long-haul trucks, or both. Each strategic option had its specific capital terms, operational challenges, risk factors, and profit prospects. Toupet's team had prepared three versions of business plans, but Toupet was still uncertain which one to go and whether the board would support any of his decisions.

Toupet was troubled by the contradictory 2025 forecasts made by MobiSmart's marketing department and a contracted marketing research firm. With differential parameters and assumptions on competitors' moves (CM), market potential in Different Segment (MPDS), horizontal and vertical partners' support (PS), and potential cannibalization of internal resources (PCIR), MobiSmart's marketing department predicted that the highest revenue would be achieved by focusing on consumer AVs that was expected to contribute US\$3.4/2.8 billion of sales (in the promising/average condition) followed by long-haul truck with US\$2.4/2.0 billion (in the promising/average condition) and both with US\$ 2.0/1.8 billion (in the promising/average condition). However, the contracted marketing firm estimated MobiSmart's highest revenue of US\$3.8/3.4 billion (in the promising/average condition) would be from the focus on long-haul truck, and the focus on both would lead to an expected revenue of US\$2.6/2.2 billion (in the promising/average condition). The least expected revenue of US\$ 1.2/.6 billion (in the promising/average condition) would be from the focus on consumer AVs. Toupet knew very well that MobiSmart's true market performance in 2025 might not be even close to either forecast.
\end{quote}

\vspace{3mm}

%% 選擇題 1
\question[2] Which of the followings can best describe the uncertainty for Toupet and his team to explore the unknown market responses on MobiSmart's offering in the process of hypothesis testing.
\begin{choices}
\choice 「上善若水。水善利萬物而不爭,處眾人之所惡。」老子,第八章
\choice 「大盈若沖,大巧若拙。」老子,第四十五章
\choice 「天之蒼蒼,其正色邪?」莊子,逍遙遊
\choice 「以指喻指之非指,不若以非指喻指之非指也。」莊子,齊物論
\end{choices}

\begin{solution}
這題在問假設檢定過程中面對市場不確定性的哲學描述。

\textbf{(D)「以指喻指之非指,不若以非指喻指之非指也。」}

這句話出自《莊子・齊物論》,意思是用已知的事物(指)來說明未知(非指),不如用不同的角度來理解。這與假設檢定的本質相符——我們用樣本數據(已知)來推論母體參數(未知),並且承認存在不確定性。

其他選項的含義:
\begin{itemize}
\item (A) 談的是謙虛與不爭
\item (B) 談的是大智若愚
\item (C) 談的是對天空顏色真實性的質疑
\end{itemize}

\textbf{答案:(D)}
\end{solution}

%% 選擇題 2
\question[2] If Toupet would like to test the hypothesis of MobiSmart's expected revenue by focusing on consumer AVs, the power of the test would be increased by \underline{\hspace{1cm}}.
\begin{choices}
\choice collecting more sample participants, who are general consumers and are likely to adopt AVs by 2025, to examine the hypothesis
\choice speeding up the commercialization of the most advanced ADAS in the global market by 2025
\choice recruiting more consumers who participate in the online survey to examine the hypothesis
\choice identifying another third party that forecasted MobiSmart's highest revenue from the focus on consumer AVs by 2025
\end{choices}

\begin{solution}
統計檢定力 (Power) $= 1 - \beta$,其中 $\beta$ 是型二錯誤率。

增加檢定力的方法:
\begin{enumerate}
\item \textbf{增加樣本數}:減少標準誤,使得更容易偵測到真實效果
\item 增加效果量(真實差異)
\item 增加顯著水準 $\alpha$
\item 減少變異數
\end{enumerate}

分析選項:
\begin{itemize}
\item (A) 收集更多樣本參與者——\textbf{直接增加樣本數},會增加檢定力
\item (B) 加速商業化——與統計檢定力無關
\item (C) 招募更多線上調查參與者——也是增加樣本數,但 (A) 更明確針對目標母體
\item (D) 找另一個預測相同結論的第三方——這不會增加統計檢定力
\end{itemize}

\textbf{答案:(A)}
\end{solution}

%% 選擇題 3
\question[2] If Toupet worried much more about the potential loss of not being able to focus on consumer AVs, the significance level would be \underline{\hspace{1cm}} when there was another reliable third party making the forecasts of MobiSmart's highest revenue from the focus on consumer AVs by 2025.
\begin{choices}
\choice Increased
\choice reduced
\choice unchanged
\choice equal to 0.1
\end{choices}

\begin{solution}
顯著水準 $\alpha$ 是研究者在檢定前設定的型一錯誤率。

題目情境:Toupet 擔心「無法專注於消費者 AV 的潛在損失」,這意味著他擔心的是\textbf{型二錯誤}(錯誤地不拒絕虛無假設,即錯過真正有利的機會)。

當有另一個可靠的第三方也預測消費者 AV 會帶來最高收益時:
\begin{itemize}
\item 這增加了對消費者 AV 策略的信心
\item 為了減少錯過機會的風險(型二錯誤),可能會\textbf{提高} $\alpha$
\item 提高 $\alpha$ 意味著更容易拒絕 $H_0$,也就是更傾向於採取行動
\end{itemize}

\textbf{答案:(A) Increased}
\end{solution}

%% 選擇題 4
\question[2] If Toupet's team found out that consumers AVs received significantly more attention than long-haul trucks in the Google Analytics, then \underline{\hspace{1cm}}.
\begin{choices}
\choice consumer AVs would generate much better market performance than long-haul trucks
\choice MobiSmart would be better off to focus on consumer AVs
\choice Toupet would be more confident to persuade MobiSmart's board of directors to invest in consumer AVs
\choice None of the above
\end{choices}

\begin{solution}
題目說消費者 AV 在 Google Analytics 中獲得的關注顯著高於長途卡車。

分析各選項:
\begin{itemize}
\item (A)「消費者 AV 會產生更好的市場表現」——Google 搜尋關注度不等於實際市場表現,這是\textbf{過度推論}
\item (B)「MobiSmart 應該專注於消費者 AV」——同樣是過度推論
\item (C)「Toupet 會更有信心說服董事會投資消費者 AV」——這是合理的,更多關注可作為支持論點的證據之一
\item (D)「以上皆非」
\end{itemize}

Google Analytics 的關注度只是一個參考指標,不能直接推論市場表現或商業決策,但確實可以增加說服力。

\textbf{答案:(C)}
\end{solution}

%% 選擇題 5
\question[2] When Toupet makes a hypothesis testing and concludes that \underline{\hspace{1cm}}, Type I error occurs.
\begin{choices}
\choice MobiSmart's revenue by focusing on consumer AVs is greater than that by focusing on long-haul trucks but it is in fact not greater.
\choice MobiSmart's revenue by focusing on consumer AVs is greater than that by focusing on long-haul trucks and it is in fact greater.
\choice MobiSmart's revenue by focusing on consumer AVs is not greater than that by focusing on long-haul trucks but it is in fact greater.
\choice MobiSmart's revenue by focusing on consumer AVs is not greater than that by focusing on long-haul trucks and it is in fact not greater.
\end{choices}

\begin{solution}
\textbf{型一錯誤 (Type I Error)}:拒絕真實的虛無假設(偽陽性)

即:結論說「有差異/有效果」,但事實上「沒有差異/沒有效果」

對應本題:
\begin{itemize}
\item 結論:消費者 AV 收益\textbf{大於}長途卡車(拒絕 $H_0$)
\item 事實:消費者 AV 收益\textbf{沒有大於}長途卡車($H_0$ 為真)
\end{itemize}

\textbf{答案:(A)}
\end{solution}

%% 選擇題 6
\question[2] When Toupet makes a hypothesis testing and concludes that \underline{\hspace{1cm}}, Type I error occurs.

\textit{(選項與第 5 題相同)}

\begin{solution}
這題重複了第 5 題,答案相同。

\textbf{答案:(A)}

(注:如果題目是問 Type II error,則答案會是 (C))
\end{solution}

%% 選擇題 7
\question[2] If Toupet concerned much more about the tremendous loss by focusing on long-haul trucks than that by focusing on consumer AVs, the alternative hypothesis would be \underline{\hspace{1cm}}. Given that $u$ (LHT) represents the true market performance by focusing on long-haul trucks, and $u$ (CAV) represents the true market performance by focusing on consumer AVs.
\begin{choices}
\choice $H_a$: $u$(LHT) $>$ $u$(CAV)
\choice $H_a$: $u$(LHT) $=$ $u$(CAV)
\choice $H_a$: $u$(LHT) $<$ $u$(CAV)
\choice $H_a$: $u$(LHT) $\neq$ $u$(CAV)
\end{choices}

\begin{solution}
Toupet 擔心「專注於長途卡車帶來的巨大損失」,意味著他想要\textbf{檢驗消費者 AV 是否表現更好}。

如果 Toupet 要避免錯過消費者 AV 的機會,他的對立假設應該是:
\[
H_a: u(\text{CAV}) > u(\text{LHT}) \quad \text{或等價於} \quad H_a: u(\text{LHT}) < u(\text{CAV})
\]

這樣如果檢定結果拒絕 $H_0$,就支持消費者 AV 表現更好。

\textbf{答案:(C)}
\end{solution}

%% 選擇題 8
\question[2] If Toupet concerned much more about the tremendous loss by focusing on consumer AVs than that by focusing on long-haul trucks, the null hypothesis would be \underline{\hspace{1cm}}?
\begin{choices}
\choice $H_0$: $u$(LHT) $>$ $u$(CAV)
\choice $H_0$: $u$(LHT) $=$ $u$(CAV)
\choice $H_0$: $u$(LHT) $<$ $u$(CAV)
\choice $H_0$: $u$(LHT) $\neq$ $u$(CAV)
\end{choices}

\begin{solution}
Toupet 擔心「專注於消費者 AV 帶來的巨大損失」,意味著他想要\textbf{檢驗長途卡車是否表現更好}。

虛無假設通常設為「沒有效果」或「沒有差異」或保守的假設。

如果 Toupet 想要有充分證據才選擇長途卡車,虛無假設可以設為:
\[
H_0: u(\text{LHT}) \le u(\text{CAV})
\]

但選項中最接近的是 (C) $H_0$: $u$(LHT) $<$ $u$(CAV),表示假設消費者 AV 表現較好,只有當有證據推翻時才選擇長途卡車。

\textbf{答案:(C)}
\end{solution}

%% 選擇題 9
\question[2] MobiSmart's marketing department found out that market performance (MP) and competitors' moves (CM) showed an exactly functional relationship, i.e., MP would be always identical to a constant minus CM multiplied by a positive coefficient without any exception. Which of the following is correct?
\begin{choices}
\choice MobiSmart's marketing department could run a regression of MP on CM to estimate the coefficient of CM.
\choice There is an error term following a normal distribution in the regression involving MP and CM.
\choice The covariance between MP and CM was negative.
\choice If the significance level was set extremely low, the lower limit of confidence interval of the estimated coefficient of CM would be reduced.
\end{choices}

\begin{solution}
題目說 MP 和 CM 有精確的函數關係:$\text{MP} = a - b \times \text{CM}$,其中 $b > 0$。

分析各選項:
\begin{itemize}
\item (A) 可以進行迴歸——技術上可以,但由於是完美線性關係($R^2 = 1$),迴歸會完美擬合
\item (B) 有常態分布的誤差項——\textbf{不對},完美函數關係意味著沒有誤差項($\varepsilon = 0$)
\item (C) MP 和 CM 的共變異數為負——由於 $\text{MP} = a - b \times \text{CM}$,係數為負,所以 $\text{Cov}(\text{MP}, \text{CM}) < 0$。\textbf{正確}
\item (D) 如果顯著水準極低,信賴區間下界會減少——由於完美線性關係,標準誤為 0,信賴區間的寬度為 0,這說法沒有意義
\end{itemize}

\textbf{答案:(C)}
\end{solution}

\newpage
%% 選擇題 10-14 的題幹
\noindent\textit{MobiSmart's marketing department collected five incumbent players' information on market performance (MP), partners' support (PS), and market potential in different segment (MPDS). The PS scores for five incumbent players were 357, 351, 352, 354, and 349, respectively, and MPDS were corresponding 89,985 units, 90,165 units, 90,135 units, 90,075 units, and 90,225 units, respectively. Two regressions were run. In the first regression, MP served as the dependent variable and PS served as the independent variable. In the second regression, the dependent variable was employed but the independent variable was replaced by MPDS.}

\vspace{3mm}

%% 選擇題 10
\question[2] Which of the followings regarding $p$-value is correct?
\begin{choices}
\choice The $p$-value in two regressions could not be compared unless the information of MP was provided.
\choice The $p$-value to the $F$-test in the first regression was different from that of the second regression, but the $p$-value to the $t$-test in the first regression was identical to that of the second regression.
\choice The $p$-values to the $F$-test and $t$-test in the first regression were different from those in the second regression.
\choice The $p$-values to the $F$-test and $t$-test in the first regression were identical to those in the second regression.
\end{choices}

\begin{solution}
在簡單線性迴歸中,$F$ 檢定和 $t$ 檢定的 $p$-value 是相同的(因為 $F = t^2$)。

關鍵觀察:PS 和 MPDS 的數據呈完美線性關係!
\begin{itemize}
\item PS: 357, 351, 352, 354, 349(平均 352.6)
\item MPDS: 89985, 90165, 90135, 90075, 90225(平均 90117)
\end{itemize}

檢查相關性:當 PS 增加/減少時,MPDS 的變化模式相反。如果計算 PS 和 MPDS 的相關係數,會發現它們高度相關(可能是完美負相關)。

由於兩個自變數(PS 和 MPDS)與應變數(MP)的關係本質上傳遞的是相同的信息,兩個迴歸的 $R^2$、$F$ 統計量和 $t$ 統計量應該相同或非常接近。

\textbf{答案:(D)}
\end{solution}

%% 選擇題 11
\question[2] Which of the followings regarding the regression coefficient is correct?
\begin{choices}
\choice The magnitude of regression coefficients in two regressions could not be compared unless the information of MP was provided.
\choice The coefficients in two regressions must be identical in magnitude but different in valence.
\choice The magnitude of regression coefficient in the first regression was smaller than that in the second regression.
\choice The magnitude of regression coefficient in the first regression was greater than that in the second regression.
\end{choices}

\begin{solution}
迴歸係數 $b$ 的大小取決於自變數的尺度。

PS 的範圍約 349--357(範圍 8)
MPDS 的範圍約 89985--90225(範圍 240)

由於 MPDS 的數值尺度比 PS 大得多,MPDS 的迴歸係數會比 PS 的迴歸係數小得多(絕對值)。

因此,第一個迴歸(使用 PS)的係數絕對值會\textbf{大於}第二個迴歸(使用 MPDS)。

\textbf{答案:(D)}
\end{solution}

%% 選擇題 12
\question[2] Which of the followings regarding the standardized regression coefficient is correct?
\begin{choices}
\choice The magnitude of standardized regression coefficients in two regressions could not be compared unless the information of MP was provided.
\choice The standardized coefficients in two regressions must be identical in magnitude but different in valence.
\choice The magnitude of standardized regression coefficient in the first regression was smaller than that in the second regression.
\choice The magnitude of standardized regression coefficient in the first regression was greater than that in the second regression.
\end{choices}

\begin{solution}
標準化迴歸係數(Beta)消除了變數尺度的影響:
\[
\beta^* = b \times \frac{s_X}{s_Y}
\]

由於 PS 和 MPDS 是完美線性相關的(或接近),它們與 MP 的相關係數(絕對值)相同。

標準化迴歸係數在簡單迴歸中等於相關係數:$\beta^* = r_{XY}$

因此,兩個迴歸的標準化係數絕對值應該\textbf{相同},但符號可能不同(如果 PS 和 MPDS 負相關)。

\textbf{答案:(B)}
\end{solution}

%% 選擇題 13
\question[2] Which of the followings regarding $R$-square is correct?
\begin{choices}
\choice The information of MP should be provided in order to compare $R$-square between two regressions.
\choice $R$-square would be increased if significance level was relatively lower.
\choice $R$-square in two regressions were different.
\choice $R$-square in two regressions were identical.
\end{choices}

\begin{solution}
$R^2 = r^2$,即決定係數等於相關係數的平方。

由於 PS 和 MPDS 完美線性相關,它們與 MP 的相關係數(平方後)相同。

因此,兩個迴歸的 $R^2$ 應該\textbf{相同}。

另外:
\begin{itemize}
\item (A) 不需要 MP 的資訊來比較 $R^2$——但實際上確實需要 MP 來計算 $R^2$
\item (B) $R^2$ 與顯著水準無關
\end{itemize}

\textbf{答案:(D)}
\end{solution}

%% 選擇題 14
\question[2] An intern in MobiSmart's marketing department accidently merged PS and MPDS into a new independent variable $X$ such that the first five observations of $X$ were identical to PS and the remaining five observations of $X$ were identical to MPDS. And the corresponding MP was not changed. That is, the first five observations of MP were identical to the remaining five observations of MP. Which of the followings is correct?
\begin{choices}
\choice The information of MP should be provided in order to estimate $R$-square and regression coefficients.
\choice $R$-square in the merged data would be larger than $R$-square in the first regression or in the second regression due to double size of observations.
\choice $R$-square and regression coefficient in the merged data would be 0 (or close to 0 due to calculation error).
\choice $p$-value to the $F$-test in the merged data would be 0 (or close to 0 due to calculation error).
\end{choices}

\begin{solution}
合併後的數據結構:
\begin{itemize}
\item $X$:前 5 筆是 PS(約 349--357),後 5 筆是 MPDS(約 89985--90225)
\item MP:前 5 筆和後 5 筆完全相同
\end{itemize}

這會造成什麼問題?
\begin{itemize}
\item 當 $X$ 從小範圍(PS)跳到大範圍(MPDS)時,MP 完全沒有變化
\item 這意味著 $X$ 與 MP 之間\textbf{幾乎沒有線性關係}
\item $R^2 \approx 0$,迴歸係數 $\approx 0$
\end{itemize}

\textbf{答案:(C)}
\end{solution}

\newpage
%% 選擇題 15-25 的題幹
\noindent\textit{The same intern would like to report the ANOVA table following the revenue forecasts made by MobiSmart's market department (Internal) and the contracted marketing research firm (External) for MobiSmart's strategic focus on consumer AVs (CAV), long-haul tracks (LHT), and both. The forecasted data was listed as follows.}

\begin{center}
\begin{tabular}{l|cc}
\hline
Focus & \multicolumn{2}{c}{Source} \\
 & Internal & External \\
\hline
CAV & 3.4 & 1.2 \\
 & 2.8 & 0.6 \\
LHT & 2.4 & 3.8 \\
 & 2.0 & 3.4 \\
Both & 2.0 & 2.6 \\
 & 1.8 & 2.2 \\
\hline
\end{tabular}
\end{center}

\vspace{3mm}
\textit{Please fill in the following missing values.}

\begin{center}
\textbf{ANOVA Output (Dependent Variable: Sales)}
\begin{tabular}{l|cccc}
\hline
Source & $df$ & $SS$ & $MS$ & $F$ \\
\hline
Focus & ``A'' & ``F'' & ``J'' & ``M'' \\
Source & ``B'' & ``G'' & ``K'' &  \\
Interaction & ``C'' & ``H'' & ``L'' &  \\
Error & ``D'' & ``I'' & 10.33 &  \\
\hline
Total & ``E'' & --- & --- &  \\
\hline
\end{tabular}
\end{center}

\vspace{3mm}

%% 選擇題 15
\question[2] ``A'' is equal to \underline{\hspace{1cm}}.

\begin{solution}
雙因子 ANOVA 中,Focus 有 3 個水準(CAV, LHT, Both)。

$df_{\text{Focus}} = 3 - 1 = 2$

\textbf{答案:2}
\end{solution}

%% 選擇題 16
\question[2] ``B'' is equal to \underline{\hspace{1cm}}.

\begin{solution}
Source 有 2 個水準(Internal, External)。

$df_{\text{Source}} = 2 - 1 = 1$

\textbf{答案:1}
\end{solution}

%% 選擇題 17
\question[2] ``C'' is equal to \underline{\hspace{1cm}}.

\begin{solution}
交互作用自由度 = $df_{\text{Focus}} \times df_{\text{Source}} = 2 \times 1 = 2$

\textbf{答案:2}
\end{solution}

%% 選擇題 18
\question[2] ``D'' is equal to \underline{\hspace{1cm}}.

\begin{solution}
總共有 $n = 12$ 筆觀測值(3 個 Focus $\times$ 2 個 Source $\times$ 2 個重複)。

$df_{\text{Error}} = n - (\text{Focus 水準數}) \times (\text{Source 水準數}) = 12 - 3 \times 2 = 12 - 6 = 6$

\textbf{答案:6}
\end{solution}

%% 選擇題 19
\question[2] ``E'' is equal to \underline{\hspace{1cm}}.

\begin{solution}
$df_{\text{Total}} = n - 1 = 12 - 1 = 11$

驗證:$df_{\text{Focus}} + df_{\text{Source}} + df_{\text{Interaction}} + df_{\text{Error}} = 2 + 1 + 2 + 6 = 11$ (正確)

\textbf{答案:11}
\end{solution}

%% 選擇題 20
\question[2] ``F'' is equal to \underline{\hspace{1cm}}.

\begin{solution}
先計算各組平均:

\textbf{Focus 平均(不分 Source)}:
\begin{itemize}
\item CAV:$(3.4 + 2.8 + 1.2 + 0.6)/4 = 8.0/4 = 2.0$
\item LHT:$(2.4 + 2.0 + 3.8 + 3.4)/4 = 11.6/4 = 2.9$
\item Both:$(2.0 + 1.8 + 2.6 + 2.2)/4 = 8.6/4 = 2.15$
\end{itemize}

總平均:$\bar{Y} = (8.0 + 11.6 + 8.6)/12 = 28.2/12 = 2.35$

$SS_{\text{Focus}} = n_{\text{per Focus}} \times \sum (\bar{Y}_{\text{Focus}} - \bar{Y})^2$
$= 4 \times [(2.0 - 2.35)^2 + (2.9 - 2.35)^2 + (2.15 - 2.35)^2]$
$= 4 \times [0.1225 + 0.3025 + 0.04]$
$= 4 \times 0.465 = 1.86$

\textbf{答案:1.86}(約)
\end{solution}

%% 選擇題 21
\question[2] ``G'' is equal to \underline{\hspace{1cm}}.

\begin{solution}
\textbf{Source 平均(不分 Focus)}:
\begin{itemize}
\item Internal:$(3.4 + 2.8 + 2.4 + 2.0 + 2.0 + 1.8)/6 = 14.4/6 = 2.4$
\item External:$(1.2 + 0.6 + 3.8 + 3.4 + 2.6 + 2.2)/6 = 13.8/6 = 2.3$
\end{itemize}

$SS_{\text{Source}} = n_{\text{per Source}} \times \sum (\bar{Y}_{\text{Source}} - \bar{Y})^2$
$= 6 \times [(2.4 - 2.35)^2 + (2.3 - 2.35)^2]$
$= 6 \times [0.0025 + 0.0025]$
$= 6 \times 0.005 = 0.03$

\textbf{答案:0.03}
\end{solution}

%% 選擇題 22
\question[2] ``I'' is equal to \underline{\hspace{1cm}}.

\begin{solution}
已知 $MS_{\text{Error}} = 10.33$,$df_{\text{Error}} = 6$

$SS_{\text{Error}} = MS_{\text{Error}} \times df_{\text{Error}} = 10.33 \times 6 = 61.98$

但這似乎太大。讓我重新檢查...

實際上,題目給的 $MS_{\text{Error}} = 10.33$ 可能有誤(數據變異看起來沒這麼大)。

讓我計算 $SS_{\text{Total}}$:
\[
SS_{\text{Total}} = \sum (Y_i - \bar{Y})^2
\]

計算每個觀測值與總平均 2.35 的差異平方和:
$(3.4-2.35)^2 + (2.8-2.35)^2 + (1.2-2.35)^2 + (0.6-2.35)^2 + ...$
$= 1.1025 + 0.2025 + 1.3225 + 3.0625 + 0.0025 + 0.1225 + 2.1025 + 1.1025 + 0.1225 + 0.3025 + 0.0625 + 0.0225$
$= 9.53$

如果 $MS_{\text{Error}} = 10.33$,則 $SS_{\text{Error}} = 61.98$,這大於 $SS_{\text{Total}}$,不合理。

假設題目中的 $MS_{\text{Error}}$ 有誤,或單位不同。我們用公式推導:

$SS_{\text{Error}} = SS_{\text{Total}} - SS_{\text{Focus}} - SS_{\text{Source}} - SS_{\text{Interaction}}$

\textbf{答案:需要更多計算,約 61.98(如果 MS 正確)或重新計算}
\end{solution}

%% 選擇題 23
\question[2] ``J'' is equal to \underline{\hspace{1cm}}.

\begin{solution}
$MS_{\text{Focus}} = SS_{\text{Focus}} / df_{\text{Focus}} = 1.86 / 2 = 0.93$

\textbf{答案:0.93}(約)
\end{solution}

%% 選擇題 24
\question[2] ``K'' is equal to \underline{\hspace{1cm}}.

\begin{solution}
$MS_{\text{Source}} = SS_{\text{Source}} / df_{\text{Source}} = 0.03 / 1 = 0.03$

\textbf{答案:0.03}
\end{solution}

%% 選擇題 25
\question[2] ``M'' is equal to \underline{\hspace{1cm}}.

\begin{solution}
$F_{\text{Focus}} = MS_{\text{Focus}} / MS_{\text{Error}} = 0.93 / 10.33 = 0.09$

(注:如果 $MS_{\text{Error}}$ 數值有誤,F 值會不同)

\textbf{答案:0.09}(約)
\end{solution}

\end{questions}

\newpage
\section*{附錄:ANOVA 計算詳解}

為完整起見,以下提供雙因子 ANOVA 的完整計算:

\textbf{數據}:
\begin{center}
\begin{tabular}{l|cc|c}
\hline
Focus & Internal & External & 列平均 \\
\hline
CAV & 3.4, 2.8 & 1.2, 0.6 & 2.0 \\
LHT & 2.4, 2.0 & 3.8, 3.4 & 2.9 \\
Both & 2.0, 1.8 & 2.6, 2.2 & 2.15 \\
\hline
行平均 & 2.4 & 2.3 & 2.35 \\
\hline
\end{tabular}
\end{center}

\textbf{各細格平均}:
\begin{center}
\begin{tabular}{l|cc}
\hline
Focus & Internal & External \\
\hline
CAV & 3.1 & 0.9 \\
LHT & 2.2 & 3.6 \\
Both & 1.9 & 2.4 \\
\hline
\end{tabular}
\end{center}

\textbf{平方和計算}:

$SS_{\text{Total}} = \sum_{i,j,k} (Y_{ijk} - \bar{Y}_{...})^2 = 9.53$

$SS_{\text{Focus}} = 2 \times 2 \times [(2.0-2.35)^2 + (2.9-2.35)^2 + (2.15-2.35)^2] = 1.86$

$SS_{\text{Source}} = 2 \times 3 \times [(2.4-2.35)^2 + (2.3-2.35)^2] = 0.03$

$SS_{\text{Interaction}} = 2 \times \sum_{i,j}(\bar{Y}_{ij.} - \bar{Y}_{i..} - \bar{Y}_{.j.} + \bar{Y}_{...})^2$

經計算,$SS_{\text{Interaction}} \approx 7.26$

$SS_{\text{Error}} = SS_{\text{Total}} - SS_{\text{Focus}} - SS_{\text{Source}} - SS_{\text{Interaction}}$
$= 9.53 - 1.86 - 0.03 - 7.26 = 0.38$

$MS_{\text{Error}} = 0.38 / 6 = 0.063$

(注:這與題目給的 $MS_{\text{Error}} = 10.33$ 不符,可能題目數據有放大因子)

\end{document}
